There are many ways to design and implement applications for a distributed cluster, a set of
computers (typically labeled as \emph{computing nodes}) that have their own independent processors
and memory, and are connected together with a computer network. In the context of
\gls{hpc}, such distributed clusters are called \emph{supercomputers}, and one of
their distinguishing features is that all the computers of the cluster reside within one physical
location, and they are connected with a very high-speed and low-latency network connection. There
are also many other kinds of distributed clusters, such as data centers or cloud-based distributed
systems, however this thesis focuses almost exclusively on \gls{hpc} systems and
supercomputers.

Distributed applications are implemented using various \emph{programming models}, which should be able
to provide a way to efficiently utilize the available computing resources of the cluster, and to
allow expressing communication patterns that enable the individual cluster nodes to cooperate and
exchange data. Communication between nodes is crucial, as that is what allows distributed clusters
to offer unparalleled performance by distributing the computational load amongst multiple computers
and thus achieving speedup through parallelization.

There are many different programming models and tools for creating distributed applications, and it
can be challenging to understand how do they relate to one another. It is especially difficult to
navigate the area of \emph{task-based programming}, which is the main focus of this thesis, because common
umbrella terms like \emph{task}, \emph{task graph}, \emph{\gls{dag}},
\emph{workflow} or \emph{pipeline} are being used very liberally, and they can
represent vastly different concepts. It is thus possible to encounter two programming models or
tools that both claim to use ``task-based programming'', even though they might have very little in
common.

This chapter provides a broad overview of the most popular approaches for implementing distributed
and parallel applications, with the focus on \gls{hpc} use-cases. It describes both
various programming models and also state-of-the-art tools that leverage them. It categorizes these
tools based on several properties, and gradually concretizes which niches of this diverse area are
most relevant for the topic of this thesis.

\section{Parallel programming models}
This section describes the most important representatives of programming models that are used in
the world of supercomputing. It divides the programming models into two broad categories; models
that express parallelization and network communication explicitly, and models that do so
implicitly.

\subsection*{Explicit parallelization}
One way to design distributed applications is to leverage programming models that express the
parallelization of computation and the exchange of data between nodes explicitly. This has been the
predominant way of creating \gls{hpc} software for many years, and it is still very
popular today~\cite{mpiusagestudy1,mpiusagestudy2,mpiusagestudy3}. Below are a few examples of these explicit approaches.

\begin{description}
	\item[Message passing] has historically been the most popular method for implementing \gls{hpc} software. In
		this model, a distributed computation is performed by a set of processes with separate memory
		address spaces that are running on independent computing nodes. The processes cooperate together to
		solve complex problemss by exchanging network messages (hence the term \emph{message passing}).
		Message passing applications are commonly implemented using the
		\gls{spmd}~\cite{spmd} computational model, where the implementation logic
		of all the processes participating in the computation is embedded within a single program.

		The most popular representative of this programming model is the
		\gls{mpi}~\cite{mpi} framework, which is used by a large amount of
		existing \gls{hpc} applications~\cite{mpiusagestudy2}. It defines a set of
		communication primitives, operators and data types, which can be used to perform computations,
		exchange data and synchronize progress between either two (\emph{point-to-point communication}) or multiple
		(\emph{collective communication}) processes running on distributed nodes. \Autoref{lst:mpi-example} shows a
		simple \gls{mpi} program which is designed to execute on two (potentially distributed)
		processes. The first process sends a number to the second process, which waits until that number is
		received, and then prints it to standard output. Notice how both network communication and
		synchronization is expressed explicitly, by calling the \texttt{MPI\_Send} and
		\texttt{MPI\_Recv} functions. We can also see the \gls{spmd} paradigm in practice,
		because the code for both processes is interleaved within the same program.

		\begin{listing}[h]
			\caption{Example of a simple \gls{mpi} program implemented in \texttt{C}}
			\label{lst:mpi-example}
			\begin{minted}[fontsize=\small]{c}
	#include <mpi.h>
	#include <stdio.h>

	int main() {
		MPI_Init(NULL, NULL);

		// Find out the ID of this process
		int process_id;
		MPI_Comm_rank(MPI_COMM_WORLD, &process_id);

		if (process_id == 0) {
			// Send one integer to process 1
			int value = 42;
			MPI_Send(&value, 1, MPI_INT, 1, 0, MPI_COMM_WORLD);
		} else if (process_id == 1) {
			// Receive one integer from process 0
			int value = 0;
			MPI_Recv(&value, 1, MPI_INT, 0, 0, MPI_COMM_WORLD, MPI_STATUS_IGNORE);
			printf("Process 1 received number %d from process 0\n", value);
		}

		MPI_Finalize();

		return 0;
	}
				  \end{minted}
		\end{listing}

	\item[\gls{pgas}~\cite{pgas}] is a relatively similar programming model, which also often employs the \gls{spmd}
		paradigm. Where it differs from message passing is in the way it expresses communication between
		processes. Message passing processes share their memory by communicating with other processes,
		while PGAS provides an abstraction of a shared memory address space and allows processes to
		communicate through it\footnote{To paraphrase the famous ``Do not communicate by sharing memory; instead, share memory by communicating'' quote coined by the \texttt{Go} programming language.}. \gls{pgas} provides an illusion of a
		global memory address space that is available to processes that participate in the communication,
		which makes it slightly less explicit in terms of expressing the communication patterns within the
		program, because it translates certain memory operations into network messages on behalf of the
		programmer.

		\gls{pgas} programs also often employ \emph{one-sided communication} techniques, such as
		\gls{rdma}, which allows a process to directly read or write a region of memory from
		the address space of a different process.

	\item[Shared-memory multiprocessing] is an approach that focuses on the parallelization within a single computing node, by leveraging
		multithreading to achieve speedup. In the area of \gls{hpc}, it is common to use the
		\gls{openmp}~\cite{openmp} framework to implement multithreaded applications.
		Apart from providing interfaces for parallelizing code, synchronizing threads through barriers or
		various locks, or performing atomic operations, it is also able to offload computation to various
		accelerators (like a \gls{gpu}) attached to the node. \gls{openmp} can be
		used together with the two previously mentioned programming models (it is often combined especially
		with \gls{mpi}~\cite{hybrid_openmp_mpi}), in order to achieve parallelization both
		intra-node (via multithreading) and inter-node (via network communication).

		\gls{openmp} does not only offer an \gls{api}, but it can also be integrated
		directly within a compiler, e.g.\ in \gls{gcc} for programs written in the
		\texttt{C} or \texttt{C++} programming languages. This enables it to provide
		source code annotations (called \emph{pragmas}), which allow the programmer to parallelize
		a region of code with very little effort. An example of this can be seen in \Autoref{lst:openmp-example},
		where a loop is parallelized simply by adding a single annotation to the source code.

		\begin{listing}
			\caption{Example of a simple \gls{openmp} annotation}
			\label{lst:openmp-example}
			\begin{minted}[fontsize=\small]{c}
void compute_parallel(int* items, int count) {
    // This loop is executed in parallel
    #pragma omp parallel for
    for (int i = 0; i < count; i++) {
        items[i] = compute(i);
    }
}
        \end{minted}
		\end{listing}
\end{description}

These explicit programming models share a lot of desirable properties. They give their users a lot
of control over the exchange of data between individual cores and distributed nodes, which allows
creating very performant programs. Having the option to explicitly describe how will the individual
cores and nodes cooperate also enables expressing arbitrarily complex parallelization patterns and
data distribution strategies. However, in order to fully exploit the performance potential of
explicit parallelization, the programmer must have advanced knowledge of the \gls{cpu}
or \gls{gpu} hardware micro-architecture~\cite{intel_developer_manual} and the memory model
of the used programming language~\cite{cpp11_standard}.

Even though explicitly parallelized programs can be incredibly efficient, implementing correct
applications using them is notoriously difficult. Multithreaded and distributed programs are highly
concurrent, which makes it easy to introduce various programming errors, such as deadlocks, race
conditions or data races. Especially for distributed programs, debugging such issues can be
incredibly challenging. Furthermore, programs that leverage explicitly parallel programming models
are typically implemented in languages such as \texttt{C} or \texttt{C++},
which are infamous for making it difficult to write correct, memory-safe programs without memory
errors and undefined behaviour~\cite{memory_safety_report}. Memory safety issues are even more
problematic in heavily concurrent programs, which further increases the difficulty and decreases
the speed of developing correct distributed programs.

Apart from correctness, using explicit communication interfaces can also lead to overdependence on
a specific state of available computational resources. For example, \gls{mpi} programs
typically assume a fixed number of processes participating in the computation, and
\gls{mpi} itself struggles with situations where some of the processes participating
in the computation crash or disappear, which infamously makes it challenging to implement fully
fault-tolerant \gls{mpi} programs~\cite{fault_tolerant_mpi}.

\subsection*{Implicit parallelization}
Since it can take a lot of effort to implement a correct and efficient distributed program using
explicitly parallel models, it would be unreasonable to expect that all users who want to leverage
\gls{hpc} resources to execute their experiments will ``roll up their sleeves'' and
spend months implementing an explicitly parallel \texttt{C++} program that uses
\gls{mpi} and \gls{openmp}. In fact, with scientific experiments becoming
more and more complex each year, in most cases it would not even be feasible to develop custom
(explicitly parallel) code for them from scratch. Instead, high-performance parallelized primitives
implemented by specialized performance engineers~\cite{dace} are moving into libraries
and frameworks, such as GROMACS~\cite{gromacs,gromacs_mpi} or TensorFlow~\cite{tensorflow,horovod}, that
still leverage technologies like \gls{mpi} or \gls{openmp} internally, but
they do not necessarily expose them to their end users.

This allows users of \gls{hpc} systems and scientists to focus on their problem
domain, as their responsibility shifts from implementing communication and parallelization
techniques by hand to describing high-level computational workflows using \emph{implicity parallel}
programming models that are able to automatically derive the communication and parallelization
structure from the program description. With an implicitly parallel model, the emphasis moves from
\emph{how} to perform a distributed or parallelized computation (which is the essence
of the explicit models) to \emph{what} should be computed and how are the individual
computational steps related to each other, which is usually the main aspect that users actually
want to focus on. Implicitly parallel programming models often combine aspects of two programming
paradigms, declarative and imperative. The possible parallelization opportunities are expressed
using declarative approaches, for example using various \glspl{dsl}, while sequential
parts of code usually remains imperative.

The primary benefit of the implicitly parallel approaches is that they make it easy to define a
computation that can be automatically parallelized, without forcing the user to think about how
exactly will the parallelization and network communication be performed. Execution frameworks are
then able to ingest programs implemented with implicitly parallel models and automatically execute
them on a parallel machine or even a distributed cluster in a parallel fashion, thus making it much
easier for the user to leverage available hardware resources. Implicit models are easier to use
than the explicit models, and they facilitate rapid prototyping of parallelized programs.

On the other hand, the main disadvantage of these models is the lack of control of how exactly is
parallelization performed. Therefore, programs implemented using them might not be able to achieve
the same performance as explicitly parallelized programs.

There are various models that support implicit parallelization, for example stencil
computations~\cite{stencil} or automatically parallelized functional
languages~\cite{parallel_haskell}. But by far the most popular are the many tools and models based
on tasks. Since task-based programming is a core topic of this thesis, the rest of the thesis will
focus exclusively on programming models that leverage tasks. In particular, the following section
will categorize these models based on several properties, and describe representative tools that
implement this paradigm.

\section{Computational workflows}
In recent years, it became very popular to define scientific computations running on distributed
and \gls{hpc} clusters using \emph{task-based programming models}~\cite{pegasus,workflows1,workflows_at_scale}. These
models allow their users to describe the high-level structure of their computations using
\emph{computational workflows} (also called \emph{pipelines} or \emph{task graphs}\footnote{These three terms will be used interchangeably in this thesis.}). A computational workflow is a \gls{dag} of
\emph{tasks}, atomic and independent computational blocks with separate inputs and
outputs that can depend on one another, which can be executed in a self-contained way. Such
workflows can naturally express diverse scientific experiments, which typically need to execute and
compose many independent steps with dependencies between themselves (for example data
preprocessing, postprocessing and analysis computations, various simulations, etc.). They are also
very flexible and easy to use, which is what makes them popular.

Since task-based programming models are implicitly parallel, their users do not have to
imperatively specify how should their computation be parallelized, or when and how should data be
exchanged between distributed nodes. They merely declaratively describe the individual parts of the
program that can theoretically be executed in parallel (the tasks) and then pass the task graph to
a dedicated execution tool that executes the tasks, typically on a distributed cluster. Since the
program is represented with an explicit graph, the execution tool can effectively analyze its
properties (or even optimize the structure of the graph) in an automated way, and extract the
available parallelism from it without requiring the user to explicitly define how should the
program be parallelized.

It is important to note that from the perspective of a task execution tool, each task is opaque.
The tool knows how to execute it, but it typically does not have any further knowledge of the inner
structure of the task. Therefore, the only parallelization opportunities that can be extracted by
the tool have to be expressed by the structure of the task graph. A task graph containing a single
task is thus not very useful on its own. The individual tasks can of course also be internally
parallel, however this parallelization is not provided automatically by the task execution tool.
Tasks often achieve internal parallelism using shared memory multiprocessing, for example using the
\gls{openmp} framework.

Since task-based programming models are quite popular, there are hundreds of different tools and
technologies that leverage them. It can be challenging to understand how do these tools relate to
one another, because umbrella terms like \emph{task}, \emph{task graph},
\emph{\gls{dag}}, \emph{workflow} or \emph{pipeline} can represent vastly
different concepts in different contexts. For example, the term task is used for many unrelated
concepts in computer science, from an execution context in the Linux kernel, through a block of
code that can be executed on a separate thread by \gls{openmp}, to a program that is a
part of a complex distributed computational workflow. It is thus possible to encounter two
programming models or tools that both claim to use ``task-based programming'', even though they
might have very little in common. The rest of this section will thus categorize existing
state-of-the-art tools that use task-based programming models based on several properties, to put
them into a broader context, and it will also gradually specify which niches of this diverse area
are most relevant for the topic of this thesis.

\subsection*{Batch vs stream processing}
One of the most distinguishing properties that divides distributed task processing tools into two
broad categories is the approach used to trigger the execution of the workflow.
\emph{Stream processing} tools are designed to execute continuously, and react to external events
that can arrive asynchronously and at irregular intervals, while typically focusing on low latency.
A streaming computational workflow is executed every time a specific event arrives. The workflow
can then can analyze the event, and generate some output, which is then e.g.\ persisted in a
database or sent to another stream processing system. As an example, a web application can stream
its logs, which are being generated dynamically as users visit the website, to a stream processing
tool, which then analyzes the logs and extracts information out of them in real-time. Popular
representatives of stream processing are for example Apache Flink~\cite{flink} or Apache
Kafka~\cite{kafka}. Streaming-based tools can also be implemented using the
\emph{dataflow} programming model~\cite{dataflow,timely_dataflow}.

In contrast, \emph{batch processing} tools are designed to perform a specific computation over a
set (batch) of input data that is fully available before the computation starts, while focusing
primarily on maximal throughput. Such workflows are usually triggered manually by a user once all
the data is prepared, and the workflow stops executing once it has processed all of its inputs. In
certain cases, parts of the computation can be generated dynamically, while the workflow is already
executing. For example, iterative workflows perform a certain computation repeatedly in a loop,
until some condition in met. Typical representatives of batch processing tools are for example
\dask{}~\cite{dask}, SnakeMake~\cite{snakemake} or
Luigi~\cite{sciluigi}.

Streaming processing is common in the area of cloud computing, and is useful especially for
analyzing data that is being generated in real-time. It is not very common in the world of
supercomputing though, because \gls{hpc} hardware resources are typically ephemeral,
and they are not designed to be available for a single user at all times, which limits their
usefulness for handling real-time events that occur at unpredictable times. On the other hand,
batch processing workflows are a better fit for \gls{hpc} clusters, since their
execution time is bounded by the size of their input, which is often known in advance, and thus
they can be more easily mapped to ephemeral, time-limited allocations of \gls{hpc}
hardware resources. This thesis thus exclusively focuses on batch processing.

\subsection*{Task graph constraints}
Even though basically all workflow processing tools are designed to execute a
\gls{dag} of tasks, not all of them support arbitrary task graphs. Some tools use
programming models designed for specialized use-cases, which allows them to offer very high-level
and easy-to-use \glspl{dsl} and \glspl{api} that are designed to perform a
specific set of things well, but that do not allow expressing fully general task graphs.

An example of such constrained approach is the Bulk synchronous parallel~\cite{bulkparallel1}
model, which models a distributed computation with a series of steps that are executed in order.
Within each step, a specific computation is performed in parallel, potentially on multipled
distributed nodes. At the end of each step, the nodes can exchange data amongst themselves, and
they are synchronized with a global barrier, to ensure that there are no cyclic communication
patterns in the computation. Even though this model does not use fully arbitrary computational
graphs, it is still possible to express many algorithms with it~\cite{bulkparallel2}.

A popular instance of this model that has gained a lot of popularity in the area of distributed
computing is MapReduce~\cite{mapreduce}. Its goal is to allow parallel processing of large
amounts of data on a distributed cluster in a fault-tolerant way, while providing a very simple
interface for the user. It does that by structuring the computation into three high-level
operations, which correspond to individual bulk synchronous parallel steps:
\begin{enumerate}
	\item A \emph{map} operation (provided by the user) is performed on the input data. This
	      operation performs data transformation and filtering, associates some form of a
	      \emph{key} to each data item and produces a key-value tuple.
	\item A \emph{shuffle} operation (implemented by a MapReduce framework) redistributes the tuples
	      amongst a set of distributed nodes, based on the key of each tuple, so that tuples with the same
	      key will end up at the same node.
	\item A \emph{reduce} operation (provided by the user) is performed on each node. The reduction
	      typically performs some aggregation operation (such as sum) on batches of tuples, where each batch
	      contains tuples with the same key.
\end{enumerate}

\begin{listing}
	\caption{Example of a word count MapReduce implementation in Python}
	\label{lst:wordcount-example}
	\begin{minted}[fontsize=\small, tabsize=4]{python}
def word_count(context):
	file = context.textFile("shakespeare.txt")
	counts = file.flatMap(lambda line: line.split(" ")) \
		.map(lambda word: (word, 1)) \
		.reduceByKey(lambda x, y: x + y)
	output = counts.collect()
	print(output)
	\end{minted}
\end{listing}

\Autoref{lst:wordcount-example} shows a simple Python program that computes the frequency of
individual words in a body of text\footnote{This computation is commonly known as \emph{word count}.} using a popular implementation of
MapReduce called Apache Spark~\cite{spark}. Even though the programmer only needs to
provide a few very simple \emph{map} and \emph{reduce} operations, this
short program can be executed on a distributed cluster and potentially handle very large amounts of
data. Note that multiple \emph{map} and \emph{reduce} operations can be
combined in a single program, which can also be observed in this example. While this program is
implemented in the Python programming language, the most important parts of the computation are
expressed using a few very specific implicitly parallel operations (\emph{map} and
\emph{reduce}), therefore it is possible to consider this to be a sort of
\gls{dsl} for expressing distributed computations.

The primary benefit of this approach is that it is very easy to define a computation that can be
automatically parallelized and distributed. Furthermore, the ``layered'' shape of task graphs
produced by MapReduce programs facilitates implementation of fault tolerance, because there are
clearly delimited checkpoints where the intermediate results of the computation can be persisted to
disk (at the end of each \emph{map} or \emph{reduce} operation), so that
they can be later restored in case of failure. This is only possible because the tools that execute
MapReduce programs have additional knowledge and invariants about the shape of the task graph, due
to its constrained nature.

However, this also the main limitation of this model. If a computation cannot be naturally
expressed using the \emph{map} and \emph{reduce} operations, it can be
challenging, or even impossible, to implement it with the MapReduce paradigm. In particular,
MapReduce assumes forward data-flow, where data is being sent through the computational pipeline in
one direction. This can be problematic for iterative computations that have to execute repeatedly,
e.g.\ until some condition is reached, and which typically use the output of a previous iteration
as an input for the next iteration, thus creating a loop in the flow of data.

Another potential disadvantage, which is shared by most implicitly parallel models, is that
MapReduce programs might not be able to achieve performance comparable to programs implemented
using explicit parallelism. This is especially problematic for certain implementations of MapReduce
that are known to be very inefficient, such as Apache Hadoop~\cite{hadoop}. In general, when
executing distributed computations, it is always a good idea to first make sure that it is not
possible to solve the same problem faster with a single-threaded program running on
consumer-grade hardware~\cite{cost}.

While MapReduce and similar models are very useful for computations that have a regular
structure, \gls{hpc} scientific workflows can be quite complex, and it is not always possible to
express them using such a constrained programming model. We will thus further only consider tools
and models that allow executing arbitrarily shaped task graphs.

\section{Task granularity}
Another important aspect is ...

- low/high/ultra high

Granularity is also related to the fact if the tasks can be efficiently distributed at all.

%TODO
It is used in various technologies and frameworks, ranging from fine-grained tasks that execute a
single function or just a handful of instructions~\cite{starpu,openmp} to coarse-grained task
workflows that execute binaries which can run for hours or even days~\cite{dask, snakemake, nextflow}. This
thesis primarily focuses on the latter type of task graphs, which represent very general
computations (either functions or binaries) with various levels of granularity, that are intended
to be distributed among multiple nodes of a distributed cluster.

- architecture?

%Taxonomy - https://link.springer.com/article/10.1007/s11227-018-2238-4 - HyperShell
%(https://hyper-shell.readthedocs.io/en/latest/index.html) - flux - parsl - legate - pygion - dask -
%ray - snakemake - merlin - airflow - luigi - autosubmit - airflow -
%https://materialsproject.github.io/fireworks/ - pydra - (take more from references)
