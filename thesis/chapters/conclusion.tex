The main goals of this thesis were to describe and analyze existing challenges of task graph
execution on \gls{hpc} systems, with the main focus on efficiency and usage
ergonomics, and design and implement approaches for alleviating them. The main issues affecting
\gls{hpc} task graph execution were described in~\Autoref{ch:sota}, which
has identified several areas that can cause problems when executing task graphs on
modern heterogeneous \gls{hpc} clusters, and outlined motivation for work
presented in the rest of the thesis.

The following two chapters focused on the performance and efficiency aspects of executing task
graphs.~\Autoref{ch:estee} has introduced \estee{}, a simulation
environment designed for prototyping task schedulers and benchmarking various aspects of task graph
execution. We have used \estee{} to perform a comprehensive study of several task
scheduling algorithms and examine several hypotheses of the effect of various factors on the
performance of the scheduler. Through our experiments, we have found that the
\emph{b-level} heuristic combined with work-stealing provides a solid baseline
scheduling algorithm that achieves good performance across various kinds of task graphs. This
scheduling approach was then used for implementing efficient scheduling in
the~\rsds{} and~\hyperqueue{} task runtimes. Our results have also
indicated that while important, task scheduling might not be the most crucial factor that affects
the performance of task graph execution in certain scenarios, and that even a completely random
scheduler might be competitive.

\Autoref{ch:rsds} focused on the performance bottlenecks of a real-world task runtime
\dask{} in a non-simulated
environment, to leverage our insights gained from the previous scheduler experiments. Our analysis
has shown that \dask{} is severely limited by its implementation
characteristics, moreso than its scheduler. In particular, it is limited by its choice of Python as
an implementation language, which can introduce massive overhead particularly for
\gls{hpc} use-cases. We have proposed and implemented~\rsds{},
a backwards-compatible implementation of the \dask{} server written in Rust,
which was carefully designed to minimize runtime overhead. Our experiments demonstrated that such
an optimized server implementation can improve the scalability and end-to-end performance of
\dask{} workflows by several times, even though it uses a simpler scheduling
algorithm.

\Autoref{ch:hyperqueue} combined the gained performance insights with a focus on
the ergonomic aspects of designing and executing task graphs on supercomputers. It introduced a
general meta-scheduling approach designed primarily to avoid complications caused by the need to
interact with \gls{hpc} allocation managers and also to support the complex nature
of modern heterogeneous clusters. The described method completely separates task submission from
computational resource provisioning, which enables fully dynamic and automatic load balancing even
across different allocations. It is combined with a resource management system that allows tasks to
express complex resource requirements in a generic way, while still making it trivial to provision
computational resources.

The described approach has been implemented in \hyperqueue{}, an
\gls{hpc}-optimized task runtime that was designed to make it simple to execute
task graphs in the presence of allocation managers. The most important features of \hyperqueue{}
have been described in this chapter, such as its support for complex resource
management and multi-node tasks, fault-tolerant task execution and automatic submission of
allocations. The overhead and scalability of this task runtime were also evaluated on several
benchmarks designed to push it to its performance limits. The results of these
experiments indicate that it does not introduce significant overhead and can be used to efficiently
scale task graphs to a large amount of computational resources.

\hyperqueue{} is a culmination of the work described in this thesis; it was
designed as a unified solution for removing bottlenecks and challenges from task graph execution on
supercomputers. The following list describes how \hyperqueue{} deals with challenges
introduced by~\Autoref{ch:sota}, and also which improvements could be made to it as future
work to further improve its ability to provide ergonomic and efficient task graph execution on
\gls{hpc} clusters.
\begin{description}[wide=0pt]
	\item[Interaction with allocation manager] The used meta-scheduling approach removes the need for the workflow author to think about mapping
		tasks to allocations or dealing with various allocation limits. And thanks to the automatic
		allocator, users do not even need to submit any allocations by themselves. The automatic allocator
		could be extended with task duration or allocation start time predictions in the future, to
		improve its decisions on when to actually submit allocations.
	\item[Cluster heterogeneity] \hyperqueue{} provides comprehensive support for heterogeneous
		clusters by enabling tasks to specify arbitrary resource requirements, and by matching these
		requirements with resources provided by workers. In addition to basic resources, support for
		\gls{numa}, \gls{gpu} core pinning and time requests, it also
		supports complex resource requirements in the form of resource variants and fractional resources,
		which can deal with the most complex resource management scenarios. Workers also provide automatic
		detection of available resources, which further improves the ergonomics of running workflows with
		heterogeneous clusters.
	\item[Data transfers] \hyperqueue{} does not currently support direct data transfers
		between tasks, i.e.\ enabling tasks to pass their outputs as inputs to dependent tasks through
		network communication. This limits its applicability (or at least its ergonomics) in scenarios that
		need to exchange large amounts of outputs between tasks. Implementing data transfers would expand
		the set of workflows that can be naturally executed with it, and it could also improve performance
		by avoiding filesystem \gls{io}.

		For use-cases that are limited by \gls{io} bandwidth due to creating too many
		standard (error) output files on distributed filesystems, \hyperqueue{} offers
		\emph{output streaming}, which is able to avoid filesystem limitations by streaming task output
		to the server and storing it in a single file.
	\item[Fault tolerance] \hyperqueue{} offers fault tolerance by default; tasks that do not
		finish computing successfully due to reasons outside their control are automatically rescheduled to
		a different worker, without requiring any manual user intervention. Workers are designed to be
		ephemeral; because they are usually executed in relatively short-running allocations, their
		failures are thus handled gracefully. The server itself is also fault-tolerant and can reload its
		task database after being restarted.
	\item[Multi-node tasks] \hyperqueue{} provides built-in support for multi-node tasks, and
		can even combine them with standard single-node tasks within the same task graph. It also provides
		basic integration with common multi-node technologies like \gls{mpi} to make their
		usage in multi-node tasks easier. Multi-node tasks could be further extended to be more granular,
		so that a multi-node task would not necessarily require its whole node for execution, and for some
		use-cases it would also be very useful to have the ability to combine multi-node tasks with data
		transfers.
	\item[Scalability] Experiments presented in~\Autoref{hq:evaluation} demonstrate that \hyperqueue{} is
		able to scale to \gls{hpc}-level workflows and that it does not introduce
		significant overhead over executing tasks manually. Its performance could be further improved by
		integrating \gls{hpc}-specific technologies, such as
		InfiniBand~\cite{infiniband} or \gls{mpi}, to speed-up network
		communication or filesystem \gls{io}, or by adding support for stateful task
		environments. These could help avoid the need to create a separate Linux process for each executed
		task, which can have a non-trivial overhead on certain \gls{hpc} clusters, as was
		demonstrated by our experiments. %TODO: reword if Ada manages to finish dynamic jobs in time? :)
	\item[Iterative computation] While it is possible to express task graphs that need to be created in a dynamic fashion or that
		leverage iterative computation using the Python \gls{api}, it is currently not
		very ergonomic for the users, as separate \hq{} jobs need to be submitted in
		this case. As a future extension, \hyperqueue{} could allow adding tasks to existing
		jobs fully dynamically, which would make it much simpler to express use-cases where the structure
		of the task graph is not fully known before its execution starts.
	\item[Deployment] In terms of ease-of-deployment, \hyperqueue{} is essentially optimal; it is
		distributed as a single binary that runs fully in user-space and that does not have any
		dependencies. Its users thus do not have to install any libraries or deploy complex services on the
		target supercomputer (which can be very difficult or even impossible in some cases) in order to use
		it. It is also simple to make it completely self-contained by statically linking a
		\emph{C} standard library into it, which would remove its only runtime dependency
		on the \texttt{glibc} standard library implementation. Its Python
		\gls{api}, which is an optional component, is distributed as a standard Python
		package that can be easily installed using standard Python package management tools, on a variety
		of Python versions.
\end{description}

To summarize, this thesis makes the following contributions:
\begin{itemize}
	\item It introduces a task graph simulator for evaluating the quality of task schedulers under various
	      conditions and provides an extensive evaluation of several scheduling algorithms using this
	      simulator.
	\item It provides an analysis of the performance bottlenecks of a state-of-the-art task runtime
	      \dask{} and introduces an alternative implementation of its server that provides
	      significant performance improvements in \gls{hpc} scenarios while retaining
	      backwards compatibility.
	\item Primarily, it proposes an approach for effortless execution of task graphs in the presence of
	      \gls{hpc} allocation managers and provides an implementation of this approach in
	      the \hyperqueue{} task runtime.
\end{itemize}

\section{Impact}
This final section summarizes the impact that \rsds{} and
\hyperqueue{} had on several real-world projects.

\subsection*{\rsds{}}
After we had an indication that \rsds{} could be leveraged to improve the
efficiency of existing \dask{} workflows, we contacted the authors and
maintainers of \dask{} and presented them our research and
\rsds{}. Although replacing their server implementation or switching from Python
to a different implementation language was not a feasible approach for them, some of the ideas
implemented in \rsds{} have since been adapted in the \dask{}
project. This helped alleviate some of the bottlenecks that we have discovered with our
experiments, and improved the performance of \dask{} in
general\footnoteurl{https://github.com/dask/distributed/issues/3139}\footnoteurl{https://github.com/dask/distributed/issues/3783}\footnoteurl{https://github.com/dask/distributed/issues/3872}.

\subsection*{\hyperqueue{}}
\hyperqueue{} has already been adopted in several projects, and it is also being
actively used by various researchers and teams across several European \gls{hpc}
centers. It has been proposed as one of the designated ways for executing
\gls{hpc} computations in several supercomputing centers and clusters, such as
LUMI~\cite{it4i-lumi}, CSC-FI~\cite{puhti-hq,puhti-hq-2},
IT4Innovations~\cite{it4i-hq} or CINECA~\cite{cineca}.

In addition to being used directly by \gls{hpc} users and scientists,
\hyperqueue{} can also be integrated as a general task execution system into other
tools, thanks to its sophisticated resource management and task scheduling capabilities. This has
been leveraged by several workflow management systems that have integrated
\hyperqueue{} as one of their task execution backends, such as
Aiida~\cite{aiida-hq}, NextFlow~\cite{nextflow-hq},
UM-Bridge~\cite{umbridge}, StreamFlow~\cite{streamflow-hq},
ERT~\cite{ert} or HEAppE~\cite{heappe}.

\hyperqueue{} is also used in various research projects. As an example,
scientists from the Czech Academy of Sciences use it to execute simulations that analyze data from
the ATLAS~\cite{atlas} experiment performed at CERN. Thanks to
\hyperqueue{}, they were able to improve the achieved hardware utilization on the
IT4Innovations Karolina~\cite{karolina} supercomputer by 30\%, which saves them tens of
hundreds of node hours per year~\cite{cern-hq}. \hyperqueue{} has also
been used to execute workflows in several projects funded by the European Union, such as
EVEREST~\cite{everest}, ACROSS~\cite{across},
EXA4MIND~\cite{exa4mind} and MaX~\cite{max}. It was especially useful
for the LIGATE~\cite{ligate} project, where it was used to implement several
\gls{md} workflows that were executed using hundreds of thousands of
\gls{cpu} and \gls{gpu} hours on the most powerful European
supercomputers.

Given the use-cases mentioned above, I believe that the practical applicability of the proposed
task graph execution design has been demonstrated, and that this thesis has thus achieved the goals
that it originally set out to. I am confident that \hyperqueue{} provides a tangible
benefit in terms of ergonomic and efficient execution of task graphs on supercomputers, and that it
resolves many of the challenges that have been described extensively in this thesis through its
\gls{hpc}-driven design. I hope that \hyperqueue{} will eventually
see even more widespread usage in the \gls{hpc} community.
