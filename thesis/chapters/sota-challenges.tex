There is a large body of tools designed for executing arbitrary task graphs on diverse computing
platforms, ranging from consumer-grade laptops, through cloud deployments, to distributed and
\gls{hpc} clusters. They are known under various terms, such as workflow
management systems, job managers, distributed job schedulers or orchestrators. We will use the term
\emph{task runtime} for all such task execution tools in this thesis, as has already been
discussed in the previous chapter.

Examples of such task runtimes include e.g.\ \dask{}~\cite{dask},
Parsl~\cite{parsl}, Ray~\cite{ray},
PyCOMPSs~\cite{pycompss}, HyperLoom~\cite{hyperloom},
Pydra~\cite{pydra}, Snakemake~\cite{snakemake},
SciLuigi~\cite{sciluigi}, Merlin~\cite{merlin},
Autosubmit~\cite{autosubmit}, NextFlow~\cite{nextflow},
StreamFlow~\cite{streamflow}, AiiDA~\cite{aiida},
FireWorks~\cite{fireworks} or Apache Airflow~\cite{airfow}. Each task
runtime defines its own instance of a task-based programming model, and has a different set of
trade-offs in areas such as performance and scalability, fault tolerance, data provenance,
ease-of-use, ease-of-deployment and others.

Since this thesis focuses on the ergonomics and performance aspects of executing task graphs on
\gls{hpc} clusters, this chapter discusses various challenges, bottlenecks and
shortcomings that arise when the existing task runtimes are used for this specific use-case. It
introduces a unique set of constraints, stemming both from the inherent complexity and
idiosyncrasies of \gls{hpc} software and hardware, and also from the sheer
computational scale required to efficiently utilize \gls{hpc} resources. In
addition, we will also mention various desired properties and features of task runtimes that could
alleviate the mentioned challenges.

\section{Allocation manager}
\label{sec:allocation-manager}
Users of \gls{hpc} clusters are not typically allowed to directly perform
arbitrary computations on the computational nodes (machines designed to perform expensive
computations) of the cluster. Instead, they connect to machines usually called
\emph{login nodes}, from which they have to enqueue their desired computation into a queue
handled by a submission system that manages the hardware resources of the cluster. We will use the
term \emph{allocation manager} for these submission systems and the term
\emph{allocation}\footnote{The term \emph{job} is also commonly used for the concept of
\gls{hpc} computational requests. However, this term will be used for a different concept described later in the thesis, therefore we use \emph{allocation} instead.} for a computational request submitted by a
user into these managers.

Allocation managers are needed to provide fair access to the resources of the cluster, because
\gls{hpc} clusters are typically used by many people at the same time. Without
centralized management, hardware resources could be inadvertently shared amongst multiple users at
once, which could have undesirable performance and security implications, and could lead to
oversubscription. Furthermore, usage of these clusters is usually metered. Users can typically only
use a certain amount of resources assigned to their \emph{computational project}, and when their
resources run out, they have to ask (or pay) for more resources. Allocation managers thus also
implement user and project accounting, so that there is a clear historical record of how many
resources were consumed by individual users of the cluster.

The majority of \gls{hpc} clusters~\cite{slurm-schedmd} use one of the two
most popular allocation managers, Slurm~\cite{slurm} and
\gls{pbs}~\cite{pbs} (or some of its many derivatives,
such as TORQUE~\cite{torque} or OpenPBS~\cite{openpbs}). Unless otherwise noted, Slurm will be used
as a default representative of allocation managers in the rest of this thesis.

The following process describes how computations are typically executed on
\gls{hpc} clusters that use an allocation manager:

\begin{enumerate}
    \item The user enqueues a computational request (allocation) into the manager from a login node. The
    request typically has to specify at least how many nodes should be allocated and what is the
    maximum duration of the computation (usually labeled as \emph{wall-time}), after which
    the computation will be forcibly stopped by the manager. It can also contain additional
    configuration, such as what kinds of nodes should be allocated or what is the priority of the
    computation.
    \item The allocation manager puts the request into a queue and schedules it to be executed at some time
    in the future. Since users submit their allocations into the manager continuously, each allocation
    has different properties and priorities, and it is not possible to exactly predict for how long
    will an allocation run, the schedule can be quite dynamic. In other words, users can wait seconds,
    minutes, hours or even days before their allocation starts to execute, and it is not always easy to
    predict how long will this duration be.
    \item Once the allocation gets to the front of the queue, and there are enough resources available, the
    manager provisions the requested amount of hardware resources (typically a number of whole
    computational nodes) and either executes a script or provides the user with an interactive terminal
    session on one of the allocated nodes. Allocations are often configured in a way that provides
    exclusive access to the hardware resources, in which case no other user will be able to use the
    allocated resources (nodes) until the allocation finishes.
    \item Once the executed script finishes (or the wall-time duration is reached), the allocation ends, and
    its hardware resources are released, so that they can be used by another allocation.
\end{enumerate}

Although it might not be obvious from the above description, this process presents perhaps the
largest obstacle for ergonomic execution of task graphs.


Due to the limits imposed by it, it might be necessary to partition task graphs into multiple
subgraphs in order to be fit them within an allocation, which can be challenging. Furthermore, it
can result in non-optimal usage of hardware resources, because tasks from (sub)graphs submitted in
separate allocations will only be load balanced within their own allocation, and not across
different allocations.

The primary issue of this mechanism is that it strictly ties together two separate aspects --
\emph{what} does the user want to compute (the script that will be computed in the
allocation) and \emph{where} should the computation take place (specific hardware
resources and computational nodes). As was described in the process above, both of these things
have to be specified together in an allocation.


This approach works well when the granularity of the computation very closely matches the
granularity of the requested hardware resources, and when the computation can make use of all the
available hardware resources for the whole duration of the allocation. For example, distributed
applications implemented using \gls{mpi} typically expect that they will be
executed on a fixed amount of nodes, and that they will (ideally) use all their resources for the
whole computation. They can also run for a potentially long time, e.g.\ hours, days or even more.
Due to these properties, these applications fit the allocation model quite well.

They tend to have fairly strict limits on the number of allocations that users can submit and the
number of nodes that they can have reserved for their allocations at any given time. Because of
these limits, allocations tend to be quite coarse-grained. They typically ask for a whole node at
minimum, and usually run at least for minutes, but more typically hours or even days.

Since users have to create allocations to compute anything on the cluster, and thus they cannot
simply execute their task graphs directly, a question naturally arises -- how to map tasks (or task
graphs) to allocations in a way that will efficiently utilize \gls{hpc}
resources? Several ways of performing this task-to-allocation mapping are described below, however
all of them come with significant disadvantages.

\Autoref{ch:hyperqueue} will discuss the interaction of task graphs and allocation managers in
more detail, and describe a design for overcoming the mentioned challenges.

%TODO: https://merlin.readthedocs.io/en/latest/tutorial/1_introduction/?h=many#how-can-merlin-run-so-many-simulations

\subsubsection*{Execute the whole task graph in a single allocation}
The simplest situation is when a task graph can be executed with a single allocation. If it does
not have a large number of tasks, or if it can be executed relatively quickly, users can create an
allocation that will compute the whole task graph. This approach is quite simple for the user,
since they just execute the task graph using a task runtime in the same way as they would on a
cluster without an allocation manager, or on a personal computer. The only difference is that they
have to define and submit an allocation that will bootstrap the computation.

However, since allocations are bound both by node count and a time limit, this approach is only
usable for rather small task graphs. Indeed, if the computation is short, it might not even make
sense to use an \gls{hpc} cluster to compute it. A more realistic scenario is
that even if an individual task graph can be executed quickly, users might want to execute many
such task graphs (for example to execute many experiments with different parametrizations). This
situation can be seen as a special case of a large task graph that consists of many disjoint
components (smaller task subgraphs). In this case, it will typically not be possible to execute all
such task graphs inside a single allocation.

\subsubsection*{Execute each task as an individual allocation}
From a certain point of view, \gls{hpc} allocation managers can also be viewed as
task runtimes that operate on a very coarse level -- their tasks being allocations that potentially
span hundreds of nodes, run for days or even longer and consist of many different program
executions. Ideally, there would be no difference between an allocation manager and a task runtime,
and users would just be able to construct an arbitrarily granular task graph and execute it
directly on an \gls{hpc} cluster in a straightforward way.

% https://rse.princeton.edu/2020/01/monitoring-slurm-efficiency-with-reportseff/
While this approach can certainly look tempting, in practice it is not usually feasible to use the
currently popular allocation managers (\gls{pbs} and Slurm) in this way, because
they operate on a level that is far too coarse for complex task graphs. While they do support
expressing dependencies between individual allocations in a crude way, they tend to have large
overhead per each allocation\todo[inline]{cite}, which can be several orders of magnitude
larger than for typical task runtimes (e.g.\ seconds vs milliseconds). Furthermore, they seldom
allow the user to create more than a few hundreds of allocations at the same time, both to provide
fairness and also because they simply cannot scale to such a number of allocations.

It should be noted that even though there is definitely room for improving the performance of
\gls{hpc} allocation managers, some of their complexity and performance
limitations are inherent. They have to provide accurate accounting, handle robust and secure
cleanup of hardware resource sprovided to allocations, manage user and process isolation on the
computational nodes, ensure user fairness and many other things. Many of these responsibilities are
out of scope for task runtimes, which enables them to achieve higher performance.

Another problem is node granularity. For ``small'' tasks that only use e.g.\ a few cores, users
would like to schedule and execute multiple tasks on a single node at the same time, to leverage
the available hardware resources efficiently. While allocation managers are able to create
allocations that require only a fraction of a node, this functionality is not always available.
Either for security reasons, because tasks from multiple users can then run on the same node at the
same time, which reduces user isolation, or for performance reasons, because the overhead of
scheduling a large number of allocations (in theory many allocations per each node) can become
unmanageable for the allocation manager~\cite{it4i_node_scheduling_policy}. When the manager is configured
in a way that a single allocation has to span at least a (complete) single node, it can lead to
wasted resources if a single task is unable to leverage the whole computational node.

Another reason why users might not want to use the allocation manager directly as a task runtime is
that it is useful to debug and prototype task graphs in a small-scale scenario (e.g.\ locally, on a
personal computer), before executing it on a large-scale \gls{hpc} system.
However, it can be quite challenging for users to deploy systems like \gls{pbs}
or Slurm locally. Therefore, they would need to use a different task runtime locally than on the
target \gls{hpc} platform, which woult not be very practical.

The mentioned issues stem from a dichotomy between the coarse-grained focused allocation manager
and more fine-grained focused task runtimes, which can create a barrier for
\gls{hpc} users. Users that want to execute a task graph on an
\gls{hpc} system thus usually use a separate task runtime (e.g.\
Dask~\cite{dask}) rather than using the allocation manager directly.

\subsubsection*{Partition the task graph into a smaller number of allocations}
The third approach that users may take is to partition their task graphs into smaller subgraphs,
and then submit each subgraph as a single allocation, where the subgraph will be computed by an
independent instance of a task runtime.

This is sort of an ultimate approach that the users will probably sooner or later converge to, once
their task graph becomes sufficiently complex and large, and they will have to somehow reconcile
the coarse-grained nature of allocations with the fine-grained nature of tasks and overcome the
overhead of creating too many allocations.

This process is not straightforward, especially if users have to perform the partitioning manually.
Graph partitioning itself is a notoriously difficult problem that is
NP-hard~\cite{graph_partitioning}\todo[inline]{Ada: Is this OK?}, and it is thus difficult to decide
beforehand how exactly should the task graph be split into allocations. Furthermore, if the
partitioning of tasks into allocations is performed statically, before the computation begins, then
it might lead to suboptimal hardware utilization, as it will not be possible to load-balance tasks
across different allocations, even if multiple allocations do run concurrently.

In addition to partitioning the task graph, this approach typically requires further implementation
efforts that are outside the boundaries of the task-based programming model. As an example, the
intermediate outputs of computed tasks of a partitioned subgraph might have to be persisted (to a
storage system) before the corresponding allocation ends, and the results from multiple allocations
then have to be merged together. In order to support recomputation of failed tasks, and submission
of new tasks while a task graph is already executing, it should also be possible to periodically
submit new allocations that could dynamically provide needed computational resources, until the
whole task graph is computed. This reduces the ergonomics of using task graphs, because it
basically requires users to reimplement parts of the task runtime behavior on top of the allocation
manager, in order to overcome its limitations.

\vspace{5mm}
The gap between allocation managers and task runtimes creates a disconnect for users attempting to
scale their task graph computation. Executing a task graph on a personal computer tends to be quite
simple. After that, moving to an \gls{hpc} cluster and executing the entire task
graph inside a single allocation is also quite straightforward. But once the task graph has to be
partitioned into multiple allocations, the simple abstraction of implicitly parallel task graphs
that can be executed with a single command quickly falls apart, as the user has to perform a lot of
additional work to make this scenario execute efficiently.

Ideally, users would not have to think about the allocation manager at all; they should be able to
construct a task graph and execute it directly on an \gls{hpc} cluster in a
straightforward way, by letting some tool perform the partitioning and load balancing across
allocations automatically for them. This could be achieved either by adding support for executing
fine-grained task graphs to allocation managers or by adding support for communicating with
allocation managers to task runtimes, to enable transparent execution of task graphs on
\gls{hpc} systems. This functionality is provided by several
\emph{meta-schedulers}, which will be mentioned in the next chapter.

\section{Cluster heterogeneity}
Even though task graphs are designed to be portable and ideally should not depend on any specific
execution environment, for certain types of tasks, we need to be able to describe at least some
generic environment constraints. For example, when a task executes a program that leverages the
CUDA programming framework~\cite{cuda}, which is designed to be executed on an
NVIDIA graphics accelerator, it has to be executed on a node that has such a
\gls{gpu} available, otherwise it will not work properly.

It should thus be possible for a task to define \emph{resource requirements}, which specify
resources that have to be provided by an environment that will execute such task. For example, a
requirement could be the number of cores (some tasks can use only a single core, some can be
multithreaded), the amount of available main memory, a minimum duration required to execute the
task or (either optional or required) presence of an accelerator like a \gls{gpu}
or an \gls{fpga}. In order to remain portable and independent of a specific
execution environment, these requirements should be abstract and describe general, rather than
specific, types of resources.

The challenge related to resource requirements of \gls{hpc} tasks specifically is
the diverse kinds of hardware present in modern \gls{hpc} clusters, which have
started to become increasingly heterogeneous in recent years. This trend can be clearly seen in the
TOP500 list of the most powerful supercomputers~\cite{top500analysis}. Individual cluster
nodes contain varying amounts and types of cores and sockets, main memory,
\gls{numa} nodes or accelerators like \glspl{gpu} or
\glspl{fpga}. Since \gls{hpc} software is often designed to
leverage these modern \gls{hpc} hardware features, this complexity is also
propagated to tasks and their resource requirements.

Tasks might require a combination of several requirements, for example two
\glspl{gpu}, sixteen cores and 32 GiB of main memory. They can also be designed in
a way that allows them to leverage an open-ended range of resources, e.g.\ a task might require at
least four cores, but if more are available, it could use as many as possible. And some tasks might
even support several variants of requirements, for example a task might either use four cores and a
single \gls{gpu} (if there is one available), or it could use more cores (and no \gls{gpu})
to offset the absence of an accelerator.

A resource requirement that is fairly specific to \gls{hpc} systems is the usage
of multiple nodes per single task. This requirement is necessary for programs that are designed to
be executed in a distributed fashion, such as programs using \gls{mpi}, which are
quite common in \gls{hpc}. The use-case of tasks using multiple nodes is
discussed in more detail later in this chapter.

Existing task runtimes usually support some notion of resource requirements, although they can be
somewhat limited. Some runtimes support only a fixed set of known resources (typically the number
of cores and the amount of memory), which is not enough to describe complex resources of
heterogeneous clusters. Other runtimes, such as \dask{} or
\snakemake{}, support arbitrary resources, although they do not allow expressing
more complex patterns, such as the mentioned case where a task could define multiple resource
requirement configurations. Supporting multiple nodes for a single task is not supported by most
runtimes at all, because their programming model often assumes that a each task executes on a
single node only.

To support the mentioned scenarios, task runtimes should ideally allow users to specify arbitrarily
fine-grained and abstract resource requirements for each task, with support for multiple
requirement variants per task and multi-node tasks. They should also take these requirements into
account when scheduling, both to make sure that they are upheld, but also to use this provided
information to utilize the available hardware effectively.

\section{Data transfers}
After a task is computed, it can produce various outputs, such as standard error or output streams,
files created on a filesystem or data objects that are then passed as inputs to dependent tasks.
There are many approaches to storing and transferring these outputs.

Some task frameworks store task outputs on the filesystem, since it is relatively simple to
implement and it provides support for basic data resiliency out-of-the-box. A lot of existing
software (that might be executed by a task) also makes liberal use of the filesystem, which can
make it challenging to avoid filesystem access altogether. However, \gls{hpc}
nodes often do not contain any local disks, but instead use shared filesystems that are accessed
over a network. While this can be seen as an advantage, since with a shared filesystem it is much
easier to share task outputs amongst different workers, it can also be a severe bottleneck. Shared
networked filesystems can suffer from high latency and accessing them can consume precious network
bandwidth that is also used e.g.\ for managing computation (sending commands to workers) or for
direct worker-to-worker data exchange. Furthermore, data produced in \gls{hpc}
computations can be quite large, and thus storing it to a disk can be a bottleneck even without
considering networked filesystems.

These bottlenecks can be alleviated by transferring task outputs directly between workers over the
network (preferably without accessing the filesystem in the fast path), which is implemented e.g.\
by \dask{}. Some runtimes (for example HyperLoom~\cite{hyperloom})
also leverage \gls{ram} disks, which provide support for tasks that need to
interact with a filesystem, while avoiding the performance bottlenecks associated with disk
accesses. It is also possible to use \gls{hpc} specific technologies, such as
\gls{mpi} or InfiniBand, to improve data transfer performance by fully exploiting
the incredibly fast interconnects available in \gls{hpc} clusters.

Data outputs produced by tasks tend to be considered immutable in existing task runtimes, since a
single output can be used as an input to multiple tasks, and these might be executed on completely
different computational nodes, so it would be challenging to synchronize concurrent mutations of
these objects. A problem that can arise with this approach is that if the data outputs are large,
but the computation of tasks that work with the data is short, then the serialization overhead (or
even memory copy overhead, if the dependent task is executed on the same node) can dominate the
execution time. Such use-cases can be solved with stateful data management, for example in the form
of \emph{actors}, which can be seen as stateful tasks that operate on a single copy
of some large piece of data, without the need to transfer or copy it.

\section{Fault tolerance}
Fault tolerance is relevant in all distributed computing environments, but
\gls{hpc} systems have specific requirements in this regard. As was already
mentioned, computational resources on \gls{hpc} clusters are provided through
allocation managers. Computing nodes allocated by these managers are provided only for a limited
duration, which means that for long-running computations, some nodes will disconnect and new nodes
might appear dynamically during the lifecycle of the executed workflow. Furthermore, since the
allocations go through a queue, it can take some time before new computational resources are
available, therefore the computation can remain in a paused state, where no tasks are being
executed, for potentially long periods of time.

It is important for task runtimes to be prepared for these situations; they should handle node
disconnections gracefully, even if a task was being executed on a node that is disconnected, and
they should be able to restart previously interrupted tasks on newly arrived workers. In general,
in \gls{hpc} scenarios, worker instability and frequent disconnects should be
considered the normal mode of operation, rather than just a rare edge case.

\section{Multi-node tasks}
Many existing \gls{hpc} applications are designed to be executed on multiple
(potentially hundreds or even thousands) nodes in parallel, using e.g.\ \gls{mpi}
libraries or other communication frameworks. Multi-node execution could be seen as a special
resource requirement, which states that a task should be executed on multiple workers at once.
Support for multi-node tasks is challenging, because it affects many design areas of a task
runtime:
\begin{description}
    \item[Scheduling] When a task requires multiple nodes for execution and not enough nodes are available at a given
    moment, the scheduler has to decide on a strategy that will allow the multi-node task to execute.
    If it would be constantly trying to backfill available workers with single-node tasks, the
    multi-node tasks could be starved.

    The scheduler might thus have to resort to keeping some nodes idle for a while to enable the
    multi-node task to start as soon as possible. Another approach could be to interrupt the currently
    executing tasks and checkpoint their state to make space for a multi-node task, and then resume
    their execution once the multi-node task finishes.

    In a way, this decision-making already has to be performed on the level of individual cores even
    for single-node tasks, but adding multiple nodes per task makes the problem much more difficult.
    \item[Data transfers] It is relatively straightforward to express data transfers between single-node tasks in a task
    graph, where a task produces a set of complete data objects, which can then be used as inputs for
    dependent tasks. With multi-node tasks, the data distribution patterns become more complex, because
    when a task that is executed on multiple nodes produces a data object, the object itself might be
    distributed across multiple workers, which makes it more challenging to use it in follow-up tasks.
    The task graph semantics might have to be extended with expressing various data distribution
    strategies, for example a reduction of data objects from multiple nodes to a single node, to
    support this use-case.

    When several multi-node tasks depend on one another, the task runtime should be able to exchange
    data between them in an efficient manner. This might require some cooperation with the used
    communication framework (e.g.\ \gls{mpi}) to avoid needless repeated serialization
    and deserialization of data between nodes.
    \item[Fault tolerance] When a node executing a single-node task crashes or disconnects from the runtime, its task can be
    rescheduled to a different worker. In the case of multi-node tasks, failure handling is generally
    more complex. For example, when a task is executing on four nodes and one of them fails, the
    runtime has to make sure that the other nodes will be notified of this situation, so that they can
    react accordingly (either by finishing the task with a smaller number of nodes or by failing
    immediately).
\end{description}

Most existing task runtimes only consider single-node tasks and do not provide built-in support for
multi-node tasks, which forces users to use various workarounds, for example by emulating a
multi-node task with a single-node task that uses multi-node resources that are not managed by the
task runtime itself. To enable common \gls{hpc} use-cases, task runtimes should
be able to provide first-class support for multi-node tasks and allow them to be combined with
single-node tasks in a seamless manner. Advanced multi-node task support could be provided e.g.\ by
offering some kind of built-in integration with \gls{mpi} or similar common
\gls{hpc} technologies.

\section{Scalability}
The massive scale of \gls{hpc} hardware (node count, core count, network
interconnect bandwidth) opens up opportunities for executing large-scale task graphs, but that in
turn presents unique challenges for task runtimes. Below you can find several examples of
bottlenecks that might not matter in a small computational scale, but that can become problematic
in the context of \gls{hpc}-scale task graphs.

\begin{description}
    \item[Task graph materialization] Large computations might require building massive task graphs that contain millions of tasks. The
    task graphs are typically defined and built outside of computational nodes, e.g.\ on login nodes of
    computing clusters or on client devices (e.g.\ laptops), whose performance can be limited. It can
    take a lot of time to build, serialize and transfer such graphs over the network to the task
    runtime that runs on powerful computational nodes. This can create a bottleneck even before any
    task is executed. This has been identified as an issue in some existing task
    runtimes~\cite{dask-client-perf}.

    In such case, it can be beneficial to provide an \gls{api} for defining task
    graphs in a symbolic way by representing a potentially large group of similar tasks with a
    compressed representation to reduce the amount of consumed memory. For example, if we want to
    express a thousand tasks that all share the same configuration, and differ e.g.\ only in an input
    file that they work with, we could represent this a group of tasks with a size thousand, rather
    than storing a thousand of individual task instance in memory. This could be seen as an example of
    the classical Flyweight design pattern~\cite{gof}.

    Such symbolic graphs could then be sent to the runtime in a compressed form and re-materialized
    only at the last possible moment. In an extreme form, the runtime could operate on such graphs in a
    fully symbolic way, without ever materializing them.
    \item[Communication overhead] Scaling the number of tasks and workers will necessarily put a lot of pressure on the communication
    network, both in terms of bandwidth (sending large task outputs between nodes) and latency (sending
    small management messages between the scheduler and the workers). Using \gls{hpc}
    technologies, such as \gls{mpi} or a lower-level interface like
    \gls{rdma}, could provide a non-trivial performance boost in this regard. Some
    existing runtimes, such as \dask{}, can make use of such
    technologies~\cite{dask-ucx}.

    As we have demonstrated in~\cite{pspin, spin2}, in-network computing can be also used to
    optimize various networking applications by offloading some computations to an accelerated
    \gls{nic}. This approach could also be leveraged in task runtimes, for example to
    reduce the latency of management messages between the scheduler and workers or to increase the
    bandwidth of large data exchanges amongst workers, by moving these operations directly onto the
    network card. This line of research is not pursued in this thesis, although it could serve as an
    interesting idea to be explored.
    \item[Scheduling] Task scheduling is one of the most important responsibilities of a task runtime, and with large
    task graphs, it can become a serious performance bottleneck. Existing task runtime, such as
    \dask{}, can have problems with keeping up with the scale of
    \gls{hpc} task graphs. The scheduling performance of various task scheduling
    algorithms will be examined in detail in \Autoref{ch:estee}. \autoref{ch:rsds}
    then will describe the performance overhead of the \dask{} runtime and its
    scheduler.
    \item[Runtime overhead] A typical architecture of a task runtime consists of a centralized component that handles task
    assignments, and a set of connected workers. As we have shown in~\cite{rsds}, task
    runtimes with this architecture should be mindful of their task execution and communication
    overhead. For example, even with an overhead of just $1ms$ per task, executing
    a task graph with a million tasks would result in total accumulated overhead of more than fifteen
    minutes! Our results indicate that increasing the performance of the central scheduling and
    management component of a task runtime can have a large positive effect on the overall time it
    takes to execute the whole task graph. This is explored in detail in \Autoref{ch:rsds}.

    However, the performance of the central server cannot be increased endlessly, and from some point,
    a centralized architecture will become a bottleneck. Even if the workers exchange data outputs
    directly between themselves, the central component might become overloaded simply by coordinating
    and scheduling the workers using management messages.

    In that case, a decentralized architecture could be leveraged to avoid the reliance on a central
    component. Such a decentralized architecture can be found e.g.\ in Ray~\cite{ray}.
    However, to realize the gains of a decentralized architecture, task submission itself has to be
    decentralized in some way, which might not be a natural fit for common task graph workflows. If all
    tasks are generated from a single component, the bottleneck will most likely remain even in an
    otherwise fully decentralized system.
\end{description}

\section{Iterative computation}
There are various \gls{hpc} use-cases that are inherently iterative, which means
that they perform some computation repeatedly, until some condition (which is often determined
dynamically) is satisfied. For example, the training of a machine learning model is typically
performed in iterations (called epochs) that continue executing while the prediction error of the
model keeps decreasing. Another example could be a molecular dynamics simulation that is repeated
until a desired accuracy (or other property) has been reached.

One approach to model such iterative computations would be to run the whole iterative process inide
a single task. While this is simple to implement, it might not be very practical, since such
iterative processes can take a long time to execute, and performing them in a single task would
mean that we could not leverage desirable properties offered by the task graph abstraction, for
example fault tolerance. Since the computation \emph{within} a single task is
typically opaque to the task runtime, when the task would fail, it would need then need to be
restarted from scratch.

A better approach might be to model each iteration as a separate task. In such case, the task
runtime might be able to restart the computation from the last iteration, if a failure occurs.
However, this approach can be problematic if the number of iterations is not known in advance,
since some task runtimes expect that the structure of the task graph will be immutable once the
graph has been submitted for execution.

To support iterative computation, task runtimes should allow their users to stop the execution of a
task graph (or its subgraph) once a specific condition is met, and also to add new tasks to the
task graph in a dynamic fashion, if it is discovered during its execution that more iterations are
needed. \dask{} and \ray{} are examples of task runtimes
that allow adding tasks to an existing task graph dynamically.

\section{Deployment}
Even though it might sound trivial at first, an important aspect that affects the ergonomics of
executing task graphs is the ease of deploying the task runtime on an \gls{hpc}
cluster. Supercomputing clusters are notable for providing only a severely locked-down environment
for their users, which does not grant elevated privileges and requires users to either compile and
assemble the runtime dependencies of their tools from scratch or choose from a limited set of
precompiled dependencies available on the cluster, which can have incompatible versions or a
non-optimal configuration.

Deploying any kind of software (including task runtimes) that has non-trivial runtime dependencies,
or non-trivial installation steps, if it is not available in a pre-compiled form for the target
cluster, can serve as a barrier for using it on an \gls{hpc} cluster. Many
existing task runtimes are not trivial to deploy. For example, several task runtimes, such as \dask{},
\snakemake{} or \pycompss{}, are implemented in Python, a language
that typically needs an interpreter to be executed, and also a set of packages in specific
versions, that need to be available on the filesystem of the target cluster.

Installing the correct version of a Python interpreter and Python package dependencies can already
be slightly challenging in some cases, but there can also be other issues, caused by the
dependencies of the task runtime conflicting with the dependencies of the software executed by the
task graph itself. For example, since \dask{} is a Python package, it depends
on a certain version of a Python interpreter and a certain set of Python packages. It supports
tasks that can directly execute Python code in the process of a \dask{} worker.
However, if the executed code requires a different version of a Python interpreter or Python
packages than \dask{} itself, this can lead to runtime errors or subtle
behavior changes\footnote{It is not straightforward to use multiple versions of the same Python package dependency in a
single Python virtual environment. Conflicting versions thus simply override each other.}.

Since it is probable that users of \gls{hpc} clusters will already have enough
issues with installing the software executed by their workflows, the task runtime itself should
ideally provide frictionless deployment, and it should not interact with external dependencies
required by the executed tasks.

\section*{Summary}
Even though more \gls{hpc} peculiarities could always be found, it is
already clear from all the mentioned challenges that \gls{hpc} use-cases that
leverage task graphs can be very complex. While existing task runtimes are able to deal with some of
the mentioned challenges with varying degrees of succes, a unified, dedicated and
\gls{hpc}-tailored approach for executing task graphs could provide added value to workflow authors, both in terms of
ergonomics and performance. A design for such an approach will be explored in
\Autoref{ch:hyperqueue}, which describes the \hyperqueue{} task runtime.
