In order to design approaches for seamless execution of task graphs on supercomputers, it is first
necessary to examine the current limitations and issues faced by users that want to use task graphs
in an \gls{hpc} setting. This chapter describes several challenges in this area, which
belong to the two broad categories that form the main focus of this thesis, namely efficiency (how
to achieve high hardware utilization and make task graph execution scalable) and ergonomics (how to
make it easy for users to define and execute task graphs). Most of these challenges stem from the
sheer scale of \gls{hpc} workloads and clusters and from the complexity and
idiosyncrasies of supercomputers, which have historically been designed for different kinds of
programming models.

Execution ergonomics (in the context of task-based programming models) is a broad area that
encompasses several aspects, such as providing an easy way to define the structure of the task
graph, allowing its execution in an effortless manner (both on a local computer and a distributed
cluster), handling fault tolerance without the user's intervention, allowing users to express
complex resource requirements and many others. In particular, we will focus on identifying
ergonomic challenges that form a barrier for executing task graphs on \gls{hpc}
clusters and that limit efficient usage of heterogeneous hardware resources.

We will also discuss how are existing task runtimes able to deal with these challenges and what
approaches do they use. This is important, because in order to achieve a seamless task graph
execution experience on supercomputers, one should use a task runtime that takes these
\gls{hpc} peculiarities in mind; using off-the-shelf tools that were not designed for
\gls{hpc} use-cases might be challenging.

There is a large body of tools designed for executing batch-oriented task graphs on diverse
computing platforms, ranging from consumer-grade laptops, through cloud deployments, to distributed
and \gls{hpc} clusters. They are known under various terms, such as task executors,
job managers, distributed job schedulers, dataflow engines or orchestrators. We will use the term
\emph{task runtime} for all such task execution tools in this thesis, as has already been
discussed in the previous chapters. These runtimes are quite popular, and they are being used for
computing all kinds of scientific workflows on \gls{hpc}
clusters~\cite{hpc_tasks, hpc_tasks_2, hpc_tasks_3, pegasus}.

Examples of such task runtimes include e.g.\ \dask~\cite{dask},
\parsl~\cite{parsl}, \ray~\cite{ray},
\pycompss~\cite{pycompss}, \textsc{HyperLoom}~\cite{hyperloom},
\gnuparallel~\cite{parallel}, \snakemake~\cite{snakemake},
\merlin~\cite{merlin},
\autosubmit~\cite{autosubmit} or \fireworks~\cite{fireworks}. Each task runtime
defines its own instance of a task-based programming model, and has a different set of trade-offs
in areas such as performance and scalability, fault tolerance, ease-of-use, ease-of-deployment and
others.

Various representatives of task runtimes will be described in this chapter in the context of the
individual task graph execution challenges. \Autoref{ch:rsds} will then provide a detailed
analysis of \dask{}, a state-of-the-art task runtime, and~\Autoref{hq:related-work}
will compare several existing task runtimes with \hyperqueue{}, an
\gls{hpc}-optimized task runtime that will be the main topic
of~\Autoref{ch:hyperqueue}.

Below you can find a list of the most important identified challenges; they will be described in
detail in the rest of this chapter.

\begin{description}[wide=0pt, itemsep=1pt]
	\item[Allocation manager] Any computation that is to be executed on an \gls{hpc} cluster shared by many users
		typically has to go through a queue managed by an \emph{allocation manager}, which provides access to
		the cluster through transient hardware allocations. This complicates the execution of task graphs
		due to the need to assign tasks to individual allocations in order to maximize hardware utilization
		and overcome various limits. Ideally, task runtimes should offer a way to automate this process.
	\item[Cluster heterogeneity] Modern \gls{hpc} clusters are very heterogeneous and might contain various kinds of
		accelerator devices. It is important for a task runtime to offer complex support for arbitrary
		resource requirements in order to take advantage of complex \gls{hpc} hardware and
		achieve high utilization of hardware.
	\item[Performance and scalability] The large scale of \gls{hpc} computations and the vast amount of available hardware
		resources introduce unique performance challenges. Task runtimes should have minimal overhead in
		order to efficiently execute even very large task graphs.
	\item[Fault tolerance] The scale of \gls{hpc} task graphs and clusters and the transient nature of allocations
		makes failures an ordinary occurrence, rather than a rare edge case. This necessitates special
		considerations for the design of task runtimes.
	\item[Multi-node tasks] Many \gls{hpc} applications are designed to run on multiple nodes in parallel. This is
		an uncommon requirement for most task-based programming models; it requires special support to make
		this use-case a first-class concept.
	\item[Deployment] Supercomputers typically provide a severely locked-down user environment, where it can be quite
		challenging to build and deploy software that requires non-trivial build or runtime dependencies.
		Task runtimes should thus be trivial to deploy in order to facilitate their usage on
		\gls{hpc} clusters.
	\item[Programming model] \gls{hpc} task graphs can be very diverse, and
		range from structurally simple task graphs that contain a large number of tasks to very
		heterogeneous workflows with complex dependencies. Task runtimes should ideally provide a
		programming model that is able to support diverse use-cases in an ergonomic way.
\end{description}

\section{Allocation manager}
\label{challenge:allocation-manager}
Users of \gls{hpc} clusters are not typically allowed to directly perform arbitrary
computations on the computational nodes (machines designed to perform expensive computations) of
the cluster. Instead, they connect to machines that are usually called \emph{login nodes},
from which they have to enqueue their desired computation into a queue handled by a submission
system that manages the hardware resources of the cluster. We will use the term
\emph{allocation manager} for these submission systems and the term
\emph{allocation}\footnote{The term \emph{job} is also commonly used for the concept of \gls{hpc}
computational requests.} for a computational request submitted by a user into
these managers.

Allocation managers are required for providing fair access to the resources of the cluster, because
\gls{hpc} clusters are typically used by many people at the same time. Without a
centralized management, hardware resources could be inadvertently shared among multiple users at
once, which could have undesirable performance and security implications, and could lead to
oversubscription. Furthermore, usage of these clusters is usually metered. Users can typically only
use a certain amount of resources assigned to their \emph{computational project}, and when their
resources run out, they have to ask (or pay) for more resources. Allocation managers thus also
implement user and project accounting, so that there is a clear historical record of how many
resources were consumed by individual users of the cluster.

The majority of \gls{hpc} clusters~\cite{slurm-schedmd} use one of the two most
popular allocation manager implementations, either Slurm~\cite{slurm} or
\gls{pbs}~\cite{pbs}\footnote{Or some of its many derivatives,
such as TORQUE~\cite{torque} or OpenPBS~\cite{openpbs}}. For simplicity, Slurm will
be used as a default representative of allocation managers in the rest of this thesis (unless
otherwise noted), since it shares essentially all the important described constraints with
\gls{pbs}.

The following process describes how computations are typically executed on \gls{hpc}
clusters that use an allocation manager:

\begin{enumerate}
	\item The user enqueues a computational request (allocation) into the manager from a login node. The
	      request typically has to specify at least how many nodes should be allocated and the maximum
	      duration of the computation (usually labeled as \emph{wall-time}), after which the
	      computation will be forcibly stopped by the manager. It can also contain additional configuration,
	      such as what kinds of nodes should be allocated or what is the priority of the request.
	\item The allocation manager puts the request into a queue and schedules it to be executed at some time
	      in the future. Since users submit their allocations into the manager continuously, each allocation
	      has different properties and priorities, and it is not possible to precisely predict for how long
	      an allocation will run, the schedule can be very dynamic and unpredictable, so users might have to
	      wait seconds, minutes, hours or even days before their allocation starts to execute.
	\item Once the allocation gets to the front of the queue and there are enough resources available, the
	      manager provisions the requested amount of hardware resources (typically a number of whole
	      computational nodes) and either executes a script (a \emph{batch} allocation) or provides
	      the user with an interactive terminal session on one of the allocated nodes (an
	      \emph{interactive} allocation). Allocations are often configured in a way that provides their
	      authors with exclusive access to the allocated hardware resources, in which case no other user will
	      be able to use these resources (nodes) until the allocation finishes.
	\item Once the executed script finishes (or the wall-time duration is reached), the allocation ends, and
	      its hardware resources are released so that they can be used by another allocation.
\end{enumerate}

Although it might not be obvious from the above description at first, the presence of an allocation
manager presents perhaps the largest obstacle for ergonomic execution of task graphs on
\gls{hpc} clusters. Instead of executing their task graphs directly on the cluster,
users first have to think about how to split their task graph into separate allocations and manage
their submission and execution. While this is true of any computation executed on
\gls{hpc} clusters in general, in the case of task graphs it is especially difficult
due to their complex structure, and also because the concept of allocations was historically
created with different programming models in mind. Mapping task workflows to
\gls{hpc} allocations can thus be non-trivial~\cite{glume,slurm-workflow}.

To execute a task graph using an allocation manager, it is desirable to find a way to map tasks to
allocations in a way that efficiently utilizes \gls{hpc} resources and is as
transparent as possible for the users. There are several approaches that can be used for performing
this task-to-allocation assignment. In order to understand the various trade-offs and constraints
being involved (which are mostly caused by various performance bottlenecks and limitations of
existing allocation managers), we will first describe three straightforward methods for mapping
tasks to allocations, which can be used even without any special support from a task runtime. After
that, we will examine more automated approaches taken by existing task runtimes and other research
works.

\subsection*{Single allocation per task graph}
At first, it might seem that executing a task graph within a single allocation is a simple
solution, since all the user has to do is submit an allocation that will eventually execute the
complete task graph (using some task runtime), which is a similar approach that would be used for
executing the task graph e.g.\ on a personal computer.

And it is indeed simple -- when it is possible at all. The problem is that allocation managers tend
to place fairly strict limits on the maximum possible execution duration of an allocation (the
wall-time) and also on the number of resources (nodes) that can be requested at once, to ensure a
fairer assignment of resources to users of a cluster. Therefore, when a task graph has too many
tasks, takes too long to execute, or the user wants to leverage more computational resources than
can fit in a single allocation, this approach will not work.

In fact, if the task graph computation is short enough that it can fit within a single allocation,
it might not always even make sense to use an \gls{hpc} cluster to compute it. A more
realistic scenario is that even if an individual task graph can be executed quickly, users might
want to execute many such task graphs (for example to execute many experiments with different
parametrizations). This situation can be seen as a special case of a large task graph that consists
of many disjoint components (smaller task subgraphs). In this case, it will again typically not be
possible to execute all such task graphs inside a single allocation.

Even when a task graph can be reasonably executed within a single allocation, this approach might
lead to hardware underutilization and resource waste~\cite{ehpc}. Consider a typical
data analysis or a machine learning workflow that works in several stages. First, it loads a
dataset from disk and preprocesses it, then it trains a machine-learning model, and finally it
performs some postprocessing step, which e.g.\ analyzes the resulting model. The preprocessing and
postprocessing steps are usually not very resource intensive and run on \glspl{cpu}
only, while the training step typically consumes a lot of resources and is usually executed on a
\gls{gpu} accelerator. To execute such a workflow in a single allocation, we will
need to ask for a set of resources that contain such an accelerator. The issue is that all
resources of an allocation are reserved for its whole duration; therefore, we will have to pay the
price for the expensive accelerated nodes (and prevent others from using them) even during the
execution of workflow steps in which these nodes will be underutilized or completely idle. This is
caused by the fact that all allocated resources are tied to the lifetime of the whole allocation,
and using this approach we have to ask for a set of resources that form a union of the resource
requirements of all tasks of the workflow.

One additional disadvantage of this approach is queuing latency. Large allocations that require a
long wall-time duration or request many computational nodes typically spend a much longer time in
the allocation manager queue, because it is more difficult for the manager to make sure that all
the requested resources are available at the same time. As a concrete example, ten allocations that
require a one hour wall-time can potentially be allocated much quicker than a single allocation
that requests a ten hour wall-time, because these allocations do not need to run at the same time,
and are thus easier for the manager to schedule. Similar behavior can be observed w.r.t.\ the
number of requested nodes.

This approach can be used with essentially all existing task runtimes, since it does not require
any special support from the runtime. The user simply submits a single allocation, and when it
starts, the whole task graph is computed with a task runtime all in one go. This method is thus
relatively simple for the user, although as has been mentioned, it might not be feasible for many
use-cases, and it can lead to resource waste and a long latency before receiving the first results
of finished tasks.

\subsection*{Separate allocation for each task}
The previous approach mapped all tasks to a single allocation. An opposite extreme would be to map
each task to a separate allocation. This approach might seem intuitive, because from a certain
point of view, \gls{hpc} allocation managers can also be viewed as task runtimes, if
we consider allocations to be tasks; therefore, it might be tempting to treat them as such. Both
\gls{pbs} and Slurm even support a crude notion of dependencies between allocations,
which could allow users to express computational graphs. Indeed, in an ideal world, there would be
no difference between an allocation manager and a task runtime, and users could simply construct an
arbitrarily granular task graph and execute it simply by submitting it directly into the allocation
manager.

However, in practice, this approach is mostly infeasible, at least with the currently popular
allocation managers (\gls{pbs} and Slurm), because they introduce a non-trivial
amount of overhead per each allocation~\cite{falkon}. In a best case scenario, Slurm is
able to launch a few hundred allocations per second~\cite{slurm-throughput}. However, more
realistically, users on a crowded cluster might experience at least a few hundred milliseconds
overhead per each allocation, if not more, which is order(s) of magnitude more than the overhead of
an efficient task runtime~\cite{rsds}.

Even though there is still probably room for reducing the overhead of contemporary
\gls{hpc} allocation managers, it is important to note that some of the performance
limitations are inherent. Allocation managers have to (among other things) provide accurate
accounting, handle robust and secure provisioning and cleanup of hardware resources provided to
allocations, manage user and process isolation on computational nodes and ensure user fairness.
These responsibilities are out of scope for most task runtimes, and thus it is not surprising that
they can usually achieve much higher performance. While in theory, it could be possible to design
an allocation manager that can also act as a (performant) task runtime at the same time (some
efforts have been made on this front e.g.\ by Flux~\cite{flux}, which will be described
later below), it is probably infeasible to modify the two most prominent allocation managers that
are used almost ubiquitously in the current \gls{hpc} ecosystem to support this
use-case.

Due to this overhead, allocation managers usually limit the number of allocations that can be
enqueued by a single user at any given time (e.g.\ to a few hundred or a thousand), to ensure that
the manager is not overloaded and that it can continue to serve requests from other users of the
cluster. Therefore, unless the task graph is relatively small, it will most likely be infeasible to
create a separate allocation for each task.

It should be noted that both Slurm and \gls{pbs} allow partially overcoming this
limitation through \emph{allocation arrays} (also labeled as job arrays). This concept allows users to submit a large amount of
computations with the same shape and properties in a single allocation. However, it still has
several disadvantages. One is a lack of flexibility; it does not easily allow submitting
heterogeneous tasks with different resource requirements, and it also has only a very crude support
for dependencies between the submitted tasks. Another disadvantage is fault tolerance; if some of
the tasks of the array fail, users have to manually identify and resubmit them in another
allocation, which is far from effortless. Furthermore, clusters impose limits for allocation arrays
as well, and thus even the array approach might not be enough to encompass all tasks of massive
task graphs.

Apart from the maximum allocation count, there are other limitations that can be imposed by
allocation managers. Some tasks of scientific workflows can be granular, and require only few
resources, e.g.\ a single \gls{cpu} core. To avoid wasting resources, an allocation
mapped to such a task should thus ask only for the exact set of resources needed by it. However,
allocation managers are sometimes configured in a way that only offers node-level granularity of
hardware resources, and thus does not allow users to request less than a whole computational
node~\cite{it4i_node_scheduling_policy}. In these cases, it would be wasteful if we had to request a whole
computational node e.g.\ for a simple task that runs for just a couple of seconds and requires only
a couple of cores.

There are two primary reasons why allocation managers limit the granularity of allocations. First,
reducing the granularity of allocations lessens the overall overhead. If users are able to ask
e.g.\ for individual cores of each node, the manager would have to manage many more allocations
than when users ask for complete nodes, which could quickly become unmanageable. The second reason
is security. If the manager provides the ability to reserve only a fraction of a node, allocations
from multiple users can execute on the same node at the same time. This reduces isolation between
users, which might potentially have security implications. It can also affect performance, since
concurrently running allocations might compete for some shared resource on the same node (e.g.\ the
operating system process scheduler). Since each node runs a single instance of an operating system,
it forms a natural isolation domain, and it is thus also frequently used as a unit of granularity
for allocations.

Another reason why using an allocation manager directly as a task runtime might be impractical is
that it makes debugging and prototyping task graphs more difficult. It is useful to have the
ability to examine the execution of a workflow locally, e.g.\ on the user's personal computer,
before running it on a cluster. However, if the task graph were implemented directly in an
allocation manager, users would have to deploy tools like \gls{pbs} or Slurm locally
in order to debug their workflows, which could be challenging.

To summarize, using an allocation manager directly as a task runtime is currently mostly
impractical, primarily because of the overhead associated with each allocation and the resulting
difference in task and allocation granularity, which can lead to resource waste. Therefore, users
who want to execute a task graph on an \gls{hpc} system usually use a separate task
runtime rather than defining task graphs using the allocation manager directly.

\subsection*{Task graph partitioning}
Both of the mentioned extreme approaches have severe disadvantages. A compromise between them is to
partition the task graph into smaller subgraphs and submit each subgraph as a separate allocation.
This approach allows mapping a large number of tasks into a smaller number of allocations, and thus
amortize the allocation overhead. Most users will probably sooner or later converge to a similar
approach, once their task graph becomes sufficiently large and complex, and they will have to
reconcile the coarse-grained nature of allocations with the fine-grained nature of tasks.

This process is far from straightforward if it is performed manually, i.e.\ if the user has to
manually split the task graph before submitting the allocations, and then start an independent
instance of a task runtime inside each allocation. Not just because an optimal graph partitioning
is itself a notoriously difficult NP-hard~\cite{graph_partitioning} problem in general, but also
because it requires implementation efforts outside the boundaries of the used task-based
programming model. In other words, users might need to reimplement some parts of a task runtime due
to the fact that a part of the task graph execution happens inside allocations and a part happens
outside allocations.

For example, the intermediate outputs of computed tasks of a partitioned subgraph might have to be
persisted to a filesystem before the corresponding allocation ends, and the results from multiple
allocations then have to be merged together to produce all the necessary data outputs of the
complete task graph. Another issue is that if some tasks fail and their corresponding allocation
ends, the task runtime does not get a chance to recompute them. Users will thus need to identify
such failed tasks and then selectively resubmit them into new allocations.

\subsection*{Meta-scheduling}
While the task graph partitioning approach is a general solution to the problem of running task
graphs using allocation managers, it is fairly cumbersome when it has to be performed by users
manually. There are various task runtimes that can help to (either partially or fully) automate
this process. We will use the term \emph{meta-schedulers} for these task runtimes in this thesis.
These tools typically operate on a level above the allocation manager, instead of only running
within the context of a single allocation. They are thus able to manage task graphs that span
multiple allocations.

Existing task runtimes use various approaches for providing some form of meta-scheduling. Tools
such as \gnuparallel{}~\cite{parallel} can automatically map each task to a
separate allocation. While this can be useful for running simple task graphs on supercomputers, it
does not help overcoming allocation manager submission limits.
\hypershell{}~\cite{hypershell} improves upon this by enabling users to specify a
\emph{task bundle size} parameter, which batches the configured number of tasks together in the
same allocation. While this method can help amortize the cost of allocations for large task graphs,
it is mostly only useful for task graphs without dependencies, for which it is possible to express
the partitioning using a single number. It can be used by \hypershell{} since it does not
support dependencies between tasks.

\snakemake{}~\cite{snakemake} allows its users to explicitly assign each task to
a \emph{group}. Tasks with the same group are then computed in the same allocation.
While this approach is more flexible than using a single batching parameter, it requires users to
perform the partitioning manually, which is quite challenging to do well in order to achieve good
node utilization. Without using groups, \snakemake{} executes each task in a separate
allocation, which suffers from the already mentioned disadvantages, and is even actively
discouraged on some \gls{hpc} clusters~\cite{nersc-snakemake}.

An improved partitioning process has been described in~\cite{zhang_batch_scheduling}. It partitions task
graphs based on their structure, by splitting them into levels that contain subgraphs of tasks that
are somehow related. A similar approach is used by task runtimes such as
\pegasus{}~\cite{pegasus} or \autosubmit~\cite{autosubmit}. For
example, a ``horizontal level'' partitioning groups tasks that do not have dependencies between
each other and that can run in parallel, while a ``vertical level'' partitioning groups tasks that
form a dependency chain. While this approach makes partitioning easier for users, they still need
to decide which partitioning mode should be used for specific parts of the graph. Furthermore, a
significant disadvantage is that the partitioning is performed statically, before the task graph is
executed. This can lead to suboptimal hardware utilization, because it is not possible to load
balance tasks across different allocations. It can also lead to resource waste, because allocations
created to execute a group of tasks will typically be configured so that they ask for a sum of all
the resource requirements of tasks within that group.

\textsc{E-HPC}~\cite{ehpc} employs a similar approach, where it splits
individual stages of workflows into separate allocations that contain the ideal amount of resources
for that stage. To tackle the issue of allocations that reserve too many hardware resources, it
allows these stages to dynamically modify their resource requirements through the use of
checkpointing. When a stage dynamically asks for a different set of resources, it is checkpointed,
moved to a different allocation and then restarted. While this approach does alleviate some
problems with resource waste, it is relatively inflexible; workflows have to be manually separated
into individual stages with coarse-grained resource requirements, and these stages are still
submitted as individual allocations rather than being load balanced across different allocations.

To achieve a better utilization of computational nodes, some task runtimes allow users to deploy a
generic computational provider (a worker) in an allocation, which is then able to execute any task
(based on its resource requirements), instead of directly submitting tasks into allocations. This
enables the task runtime to dynamically assign tasks to workers even across different allocations,
which helps improve the achieved hardware utilization. It also makes the partitioning process much
easier, because users simply have to spawn a set of workers and submit tasks, and the task runtime
then takes care of the rest.

Task runtimes such as \merlin{}~\cite{merlin} or
\parsl{}~\cite{parsl} support this dynamic meta-scheduling approach,
although they require users to preconfigure each worker for a specific subset of resource
requirements. For example, a worker might be configured for executing tasks that require a specific
number of cores, or tasks that require to be executed on multiple nodes etc. This can be limiting
for very heterogeneous task graphs or clusters; the task scheduler cannot make full use of all
available resources because it has to adhere to worker configurations that were predetermined by
the user.

Other task runtimes, such as \balsam{}~\cite{balsam} or
\fireworks~\cite{fireworks}, support a more flexible approach, where each worker can execute
any task whose resource requirement can be fulfilled by the resources assigned to that worker. This
enables fully dynamic load balancing across allocations and all available hardware resources.

An additional important feature that simplifies submission of workers is some form of an automatic
allocation system. As an example, \balsam{} or
\dask{}~\cite{dask} with the
\textsc{Dask JobQueue}~\cite{dask-jobqueue} plugin are able to automatically submit allocations
on behalf of the user in response to computational load. These allocations then start a generic
worker which can immediately start executing tasks. This removes an additional manual step that
would otherwise have to be performed by users.

\subsection*{Alternatives}
Apart from meta-scheduling, there are also alternative approaches to resolving limits of allocation
managers. Some works examine what it would take to make existing allocation managers more workflow
friendly. In~\cite{slurm-workflow}, a modification to the Slurm allocation manager is proposed,
which makes it workflow-aware and adds support for fine-grained resource management to it. This
approach splits tasks of a workflow submitted within a single allocation into a larger number of
allocations, and assigns them corresponding priorities and resources based on their requirements.
While this improves certain scheduling aspects of Slurm, it still creates a separate allocation for
each task, which does not remove the overhead associated with allocations. This method also assumes
prior knowledge of the workflow structure; therefore, it is not possible to easily dynamically add
tasks or computational resources while the workflow is being executed.

Another approach is to create a new kind of an allocation manager designed with tasks and workflows
in mind. An example could be the Flux~\cite{flux} \gls{hpc} scheduler,
which serves as a re-imagination of a modern allocation manager. It treats the whole cluster as a
unified pool of resources and allows submitting computations that make use of these resources with
high granularity, down to the level of individual cores. Flux also provides dynamic management of
non-computational resources such as storage and power, and even takes them into account during
scheduling, for example by avoiding \gls{io} intensive computations when not enough
\gls{io} bandwidth is currently available. Furthermore, it enables defining generic
resources, unlike traditional allocation managers that can only manage a small set of known
resources, such as \gls{cpu} cores or memory. Providing this kind of flexibility and
granularity puts a lot of pressure on the system. Flux manages it by using a distributed and
hierarchical scheduling design, where each allocation can act as a separate Flux instance that can
then recursively subdivide its resources into additional nested allocations, and even dynamically
ask for reducing or increasing its set of resources.

While modifying allocation managers to remove some of their workflow handling issues or creating
new allocation managers are viable approaches, it should be noted that Slurm and
\gls{pbs} are currently dominating the most powerful
supercomputers~\cite{slurm-schedmd}, and that replacing (or even heavily modifying) the
allocation manager of an \gls{hpc} cluster is a very difficult process, which
requires a lot of implementation work, configuration tuning, documentation rewriting and also
retraining the cluster administrators, and more importantly its users. Furthermore, using the
allocation manager directly as a task runtime has a disadvantage that was already mentioned; it
might not be so easy to run the workflows on personal computers or \gls{ci} servers,
which limits the ability to prototype and test them.

A different approach that is designed to make it easier to use \gls{hpc} clusters is
taken by tools and frameworks such as \textsc{cumulus}~\cite{cumulus},
\textsc{Open OnDemand}~\cite{openondemand}, \textsc{HEAppE}~\cite{heappe} or
\textsc{Lexis}~\cite{lexis}. These aim to simplify usage of clusters, and thus
also remove the need for users to interact with allocations manually, by providing high-level web
interfaces for managing computations. They focus on user authorization and authentication, data
management, provisioning of resources across different clusters and also partially on
(coarse-grained) workflow execution. This approach can make it easier for users that do not have
experience with the Linux operating system or interacting with the terminal to leverage
\gls{hpc} resources. However, it does not resolve the limitations of allocation
managers by itself, it simply moves the problem to a different place, and forces the high-level
computational framework to deal with it. They can do that by leveraging task runtimes which are
able to perform automatic meta-scheduling. For example, \textsc{HEAppE} provides support
for integration with the \hyperqueue{}~\cite{heappe_hq} task runtime, which will
be described in~\Autoref{ch:hyperqueue}.

\section{Cluster heterogeneity}
\label{challenge:heterogeneity}
Even though task graphs are designed to be portable and ideally should not depend on any specific
computational environment, for certain types of tasks, we need to be able to describe at least some
generic computational environment constraints. For example, when a task executes a program that
leverages the CUDA programming framework~\cite{cuda}, which is designed to be executed
on an NVIDIA graphics accelerator, it has to be executed on a node that has such a
\gls{gpu} available, otherwise it will not work properly.

It should thus be possible for a task to define \emph{resource requirements}, which specify resources
that have to be provided by an environment that will execute such a task. For example, a
requirement could be the number of cores (some tasks can use only a single core, some can be
multi-threaded), the amount of available main memory, a minimum duration required to execute the
task or (either optional or required) presence of an accelerator like a \gls{gpu} or
an \gls{fpga}. In order to remain portable and independent of a specific computational
environment, these requirements should be abstract and describe general, rather than specific,
kinds of resources.

The challenge related to resource requirements of \gls{hpc} tasks specifically is the
diverse kinds of hardware present in modern \gls{hpc} clusters, which have started to
become increasingly heterogeneous in recent years. This trend can be clearly seen in the TOP500
list of the most powerful supercomputers~\cite{top500analysis}. Individual cluster nodes contain
varying amounts and types of cores and sockets, main memory, \gls{numa} nodes or
accelerators like \glspl{gpu} or \glspl{fpga}. Since \gls{hpc}
software is often designed to leverage these modern \gls{hpc} hardware features, this
complexity is also propagated to tasks and their resource requirements.

Existing task runtimes have various levels of support for resource requirements. Most task runtimes
allow configuring at least a known set of resources per task, most often the number of
\gls{cpu} cores, amount of \gls{ram} memory or the number of
\gls{gpu} accelerators. This is enough for simple use-cases, but not for
heterogeneous clusters with e.g.\ \gls{fpga} devices or multiple kinds of
\glspl{gpu} from different vendors. Some tools, such as \dask{} or
\snakemake{}, allow users to define their own resource kinds, which can be used to
describe completely arbitrary kinds of resources. However, they do not support more complex
use-cases, such as specifying that a task requires only a fraction of a resource, that some
resources can have relations between them (for example cores residing in the same
\gls{numa} node) or that a task could support multiple sets of resource requirements.

%Some tools only allow configuring resources per the whole task graph or per a
%group of tasks (submitted within the same allocation). To achieve maximum flexibility, it should
%ideally be possible to define a custom resource requirement per each task.

A resource requirement that is fairly specific to \gls{hpc} systems is the usage of
multiple nodes per single task. This requirement is necessary for programs that are designed to be
executed in a distributed fashion, such as programs using \gls{mpi}, which are quite
common in the \gls{hpc} world. The use-case of tasks using multiple nodes is
discussed in more detail later in~\Autoref{sec:multinode-tasks}.

%Data outputs produced by tasks tend to be considered immutable in existing task runtimes, since a
%single output can be used as an input to multiple tasks, and these might be executed on completely
%different computational nodes; therefore, it would be challenging to synchronize concurrent
%mutations of these objects. A problem that can arise with this approach is that if the data outputs
%are large, but the computation of tasks that work with the data is short, then the serialization
%overhead (or even memory copy overhead, if the dependent task is executed on the same node) can
%dominate the execution time. Such use-cases can be solved with stateful data management, for
%example in the form of \emph{actors}, which can be seen as stateful tasks that operate on
%a single copy of some large piece of data, without the need to transfer or copy it. The actor model
%is supported e.g.\ by \dask{} or \ray{}.

\section{Performance and scalability}
The massive scale of \gls{hpc} hardware (node count, core count, network interconnect
bandwidth) opens up opportunities for executing large-scale task graphs, but that in turn presents
unique challenges for task runtimes. Below you can find several examples of bottlenecks that might
not matter in a small computational scale, but that can become problematic in the context of
\gls{hpc}-scale task graphs.

\begin{description}[wide=0pt]
	\item[Scheduling] Task scheduling is one of the most important responsibilities of a task runtime, and with large
		task graphs, it can become a serious performance bottleneck. Existing task runtimes, such as
		\dask{}, can have problems with keeping up with the scale of \gls{hpc}
		task graphs. The scheduling performance of various task scheduling algorithms will be examined in
		detail in~\Autoref{ch:estee}.
	\item[Runtime overhead] Using a task runtime to execute a task graph will necessarily introduce some amount of overhead,
		caused by scheduling, network communication and other actions performed by the task runtime. Task
		runtimes should be mindful of the overhead they add, because it can severely affect the duration
		needed to execute a task graph. For example, even with an overhead of just \SI{1}{\milli\second}
		per task, executing a task graph with a million tasks would result in total accumulated overhead of
		more than fifteen minutes.

		Many popular task runtimes, e.g.\ \dask{}, \parsl{},
		\balsam{}, \autosubmit{} or \snakemake{} are implemented in the
		Python programming language. While Python provides a lot of benefits in terms of an ergonomic
		interface for defining task graphs, using it for the actual implementation of scheduling, network
		communication and other performance-critical parts of the task runtime can lead to severe
		bottlenecks, as Python programs are infamously difficult to optimize well.

		This aspect will be examined in detail in~\Autoref{ch:rsds}, where we study the performance
		and overhead of \dask{}, a state-of-the-art task runtime implemented of Python.

	\item[Architecture] A typical architecture of a task runtime consists of a centralized component that handles task
		scheduling, and a set of workers that connect to it over the network and receive task assignments.
		Even with a central server, the task runtime can achieve good performance, as we will show in
		Chapters~\ref{ch:rsds} and~\ref{ch:hyperqueue}. However, the performance of the server
		cannot be increased endlessly, and at some point, a centralized architecture will become a
		bottleneck. Even if the workers exchange data outputs directly between themselves, the central
		component might become overloaded simply by coordinating a vast number of workers.

		In that case, a decentralized architecture could be leveraged to avoid the reliance on a central
		component. Such a decentralized architecture can be found e.g.\ in
		\ray{}~\cite{ray}. However, to realize the gains of a decentralized
		architecture, task submission itself has to be decentralized in some way, which might not be a
		natural fit for common task graph workflows. If all tasks are generated from a single location, the
		bottleneck will most likely remain even in an otherwise fully decentralized system.
	\item[Communication overhead] Scaling the number of tasks and workers will necessarily put a lot of pressure on the communication
		network, both in terms of bandwidth (sending large task outputs between nodes) and latency (sending
		small management messages between the scheduler and the workers). Using \gls{hpc}
		technologies, such as \gls{mpi} or a lower-level interface like
		\gls{rdma}, could provide a non-trivial performance boost in this regard. Some existing
		runtimes, such as \dask{}, can make use of such technologies~\cite{dask-ucx}.

		As we have demonstrated in~\cite{pspin, spin2}, in-network computing can be also used to
		optimize various networking applications by offloading some computations to an accelerated
		\gls{nic}. This approach could also be leveraged in task runtimes, for example to
		reduce the latency of management messages between the scheduler and workers or to increase the
		bandwidth of large data exchanges among workers, by moving these operations directly onto the
		network card. This line of research is not pursued in this thesis, although it could serve as a
		future research direction.
	\item[Task graph materialization] Large computations might require building massive task graphs that contain millions of tasks. The
		task graphs are typically defined and built outside of computational nodes, e.g.\ on login nodes of
		computing clusters or on client devices (e.g.\ laptops), whose performance can be limited. It can
		take a lot of time to build, serialize and transfer such graphs over the network to the task
		runtime that runs on powerful computational nodes. This can create a bottleneck even before any
		task is executed; it has been identified as an issue in existing task
		runtimes~\cite{dask-client-perf}.

		In such case, it can be beneficial to provide an \gls{api} for defining task graphs in
		a symbolic way that could represent a potentially large group of similar tasks with a compressed
		representation to reduce the amount of consumed memory. For example, if we want to express a
		thousand tasks that all share the same configuration, and differ e.g.\ only in an input file that
		they work with, we could represent this as a group of thousand similar tasks, rather than storing a
		thousand individual task instances in memory. This could be seen as an example of the classical
		Flyweight design pattern~\cite{gof}.

		Such symbolic graphs could then be sent to the runtime in a compressed form and re-materialized
		only at the last possible moment. In an extreme form, the runtime could operate on such graphs in a
		fully symbolic way, without ever materializing them. \dask{} is an example of a task
		runtime that supports such symbolic task graph representations.
	\item[Data transfers] After a task is computed, it can produce various outputs, such as standard error or output streams,
		files created on a filesystem or data objects that are then passed as inputs to dependent tasks.
		Managing these data streams can be a bottleneck if the number of tasks is large.

		Some task runtimes, such as \snakemake{}, store all task outputs on the filesystem, since
		it is relatively simple to implement and it provides support for basic data resiliency
		out-of-the-box. A lot of existing software (that might be executed by a task) also makes liberal
		use of the filesystem, which can make it challenging to avoid filesystem access altogether.
		However, \gls{hpc} nodes might not contain any local disks. Instead, they tend to use
		shared filesystems accessed over a network. While this can be seen as an advantage, since with a
		shared filesystem it is much easier to share task outputs among different workers, it can also be a
		severe bottleneck. Shared networked filesystems can suffer from high latency and accessing them can
		consume precious network bandwidth that is also used e.g.\ for managing computation (sending
		commands to workers) or for direct worker-to-worker data exchange. Furthermore, data produced in
		\gls{hpc} computations can be large, and thus storing it to a disk can be a bottleneck
		even without considering networked filesystems.

		These bottlenecks can be alleviated by transferring data directly over the network (preferably
		without accessing the filesystem in the fast path). Direct network transfer of task outputs between
		workers is implemented e.g.\ by \dask{}. Some runtimes (such as
		\textsc{HyperLoom}~\cite{hyperloom}) also leverage \gls{ram} disks, which
		provide support for tasks that need to interact with a filesystem while avoiding the performance
		bottlenecks associated with disk accesses. Some task runtimes, such as \hypershell{} or
		\pegasus{}, allow streaming standard output and error streams over the network to a
		centralized location to reduce filesystem pressure. For all of these approaches, it is also
		possible to use \gls{hpc}-specific technologies, such as
		InfiniBand~\cite{infiniband} or \gls{mpi}, to improve data transfer performance
		by fully exploiting the incredibly fast interconnects available in modern \gls{hpc}
		clusters.
\end{description}

\section{Fault tolerance}
Fault tolerance is relevant in all distributed computational environments, but
\gls{hpc} systems have specific requirements in this regard. As was already
mentioned, computational resources on \gls{hpc} clusters are provided through
allocation managers. Computational nodes allocated by these managers are provided only for a
limited duration, which means that for long-running computations, some nodes will disconnect and
new nodes might appear dynamically during the lifecycle of the executed workflow. Furthermore,
since the allocations go through a queue, it can take some time before new computational resources
are available. Therefore, the computation can remain in a paused state, where no tasks are being
executed, for potentially long periods of time.

It is important for task runtimes to be prepared for these situations; they should handle node
disconnections gracefully even if a task is being executed on a node that suddenly disconnects, and
they should be able to restart previously interrupted tasks on newly arrived workers. In general,
in \gls{hpc} scenarios, worker instability and frequent disconnects should be
considered the normal mode of operation rather than just a rare edge case.

Tasks are usually considered to be atomic from the perspective of the runtime, i.e.\ they either
execute completely (and successfully), or they fail, in which case they might be restarted from
scratch. Task granularity thus plays an important role here, since when a task is large, then a lot
of work might have to be redone if it fails and is re-executed. Some runtimes try to alleviate this
cost by leveraging task checkpointing~\cite{task_checkpointing}, which is able to persist the
computational state of a task during its execution, and then restore it in case of a failure, thus
avoiding the need to start from the beginning.

Fault tolerance can be challenging in the presence data transfers between dependent tasks. When a
task requires inputs from its dependency, the runtime might have to store them (either in memory or
in a serialized format on disk) even after the task has started executing. Because there is always
a possibility that it will have to restart the task in case of a failure, and thus it needs to hold
on to its inputs. In some cases, it can actually be a better trade-off to avoid storing the inputs
and instead re-execute the dependencies of the task (even if they have been executed successfully
before) to regenerate the inputs. This can help reduce memory footprint, albeit at the cost of
additional computation time, and it also might not work well if the execution of dependencies are
not deterministic.

Because of its importance, fault-tolerant task execution is generally well-supported in most
existing task runtimes. The differences are mostly in the level of automation. For example,
\fireworks{} or \merlin{} require users to manually restart failed tasks,
while runtimes such as \dask{} or \balsam{} can restart them
automatically without user intervention.

An additional aspect of fault tolerance is persistence of submitted tasks. It can be useful to have
the ability to resume task graph computation if the whole task runtime infrastructure (e.g.\ its
primary server) crashes, to avoid needless recomputations. Some task runtimes, such as
\dask{}, do not support such fault tolerance at all, others support it optionally,
such as \ray{}, while some task runtimes are persistent by default, such as
\balsam{}.

\section{Multi-node tasks}
\label{sec:multinode-tasks}
Many existing \gls{hpc} applications are designed to be executed on multiple
(potentially hundreds or even thousands) nodes in parallel, using e.g.\ \gls{mpi} or
other communication frameworks. Multi-node execution could be seen as a special kind of resource
requirement, which states that a task should be executed on multiple workers at once. Support for
multi-node tasks is challenging, because it affects many design areas of a task runtime:
\begin{description}[wide=0pt]
	\item[Scheduling] When a task requires multiple nodes for execution and not enough nodes are available at a given
		moment, the scheduler has to decide on a strategy that will allow the multi-node task to execute.
		If it were constantly trying to backfill available workers with single-node tasks, then multi-node
		tasks could be starved.

		The scheduler might thus have to resort to keeping some nodes idle for a while to enable the
		multi-node task to start as soon as possible. Another approach could be to interrupt the currently
		executing tasks and checkpoint their state to make space for a multi-node task, and then resume
		their execution once the multi-node task finishes.

		In a way, this decision-making already has to be performed on the level of individual cores even
		for single-node tasks, but adding multiple nodes per task makes the problem much more difficult.
	\item[Data transfers] It is relatively straightforward to express data transfers between single-node tasks in a task
		graph, where a task produces a set of complete data objects, which can then be used as inputs for
		dependent tasks. With multi-node tasks, the data distribution patterns become more complex, because
		when a task that is executed on multiple nodes produces a data object, the object itself might be
		distributed across multiple workers, which makes it more challenging to use in follow-up tasks. The
		task graph semantics might have to be extended by expressing various data distribution strategies,
		for example a reduction of data objects from multiple nodes to a single node, to support this
		use-case. This use-case is supported e.g.\ by \pycompss~\cite{pycompss}.

		When several multi-node tasks depend on one another, the task runtime should be able to exchange
		data between them in an efficient manner. This might require some cooperation with the used
		communication framework (e.g.\ \gls{mpi}) to avoid needless repeated serialization and
		deserialization of data between nodes.
	\item[Fault tolerance] When a node executing a single-node task crashes or disconnects from the runtime, its task can be
		rescheduled to a different worker. In the case of multi-node tasks, failure handling is generally
		more complex. For example, when a task is being executed on four nodes and one of them fails, the
		runtime has to make sure that the other nodes will be notified of this situation, so that they can
		react accordingly (either by finishing the task with a smaller number of nodes or by failing
		immediately).
\end{description}

Some task runtimes only consider single-node tasks in their programming model, which forces users
to use various workarounds, such as emulating a multi-node task with a single-node task that uses
additional resources that are not managed by the task runtime itself. In other tools, such as
\autosubmit{}, it is only possible to express node counts on the level of individually
submitted allocations. While this is useful for running coarse-grained \gls{mpi}
applications, it does not allow combining these multi-node applications with other, fine-grained
single-node tasks in the same task graph. One notable exception is
\pycompss{}~\cite{pycompss}, which allows decorating Python functions that
represent tasks with an annotation that specifies how many nodes does that task require, and
combine these tasks with other single-node tasks.

%To enable common \gls{hpc} use-cases, task runtimes should be able to provide
%first-class support for multi-node tasks and allow them to be combined with single-node tasks in a
%seamless manner. Advanced multi-node task support could be provided e.g.\ by offering some kind of
%built-in integration with \gls{mpi} or similar common \gls{hpc}
%technologies.

\section{Deployment}
\label{challenge:deployment}
Even though it might sound trivial at first, an important aspect that can affect the ergonomics of
executing task graphs (or any kind of computation, in general) on an \gls{hpc}
cluster is ease-of-deployment. Supercomputing clusters are notable for providing only a severely
locked-down environment for their users, which does not grant elevated privileges and requires
users to either compile and assemble the build and runtime dependencies of their tools from scratch
or choose from a limited set of precompiled dependencies available on the cluster. However, these
precompiled dependencies can be outdated or incompatible with one another.

Deploying any kind of software (including task runtimes) that has non-trivial runtime dependencies
or non-trivial installation steps (if it is not available in a precompiled form for the target
cluster) can serve as a barrier for using it on an \gls{hpc} cluster. Many existing
task runtimes are not trivial to deploy. For example, several task runtimes, such as
\dask{}, \snakemake{} or \pycompss{}, are implemented in
Python, a language that typically needs an interpreter to be executed, and also a set of packages
in specific versions that need to be available on the filesystem of the target cluster. Installing
the correct version of a Python interpreter and Python package dependencies can already be slightly
challenging in some cases, but there can also be other issues, such as the dependencies of the task
runtime conflicting with the dependencies of the software executed by the task graph itself. For
example, since \dask{} is a Python package, it depends on a certain version of a
Python interpreter and a certain set of Python packages. It supports tasks that can directly
execute Python code in the process of a \dask{} worker. However, if the executed
code requires a different version of a Python interpreter or Python packages than
\dask{} itself, this can lead to runtime errors or subtle behavior
changes\footnote{It is not straightforward to use multiple versions of the same Python package dependency in a
single Python virtual environment, because conflicting versions simply override each other.}.

A much bigger challenge is if a task runtime requires some external runtime dependencies. For
example, \merlin{} requires a message broker backend, such as RabbitMQ or Redis,
\fireworks{} needs a running instance of the MongoDB database, and
\balsam{} uses the PostgreSQL database for storing task metadata. Compiling,
configuring and deploying these complex runtime dependencies on \gls{hpc} clusters
can be quite challenging.

\section{Programming model}
One important aspect that affects the simplicity of defining and executing task graphs is the
programming model and interfaces used to describe the execution logic, resource requirements,
dependencies and other properties of tasks and task graphs. Below is a list of areas that affect
the ergonomic aspects of task graph definition.

\begin{description}[wide=0pt]
	\item[Task definition interfaces] The method used to define tasks affects how easy it is to work with the task runtime, and how
		complex use-cases can be expressed with it. Existing interfaces usually belong to one of three
		broad categories. Tools such as \gnuparallel{} or \hypershell{} offer a
		\gls{cli}, which enables defining tasks that should be computed directly in the
		terminal. This provides a quick way to submit structurally simple task graphs, although it is
		difficult to describe complex dependencies between tasks using this approach. A more general method
		of defining task graphs is to use a declarative workflow file, often using the
		\gls{yaml} format. This approach is used e.g.\ by \pegasus{},
		\autosubmit{} or \merlin{}. Since they are declarative, workflow files make
		it easier to analyze the task graph description, for example for providing data provenance.
		However, they can also be incredibly verbose, and are not suited for task graphs with a large
		number of tasks (unless the workflow file is automatically generated or it supports some form of
		compressed representation for groups of tasks). The most general (and very popular) method for
		describing task graphs is to leverage a general purpose programming language, very often Python.
		Many task runtimes provide a Python \gls{api} for defining tasks, e.g.\
		\dask{}, \ray{}, \parsl{}, \balsam{} or
		\pegasus{}. This approach allows users to create task graphs however they like,
		although it can be unnecessarily verbose for simple use-cases.

	\item[Data transfers] Direct transfers of data between tasks have already been mentioned as a potential performance
		bottleneck, yet they also affect the ease-of-use of a given programming model. Some use-cases are
		naturally expressed by tasks passing their outputs directly to depending tasks. Support for this
		feature can make such use-cases much simpler, by not forcing users to pass intermediate task
		outputs through the filesystem.

	\item[Iterative computation] Another important aspect of the programming model is its support for iterative computation. There
		are various \gls{hpc} use-cases that are inherently iterative, which means that they
		perform some computation repeatedly, until some condition (which is often determined dynamically)
		is satisfied. For example, the training of a machine-learning model is typically performed in
		iterations (called epochs) that continue executing while the prediction error of the model keeps
		decreasing. Another example could be a molecular dynamics simulation that is repeated until a
		desired level of accuracy has been reached.

		One approach to model such iterative computations would be to run the whole iterative process
		inside a single task. While this is simple to implement, it might not be very practical, since such
		iterative processes can take a long time to execute, and performing them in a single task would
		mean that we could not leverage desirable properties offered by the task graph abstraction, for
		example fault tolerance. Since the computation within a single task is typically opaque to the task
		runtime, if the task fails, it would then need to be restarted from scratch.

		A better approach might be to model each iteration as a separate task. In such case, the task
		runtime is able to restart the computation from the last iteration if a failure occurs. However,
		this approach can be problematic if the number of iterations is not known in advance, since some
		task runtimes expect that the structure of the task graph will be immutable once the graph has been
		submitted for execution. \dask{} and \ray{} are examples of task
		runtimes do not have this assumption; they allow adding tasks to task graphs dynamically during
		their execution.

		Another option to handle iterative tasks is provided by
		\textsc{L-DAG}~\cite{l-dag}, which suggests an approach for transforming
		iterative workflows into workflows without loops.
\end{description}

\section*{Summary}
This chapter has identified a set of challenges that can limit the usage ergonomics and efficiency
of task graphs on supercomputers, and described how do existing task runtimes deal with them. Even
though more \gls{hpc} peculiarities can always be found, it should be already clear
from all the mentioned challenges that \gls{hpc} use-cases that leverage task graphs
can be very complex and may require specialized approaches in order to reach maximum efficiency
while remaining easy to use and deploy. The most important challenge that was described is the
interaction of tasks with allocations, which has a fundamental effect on the achievable hardware
utilization of the target cluster, and on the overall simplicity of defining and executing task
graphs.

The rest of the thesis will discuss approaches for alleviating these challenges.
Chapters~\ref{ch:estee} and~\ref{ch:rsds} focus on the performance aspects of
task graph execution, namely on task scheduling and the overhead introduced by task
runtimes.~\Autoref{ch:hyperqueue} then also deals with the ergonomic challenges, and proposes a
meta-scheduling and resource management design that is built from the ground up with the mentioned
challenges in mind, to enable truly first-class task graph execution on heterogeneous
\gls{hpc} clusters.
