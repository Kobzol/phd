There are many ways to design and program distributed and parallel programs for HPC clusters.
Historically, a popular way of creating HPC applications was to implement the parallelization of
computation and the distribution of data explicitly, typically using message-passing (e.g.\
MPI\footnote{Message Passing Interface}~\cite{mpi}), global address space (e.g.\
PGAS\footnote{Partitioned global address space}~\cite{pgas}) and/or multithreading (e.g.\
OpenMP\footnote{Open Multi-Processing}~\cite{openmp}) frameworks, which are designed and fine-tuned
for HPC use-cases. With an explicitly parallel program, users have a lot of control over the
exchange of data and communication between individual cores and distributed nodes, which allows
them to create very performant programs. This approach is also quite flexible and enables
expressing arbitrarily complex parallelization and data distribution strategies.

However, even though such explicitly parallelized programs can be incredibly efficient,
implementing them is notoriously difficult. Deadlocks and race conditions, which are already
problematic for multithreaded programs, are even more of an issue for distributed programs. HPC
applications that use libraries for explicit distribution of data (such as various MPI
implementations) are also typically implemented in languages such as \texttt{C} or
\texttt{C++} that are infamous for making it difficult to write correct, memory-safe
programs without memory errors and undefined behaviour. This further increases the difficulty and
decreases the speed of developing correct distributed programs.

It would be unreasonable to expect that all scientists that want to leverage HPC to execute their
experiments will ``roll up their sleeves'' and spend months implementing an explicitly parallel
\texttt{C++} program that uses MPI. A more typical scenario is that scientists leverage
existing coarse-grained tools and frameworks that are optimized for HPC and use technologies like
MPI or OpenMP internally (e.g.\ GROMACS~\cite{gromacs}). However, with both the experiments
and HPC clusters becoming ever more complex each year, it is usually not sufficient to use only a
single tool. Scientific experiments require running many independent steps, which encompass data
transfer, preprocessing, postprocessing and the usage of potentially many different tools that have
to be combined. These experiments can also be executed in potentially many instances that have
different input parametrizations.

That is why in recent years, it became popular to define scientific computations using programming
models where the high-level communication and parallelization structure of the computation is
derived automatically from the program description. There are multiple ways of achieving that, such
as using e.g.\ the MapReduce~\cite{mapreduce} model, however this thesis focuses on one
approach in particular -- scientific workflows, also called pipelines. Workflows can describe a
complex computation composed of potentially many different tool invocations that can depend on one
another. They are often written in very high-level languages, such as Python, which allows users to
focus on defining the computation easily and also to quickly prototype. Crucially, these workflows
are typically based on a programming model that does not require explicit implementation of data
exchange and parallelism -- the task-based programming model.

The task-based programming model allows users to describe the high-level structure of the
computations that they want to perform using a \emph{task graph}. Users create the task graph
by splitting their desired computation into a set of \emph{tasks}, atomic and independent
computational blocks with explicitly described inputs and outputs that can be executed in a
self-contained way. Additional constraints can also be encoded into the task graph, e.g.\ which
data should be transferred between the individual tasks, which tasks cannot be executed until some
other task finishes its execution, or the necessary properties of an environment in which a task
will be executed.

The primary benefit of describing a computational workflow in such a way is that the resulting
program is \emph{implicitly parallel}. The author of the program does not imperatively specify how
should the computation be parallelized or when and how should data be exchanged. They merely
declaratively describe the individual parts of the program that can theoretically be executed in
parallel (the tasks) and then pass the task graph to a dedicated execution tool that executes the
tasks on a parallel machine (or even on a distributed cluster). Since the program is represented
with an explicit graph, the execution tool can effectively analyze its properties (or even optimize
the structure of the graph) in an automated way, and extract the available parallelism from it
without requiring the user to define parallelization opportunities explicitly.

Note that from the perspective of a task execution tool, each task is an opaque element. The tool
knows how to execute it, but it typically does not have any further knowledge of the inner
structure of the task. Therefore, the only parallelization opportunities that can be extracted by
the tool have to be expressed by the structure of the task graph. A task graph containing a single
task is thus not very useful on its own.

The task-based programming model is central to the topic of this thesis, and thus this chapter will
define various terms and concepts related to it. Note that there are many variations of this
programming model, based on the specific tool or environment where it is used, and thus the term
\emph{task-based programming} is heavily overloaded. It is used in various related technologies, ranging
from fine-grained tasks that execute a single function or just a handful of
instructions~\cite{starpu,openmp} to coarse-grained task workflows that execute binaries which
can run for hours or even days~\cite{dask, snakemake, nextflow}. This thesis primarily focuses on the latter
type of task graphs, which represent very general computations (either functions or binaries) with
various levels of granularity, that are intended to be distributed among multiple nodes of a
distributed cluster. A further distinction can be made between batch-oriented workflows, which
define a computation with a clearly delimited set of inputs that is already available when the
workflow starts, and which focus primarily on throughput, and streaming-oriented workflows, which
operate continuously on streams of input data that arrive just-in-time, and which usually focus on
latency. This thesis focuses exclusively on batch workflows, which are more common in HPC
scenarios\todo{cite}.

The term \emph{task} is also heavily overloaded in various areas of computer science, as
it is used for many unrelated concepts, from an execution context in the Linux operating system,
through a block of code executed by the OpenMP library, to a program executed by a complex
distributed computational workflow. Even in the area of distributed computing, the terminology can
vastly differ between different task execution tools and theoretical works.

Even though task-related terms can have a lot of different meanings, depending on the context, and
an attempt of providing a single unifying task theory would need to gloss over many important
details, it is still useful to provide a shared vocabulary of terms for this thesis. This chapter
thus defines several task-related terms in a way that allows capturing the specifics of task
execution in HPC environments, further described in Chapter~\ref{ch:challenges}, and that is
general enough so that it can be mapped to concepts used by task execution tools that will be
described in later chapters.

\section{Task graph}
A computational workflow in the task-based programming model is represented with a directed acyclic
graph (DAG) that we will label as a \emph{task graph}. From a high-level perspective, it
describes which individual steps should be performed, what are the constraints for where and in
which order are they computed and how should data be transferred between the individual steps of
the workflow.

There are many variations of task graphs, based on the computational properties that they are able
to describe. In the most basic form, task graph vertices represent computations to be performed,
and the edges represent dependencies between the computations, which enforce an order in which will
the computations be performed. However, numerous other concepts can also be encoded in a task
graph. For example, in addition to dependencies, the edges could represent abstract communication
channels, through which the outputs of one computation are transferred to become the inputs of
another computation that depends on it. There could be a special type of edge which specifies that
the outputs of a computation will be streamed to a follow-up computation, instead of being
transferred only after the previous computation has finished. There could also be e.g. a special
type of node that defines iterative computation that is performed repeatedly until some condition
is met.

The exact semantics of vertices and edges of task graphs depend heavily on the specifics of tools
that implement them, and thus it is not possible to have a single definition that would fit all
variants used ``in the wild''. To provide a general description, we will formally define a task
graph designed for batch workflows, which can capture the structure of dependencies between
computations and the notion of transferring data between them. This definition is general enough so
that it can be mapped to the task execution tools described later in this thesis.

A task graph is a tuple $(T, O, E)$, where $T$ is a set of
\emph{tasks}, $O$ is a set of \emph{data objects} and
$E \subseteq ((T\times{}O) \cup (O\times{}T))$ is a set of arcs. $(T \cup O, E)$ forms a finite directed acyclic
graph. The graph is structured in a way so that for every data object, there is exactly one task
that produces it: $\forall o\in{}O: (\exists t_1\in{}T: (t_1, o) \in E \land
	(\forall
	t_2\in{}T: (t_2, o) \in E \Rightarrow t_1 = t_2))$.

Below we will define several terms and properties that will be useful for working practically with
task graphs:
\begin{enumerate}
	\item If there is an arc between a task and a data object ($(t,o) \in (T\times{}O)$), then we call
	      $t$ the \emph{producer} of $o$ and $o$
	      the \emph{output} of $t$.
	\item If there is an arc between a data object and a task ($(o,t) \in (O\times{}T)$), then we call
	      $t$ the \emph{consumer} of $o$ and $o$
	      the \emph{input} of $t$.

	\item Let us introduce a binary relation $D_r$ over the set of tasks:
	      $\{(t_1, t_2): (t_1, t_2)\in{}(T\times{}T)\land t_1 \neq t_2 \land
		      \exists{}o\in{}O: (t_1, o)\in{}E
		      \land (o, t_2)\in{}E\}$. When $(t_1, t_2) \in D_r$, we say that $t_2$
	      \emph{directly depends} on $t_1$. We can also state that $t_2$
	      consumes the output produced by $t_1$.

	\item Let us introduce a binary relation $D$ over the set of tasks. Tasks
	      $t_1$ and $t_n$ are in this relation if there is a sequence
	      $(t_1, t_2, \ldots, t_n)$ such that $\forall i \in \{
		      1,2,\ldots,n - 1\}: (t_i, t_{i+1}) \in D_r$. When $(t_1, t_2) \in D$, we say that
	      $t_2$ \emph{depends} on $t_1$ and that
	      $t_1$ is a \emph{dependency} of $t_2$.

	\item We label tasks without any dependencies \emph{source tasks}. $S = \{ t \mid t\in{}T \land \forall{}t_d\in{}T:
		      (t_d, t)\notin D\}$. It is a
	      simple observation that unless the task graph is empty ($(T\cup{}O) = \emptyset$), there is always at
	      least one source task in the graph, because the graph is acyclic.
	\item We label tasks that are not depended upon by any other task \emph{leaf tasks}.
	      $L = \{ t \mid t\in{}T \land \forall{}t_d\in{}T: (t,
		      t_d)\notin D\}$.
\end{enumerate}

An example of a simple task graph is shown on Figure~\ref{fig:task-graph-example}. Tasks are represented
as circles, data objects as (rounded) rectangles\footnote{This representation will be used in all following task graph diagrams.} and arcs as arrows. The
source task $t_0$ generates two data objects, which are then used as inputs for
four additional tasks. The outputs of these four tasks are then aggregated by a final task
$t_5$. This can correspond e.g.\ to a workflow where $t_0$
generates some data, $t_{1-4}$ performs some calculation on that data and
$t_5$ performs some post-processing step and stores the results to disk.

\begin{figure}[h]
	\centering
	\resizebox{!}{35mm}{
	\begin{tikzpicture}
            \graph[
                grow right sep=6mm,
            ] {
                "$t_0$"[task] -> {
                    "$d_{0a}$"[data] -> {
                        "$t_1$"[task] -> "$d_{1}$"[data],
                        "$t_2$"[task] -> "$d_{2}$"[data],
                        "$t_3$"[task] -> "$d_{3}$"[data]
                    },
                    "$d_{0b}$"[data] -> {
                        "$t_4$"[task] -> "$d_{4}$"[data]
                    }
                } -> "$t_5$"[task]
            };
        \end{tikzpicture}
	}
	\caption{Simple task graph with six tasks and six data objects}
	\label{fig:task-graph-example}
\end{figure}

Note that the presented definition of a task graph does not describe its semantics -- how will the
graph be created and executed and what will be the interactions between tasks and data objects.
This depends on the specific tool or framework which uses task graphs to define its programming
model. For example, in a distributed batch oriented setting, if a task $t_2$
depends on another task $t_1$, then $t_2$ cannot start to execute
until $t_1$ has finished executing and the data objects produced by
$t_1$ have been transferred to the computational node that will execute
$t_2$.

\emph{Data object} is commonly represented as a serialized blob of data that has to be
transferred from the node where its producer was computed to the node where its consumer is to be
executed. If a task programming model does not encode any direct transfer of data between tasks,
then the data objects serve simply as ``empty'' markers of dependencies, and they do not hold any
actual data. In that case, we could even remove them from the task graph completely, and represent
direct dependencies with arcs between tasks.

\emph{Task} is a serializable description of a computation that can be executed
repeatedly. The serializability property is crucial, as it allow us to treat computations as data,
which is quite powerful, because it allows tasks to be sent between different nodes in a cluster or
stored to disk and to be transparently recomputed an arbitrary number of times. Tasks might need to
be recomputed e.g.\ after some failure occurs during their execution.

In practice, a single task will typically represent either the invocation a function (a block of
code that can be executed in a self-contained way) or the execution of a program (an executable
binary). Multiple tasks in a task graph can execute the same function or executable, and this is in
fact the common case, as task graphs are often used to parametrize a single computation with many
different input parameters.

Even though in the formal definition the inputs and outputs of tasks are sets, in practice they are
usually represented either with ordered sequences or some mapping that associates a name with each
input or output, because it is important to distinguish the identity and/or the order of each input
and output. For functions, inputs are its arguments, and output is its return value, which can be
structured so that a sequence of outputs is generated, instead of just a single output. For binary
programs, inputs are e.g.\ command-line arguments or the content of its \texttt{standard input stream}, and
the output can be e.g.\ the content of the \texttt{standard output stream} generated by the executed
program.

While inputs and outputs explicitly describe how data flows in and out of tasks, when a task is
executed, it can additionally also read the state of the environment in which it is being executed,
or change the state in a way that is observable by other tasks. For example, a function can read or
modify the value of a global variable, or an executable can read an environment variable or create
a file on a disk which is not specified as a task output. Such actions, which we will label as
\emph{side effects}, are typically not encoded within the task graph. Tasks should ideally
contain as few side effects as possible, because they can make task execution non-deterministic, by
causing it to produce different outputs when executed multiple times, which might not be desired.
To actually produce some output, most task graphs will eventually store some data to a file-system,
which can be either considered a side effect, or a first-class operation, depending on the
semantics of the task execution tool.

Apart from specifying the structure of required inputs and produced outputs, tasks can also define
various constraints for the environment in which they will be executed. We will use the term
\emph{resource requirements} for such constraints. As an example, a task that performs training of a
machine learning model might require a GPU (Graphics Processing Unit) to be present on the
computational node where the task will be executed. Other resources might include e.g.\ a minimum
required amount of memory (RAM) or a required amount of processor cores necessary to execute the
task.

\section{Task graph execution}
Task graphs merely describe some computation, therefore they have to be executed in order to
actually produce some results (outputs). This is the responsibility of a \emph{task runtime}, a
tool that analyzes task graphs and executes them in some \emph{computational environment}, e.g.\ a personal
computer or a distributed cluster. Such an environment contains a set of computational providers
that are able to execute tasks, these providers are commonly labeled as \emph{workers}. The
\emph{execution} of a task is the act of performing its described computation by a specific
worker (or a set of workers) with specific input(s), and producing some output(s).

There are many existing task runtimes, with varying architectures, features and trade-offs, which
affect factors like performance, fault tolerance or expressivity of the supported variant of the
task-based programming model. Some of these existing runtimes will be described in more detail in
Chapter~\ref{ch:sota}. We will consider a common scenario in the rest of this chapter,
which is a distributed task runtime with a centralized management component and a set of workers
that communicate with it via network or inter-process comunication.

In general, a task runtime manages and monitors all aspects of task graph execution; primarily it
manages tasks and workers. This agenda can be quite complex especially in a distributed setting,
where the workers operate on remote computational nodes connected by a network.

Worker management involves handling the lifetime of workers (their connection/disconnection),
facilitating data transfers between them or providing resiliency in case of worker failures. A
single worker is typically a program running on a computational node, which is connected to the
management component of the runtime. It receives commands from it, executes tasks and sends
information about task execution status back to it. Each worker typically manages some hardware
resources that are available for tasks during their execution. Hardware resources can be assigned
to workers in various ways. There can be a single worker per whole computational node, or there
could be multiple workers per node, each one managing a subset of the available resources (e.g.\ a
worker per CPU core).

The second main aspect that has to be handled by the runtime is the management of tasks. It has to
keep track of which tasks have already been computed, which tasks are currently being executed on
some worker(s) or which tasks are ready to be executed next because their dependencies have already
been computed. Two important responsibilities in this area are fault tolerance and scheduling.

Fault tolerance is the ability to gracefully handle task execution failures, and provide ways of
retrying failed computations. When the execution of a task fails with some error condition (e.g.\
because a worker executing the task crashes), a fault-tolerant task runtime will be able to
transparently restart it by launching a new execution of that task. We will use the term
\emph{task instance} for a specific execution of a task. Runtimes might impose some limits on
retrying failed tasks, e.g.\ by attempting to execute up to a fixed amount of task instances for
each task before giving up, to avoid endless failure loops.

Tasks are usually considered to be atomic from the perspective of the runtime, i.e.\ they either
execute completely (and successfully), or they fail, and then they have to be restarted from
scratch. Granularity thus plays an important factor here -- if a task is large, then a lot of work
might have to be redone if it fails and is re-executed. Some runtimes try to alleviate this cost by
enabling task checkpointing, which is able to save a computational state during task execution and
then restore computation from it in case of a failure, thus avoiding the need to start from the
beginning.

Fault tolerance is challenging in the presence of dependencies. When a task has inputs, the runtime
might have to store them (either in memory or in a serialized format on disk) even after the task
has started executing. Because there is always a possibility that it will have to restart the task
in case of a failure, and thus it needs to hold on to its inputs. In some cases it can be a better
trade-off to avoid storing the inputs and instead re-execute the dependencies of the task (even if
they have executed successfully before) to re-generate the inputs. This can help reduce memory
footprint, albeit at the cost of time, and it also might not work well if the dependencies are not
deterministic.

Note that the fact that we are even able to execute a task multiple times is one of the core
advantages of the task-based programming model, where tasks declaratively describe a self-contained
computation that can be re-executed arbitrarily. This crucial property of tasks makes
fault-tolerant execution of task graphs possible.

\section{Task scheduling}
Arguably the most important responsibility of a task runtime, and one that gets the most attention
in research works, is \emph{task scheduling}. It is the act of deciding in which order and on which
specific workers should each task execute, in a way that optimizes some key metric. There are many
metrics that can be optimized for, such as the latency to execute specific critical tasks, but the
most common metric is \emph{makespan} -- the duration between the start of the execution of
the first task to the completion of all tasks within the task graph. Scheduling is a crucial
activity of the runtime, as it has a profound effect on the efficiency of the whole workflow
execution. We will use the term \emph{scheduler} for a component of the task runtime that is
responsible for assigning tasks to workers.

%TODO: define what is a schedule

Figure~\ref{fig:scheduling-example} shows an example of how a schedule of a simple task graph looks like,
and how a trivial change in the schedule can severely affect the resulting makespan. The figure
contains a task graph with four tasks and two data objects. The size of the circles is proportional
to the duration of the tasks and the size of the rectangles is proportional to the size of the data
objects. Let us assume that we want to schedule this task graph to a cluster with three workers
($w_0$, $w_1$, $w_2$). Two different schedules
for this situation are shown in the figure. Schedule $S_0$ assigns tasks
$t_0$ and $t_1$ to worker $w_0$, task
$t_2$ to worker $w_1$ and task $t_3$ to worker
$w_2$. The timeline of the schedule the shows execution of tasks (gray rectangles)
and receiving of a data object from another worker (green rectangles) for each individual worker.
It is clear that with the schedule $S_1$, the task graph will be computed quicker
than with schedule $S_0$, even though the only difference between the two
schedules is the tasks $t_1$ and $t_3$ were swapped between
workers $w_0$ and $w_2$. Note that the schedule timeline assumes
that a worker can overlap the computation of a task with sending/receiving a data object, which is
a reasonable feature that is commonly provided by existing task runtimes.

\begin{figure}[h]
	\centering
	\resizebox{!}{110mm}{
	\begin{tikzpicture}
            \tikzmath{
                \tzerowidth = 15mm;
                \tonewidth = 20mm;
                \ttwowidth = 25mm;
                \tthreewidth = 35mm;
                \ozerowidth = 15mm;
                \oonewidth = 30mm;
            }
            \tikzset {
                taskstyle/.style={fill=gray, text=white, draw=none},
                objstyle/.style={fill=black!60!green, text=white, draw=none},
            }

            % T0
            \node[task, taskstyle, minimum size=7.5mm] (t0) at (0, 0.5) {$t_0$};
            \node[data, objstyle, minimum size=5mm] (d0a) at (-1, -1) {$d_0$};
            \node[data, objstyle, minimum size=10mm] (d0b) at (1, -1) {$d_1$};
            \draw [arrow] (t0) edge (d0a.north) (t0) edge (d0b.north);

            % T1 and T2
            \node[task, taskstyle, minimum size=10mm] (t1) at (-2, -2.5) {$t_1$};
            \node[task, taskstyle, minimum size=12.5mm] (t2) at (0, -2.5) {$t_2$};
            \draw [arrow] (d0a) edge (t1.north) (d0a) edge (t2.north);

            % T3
            \node[task, taskstyle, minimum size=17.5mm] (t3) at (2, -2.5) {$t_3$};
            \draw [arrow] (d0b.south) edge (t3.north);

            % Move below to draw the timelines
            \tikzset{shift={(-4,-4)}}

            \node[anchor=west] at (0, 0) {Schedule $S_0$: $w_0$=\{$t_0$, $t_1$\}, $w_1$=\{$t_2$\}, $w_2$=\{$t_3$\}};

            \tikzset{shift={(0,-0.75)}}
            \node (tim1A) at (0, 0) {$w_0$};
            \draw[arrow] (tim1A.east) -- ++(9, 0);
            \node[below = 0.5 of tim1A.south] (tim1B) {$w_1$};
            \draw[arrow] (tim1B.east) -- ++(9, 0);
            \node[below = 0.5 of tim1B.south] (tim1C) {$w_2$};
            \draw[arrow] (tim1C.east) -- ++(9, 0);

            % Timeline 1, row 1
            \node[taskstyle, minimum width=\tzerowidth, right = 0.2 of tim1A.east] (tim1t0) {$t_0$};
            \node[taskstyle, minimum width=\tonewidth, right = 0.1 of tim1t0.east] (tim1t1) {$t_1$};

            % Timeline 1, row 2
            \node[objstyle, minimum width=\ozerowidth, below = 1 of tim1t1.west, anchor=west]
            (tim1o0) {$d_0$ ($w_0$)};
            \node[taskstyle, minimum width=\ttwowidth, right = 0.1 of tim1o0.east] (tim1t2) {$t_2$};

            % Timeline 1, row 3
            \node[objstyle, minimum width=\oonewidth, below = 1 of tim1o0.west, anchor=west]
            (tim1o1) {$d_1$ ($w_0$)};
            \node[taskstyle, minimum width=\tthreewidth, right = 0.1 of tim1o1.east]
            (tim1t3) {$t_3$};

            \draw[dashed, draw=red] (tim1t0.west) -- ++(0, -2.75) --
            ([shift=({0,-0.75})]tim1t3.east) -- (tim1t3.east);

            \node[text=red] at (4.5, -3) {Makespan};

            % Move below to draw the timelines
            \tikzset{shift={(0,-4)}}

            \node[anchor=west] at (0, 0) {Schedule $S_1$: $w_0$=\{$t_0$, $t_3$\}, $w_1$=\{$t_2$\}, $w_2$=\{$t_1$\}};

            \tikzset{shift={(0,-0.75)}}
            \node (tim2A) at (0, 0) {$w_0$};
            \draw[arrow] (tim2A.east) -- ++(9, 0);
            \node[below = 0.5 of tim2A.south] (tim2B) {$w_1$};
            \draw[arrow] (tim2B.east) -- ++(9, 0);
            \node[below = 0.5 of tim2B.south] (tim2C) {$w_2$};
            \draw[arrow] (tim2C.east) -- ++(9, 0);

            % Timeline 2, row 1
            \node[taskstyle, minimum width=\tzerowidth, right = 0.2 of tim2A.east] (tim2t0) {$t_0$};
            \node[taskstyle, minimum width=\tthreewidth, right = 0.1 of tim2t0.east] (tim2t1)
            {$t_3$};

            % Timeline 2, row 2
            \node[objstyle, minimum width=\ozerowidth, below = 1 of tim2t1.west, anchor=west]
            (tim2o0) {$d_0$ ($w_0$)};
            \node[taskstyle, minimum width=\ttwowidth, right = 0.1 of tim2o0.east] (tim2t2) {$t_2$};

            % Timeline 2, row 3
            \node[objstyle, minimum width=\ozerowidth, below = 1 of tim2o0.west, anchor=west]
            (tim2o1) {$d_0$ ($w_0$)};
            \node[taskstyle, minimum width=\tonewidth, right = 0.1 of tim2o1.east] (tim2t3) {$t_1$};

            \draw[dashed, draw=red] (tim2t0.west) -- ++(0, -2.75) --
            ([shift=({0,-1.75})]tim2t2.east) -- (tim2t2.east);
        \end{tikzpicture}
	}
	\caption{Simple task graph and two different schedules}
	\label{fig:scheduling-example}
\end{figure}

Optimal scheduling of tasks to workers is NP-hard~\cite{Ullman1975}, even in the most basic
scenarios, when the exact duration of executing each task is known, and even if we do not consider
network costs of transferring data between workers. Task runtimes thus resort to various heuristics
tailored to their users' needs. Some classic task scheduling heuristics and their comparisons can
be found in~\cite{hlfet1974,kwok1998benchmarking,hagras2003static,wang2018list,estee}.

The scheduling heuristics have to take many factors into consideration when deciding on which
worker should a task execute:

\begin{description}
	\item[\textbf{Resource requirements}] The scheduler should respect all resource requirements specified by individual tasks. The runtime
		thus has to observe the (dynamically changing) available resources of each worker and schedule
		tasks accordingly to uphold their requirements. This can be challenging especially in the presence
		of complex resource requirements.
	\item[\textbf{Data transfer cost}] If the runtime operates within a distributed cluster, one of the most important scheduling aspects
		that it needs to consider is the transfer cost of data between workers over the network. All
		benefits gained by computing a task on another worker to achieve more parallelization might be lost
		if it takes too much time to send the data (task outputs) to that worker.

		The scheduler thus has to carefully balance the communication-to-computation ratio, based on the
		available network bandwidth, sizes of outputs produced by tasks and the current utilization of
		workers.
	\item[\textbf{Scheduling overhead}] The overhead of computing the scheduling decisions itself also cannot be underestimated. As already
		stated, computing an optimal solution is infeasible, but even heuristical approaches can have
		wildly different performance characteristics. Producing a lower quality schedule sooner, rather
		than a higher quality schedule later, can be sometimes beneficial, as we have demonstrated
		in~\cite{estee, rsds}.
	\item[\textbf{Memory consumption}] It is desirable to execute as many tasks in parallel on a given worker (with respect to its
		available parallelism), to speed up the completion of the whole workflow. However, the scheduler
		must also balance the amount of executing tasks according to the total memory that they consume. In
		general, it is quite hard to predict for how long will a task execute, and how much (peak) memory
		will it consume. When a task executes longer than expected, the workflow will still be computed, it
		will just take more time. But when a task uses more memory than expected, the scheduler puts many
		such tasks on a worker, and the worker then runs out of memory, the worker will probably crash. If
		this happens repeatedly, it can stall or completely stop the execution of the workflow. For these
		kinds of memory-intensive workflows, the scheduler should assign fewer tasks at the same time to a
		single worker, or reduce memory consumption in some other way, e.g.\ by keeping less cached data
		objects in worker's memory.
\end{description}
