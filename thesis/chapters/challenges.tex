Task-based programming is quite popular and useful, as it provides a simple way to define complex
workflows which can be then executed in a wide range of environments, ranging from consumer
laptops, through cloud deployments, to distributed HPC clusters. However, as any programming model,
it also has some disadvantages and problems. It is crucial to understand these limitations in order
to design approaches for overcoming them.

This chapter thus describes various challenges and requirements of this programing model, both in
terms of efficient and scalable execution, and in terms of the offered productivity and ergonomics.
It focuses specifically on challenges and requirements required by HPC use-cases, which introduce a
unique set of constraints stemming both from the inherent complexity of HPC software and hardware
and also from the sheer computational scale required to efficiently utilize HPC resources.

In addition to the challenges, we will also mention various desired properties and features which
should be offered by task runtimes in order to either support the mentioned requirements or to
alleviate the mentioned problems.

\section{Allocation manager}
The vast majority of HPC systems use some kind of allocation manager (such as
PBS/Torque~\cite{pbs} or Slurm~\cite{slurm}) to facilitate submission
of computations, resource management and user accounting~\cite{slurm-schedmd}. To perform any
computation through an allocation manager, the user has to submit an \emph{allocation}
\footnote{The term \emph{job} is also commonly used for the concept of HPC computational
requests. However, we will reserve this term for a different concept described later in the thesis and use the term
\emph{allocation} for HPC computational requests.}, a computational request that states how many nodes they want to
allocate and what is the expected runtime of their computation. The allocation is then submitted
into a \emph{queue} and starts to execute only once there are enough free
computational resources. Allocation managers are used to provide fair access to HPC resources, to
avoid their oversubscription and also to handle accounting of computation. They tend to have fairly
strict limits on the number of allocations that users can submit and the amount of nodes that they
can have reserved for their allocations at any given time.

Since users have to create computational requests through the allocation manager, and thus they
cannot simply execute their task graphs directly on an HPC cluster, a natural question arises --
how to map tasks (or task graphs) to HPC allocations in order to efficiently utilize HPC resources?
Several ways of performing this mapping are described below, however all of them come with
significant disadvantages.

\subsection*{Execute the whole task graph in a single allocation} This is the simplest case. If the
task graph does not have a large number of tasks, or if it can be executed quickly, it could be
submitted within a single allocation. This approach is quite simple for the user, since they just
execute the task graph using some task runtime in the same way as they would on a cluster without
an allocation manager or on a personal computer. The only difference is that they have to define
and submit an allocation that will bootstrap the computation.

However, since allocations are bound both by node count and time limits, this approach will only be
usable for rather small task graphs. Indeed, if the computation is short, it might not even make
sense to use an HPC cluster to compute it. A more realistic scenario is that even if an individual
task graph can be executed quickly, users might want to execute many such task graphs (for example
to execute many experiments with different parametrizations). This situation can be seen as a
special case of a large task graph that consists of many disjoint components (smaller task
subgraphs).

\subsection*{Execute each task as an individual allocation} From a certain point of view, HPC
allocation managers can also be viewed as task runtimes that operate on a very coarse level --
their tasks being allocations that potentially span hundreds of nodes, run for days or even longer
and consist of many different program executions. Ideally, there would be no difference between an
allocation manager and a task runtime, and users would just be able to construct an arbitrarily
granular task graph and execute it directly on an HPC cluster in a straightforward way.

While this approach can certainly look tempting, in practice it is not always feasible to use the
currently popular allocation managers (Slurm and PBS) in this way, because they operate on a level
that is far too coarse for complex fine-grained task graphs. They tend to have large overhead per
each allocation, which can be several orders of magnitude larger than for typical task runtimes
(e.g.\ seconds vs milliseconds). Furthermore, they seldom allow the user to create more than a few
hundreds of allocations at the same time, both to provide fairness and also because they simply
cannot scale to such an amount of allocations. And their support for expressing dependencies
between the individual tasks (allocations) is also quite basic.

It should be noted that even though there is definitely room for improving the scheduling
performance of HPC allocation managers, some of their complexity and performance limitations are
inherent. They have to provide accurate accounting, handle robust and secure cleanup of
allocations, take care of user and process isolation, ensure user fairness and many other things.
Many of these responsibilities are out of scope for task runtimes, which enables them to achieve
higher performance.

Another problem is node granularity. For ``small'' tasks that only use e.g.\ a few cores, users
would like to schedule and execute multiple tasks on a single node at the same time, to leverage
the available hardware resources efficiently. While allocation managers are able to create
allocations that require only a fraction of a node, this functionality tends to be sometimes
disabled. Either for security reasons, because tasks from multiple users can then execute on the
same node in parallel, and thus user isolation is reduced, or for performance reasons, because the
overhead of scheduling a large number of allocations (in theory many allocations per each node) can
become unmanageable for the allocation manager~\cite{it4i_node_scheduling_policy}. When the manager is set
up in a way that a single allocation has to span at least a (complete) single node, it can lead to
wasted resources if a single task cannot leverage the whole computational node.

Last, but not least, another reason why users might not want to use the allocation manager directly
as a task runtime is that it is useful to debug and prototype task graphs in a small-scale scenario
(e.g.\ locally, on a personal computer), before executing it on a large-scale HPC system. However,
it can be quite challenging for users to deploy systems like PBS or Slurm locally. Therefore, they
would need to use a different task runtime when executing the task graph locally and on the target
HPC platform, which is not very practical.

The mentioned issues create a certain dichotomy between the coarse-grained focused allocation
manager and more fine-grained focused task runtimes, and instead of facilitating simple workflow
execution on HPC clusters, it can create a barrier for users. Users that want to execute a task
graph on an HPC system thus usually use a separate task runtime (e.g.\
Dask~\cite{dask}) rather than using the allocation manager directly. Instead of
building a task graph of the whole computation and executing it with a single command, they have to
think about will they reconcile the coarse-grained nature of allocations with the fine-grained
nature of tasks. In practice, this means that they have to partition the tasks of their workflows
into allocations. As we have discussed above, if the whole task graph can be executed within a
single allocation, this is typically not an issue. However, when users need to create multiple
allocations, this process can be quite cumbersome.

\subsection*{Split the task graph into a smaller amount of allocations}
This is the ultimate approach that task graph users will probably sooner or later converge to, once
their task graph becomes sufficiently complex. When the task graph does not fit within a single
allocation, and its tasks are too fine-grained for the overhead caused by the allocation manager,
the graph has to be partitioned into smaller parts which will then be executed in individual
allocations by independent instances of some task runtime.

This process is not straightforward, especially if users have to perform the partitioning manually.
Graph partitioning itself is a notoriously difficult problem that is
NP-hard~\cite{graph_partitioning}\todo{Ada: Is this OK?}, and it is thus difficult to decide
beforehand how exactly should the task graph be split into allocations. Furthermore, if the
partitioning of tasks into allocations is performed only once, before the computation begins, then
it might lead to suboptimal hardware utilization, as tasks will not be load balanced across
allocations, even if multiple allocations run concurrently.

In addition to partitioning the task graph, further code infrastructure might have to be
implemented, outside the boundaries of the task-based programming model. As an example, the
intermediate results of computed tasks of a partitioned subgraph might have to be stored (to some
storage system) before the corresponding allocation ends, and the results from multiple allocations
then have to be merged together. To robustly handle tasks that fail or to support adding new tasks
while a task graph is already executing, additional code might be needed to periodically submit new
allocations for tasks that have not been successfully finished yet, until the whole task graph is
computed. This reduces the ergonomics of using task graphs, because it basically forces the user to
reimplement part of the task runtime behavior on top of the allocation manager, to overcome its
limitations.

\vspace{5mm}
The gap between allocation managers and task runtimes creates a certain disconnect for users
attempting to scale their task graph computation. Running on a personal computer tends to be quite
simple. After that, moving to an HPC cluster and executing the entire task graph inside a single
allocation is also quite straightforward. But once the task graph has to be partitioned into
multiple allocations, the simple abstraction of implicitly parallel task graphs that can be
executed with a single command quickly falls apart, as the user has to perform a lot of additional
work to make this scenario execute efficiently.

Ideally, users would not have to think about the allocation manager; they should be able to
construct a task graph and execute it directly on an HPC cluster in a straightforward way, by
letting some tool perform the partitioning and load balancing across allocations automatically for
them. This could be achieved either by adding support for executing fine-grained task graphs to
allocation managers or by integrating task runtimes with allocation managers to provide transparent
execution of task graphs on HPC systems.

\section{Cluster heterogeneity}
Even though task graphs are designed to be portable and ideally should not depend on any specific
execution environment, for certain types of tasks, we need to be able to describe at least some
generic environment constraints. For example, when a task executes a program that leverages the
CUDA\todo{explain} framework, which is designed to be executed on a graphics
accelerator, it has to be executed on a node that has a GPU available, otherwise it will simply not
work.

It should thus be possible for an HPC task to define \emph{resource requirements}, which specify
resources that have to be provided by an environment that will execute such task. These
requirements can be quite diverse. For example, a requirement could be the amount of cores (some
tasks can use only a single core, some can be multithreaded), the amount of available main memory,
a minimum duration required to execute the task or (either optional or required) presence of an
accelerator like a GPU or an FPGA\@. In order to remain portable and independent of a specific
execution environment, these requirements should be abstract and describe general, rather than
specific, types of resources.

The challenge related to resource requirements of HPC tasks specifically is the diverse hardware
present in modern HPC clusters, which have started to become increasingly heterogeneous in recent
years. This trend can be clearly seen in the TOP500 list of most powerful
supercomputers~\cite{top500analysis}. Individual cluster nodes contain varying amounts and
types of cores and sockets, main memory, NUMA nodes or accelerators like GPUs or FPGAs. Since HPC
software tries to leverage all these modern HPC hardware features, this complexity is also
propagated to tasks and their resource requirements, which can become relatively complex.

Some types of tasks might require a combination of several requirements, for example two GPUs,
sixteen cores and 32 GiB of main memory. Some tasks are designed in a way that allows them to
leverage an open-ended range of resources, e.g.\ a task might require four cores, but if more are
available, it could use as many as possible. Furthermore, some tasks might even support several
variants of requirements, for example a task might either use four cores and a single GPU (if there
is one available), or it could use more cores (and no GPU) to offset the absence of an accelerator.

A resource requirement that is fairly specific to HPC systems is the requirement of using multiple
nodes per single task. This requirement is necessary for programs that are designed to be executed
in a distributed fashion, such as programs using MPI, which are quite common in HPC. This
requirement is not supported in many task runtimes, because their programming model assumes that a
task performs an atomic computation that executes on a single node. The use-case of tasks using
multiple nodes is discussed in more detail later in this chapter.

To support the mentioned scenarios, task runtimes should allow users to specify arbitrarily
fine-grained and abstract resource requirements for each task. They should also allow users to
attach resources that will satisfy these requirements to each individual instance of an execution
environment that will execute the tasks. Runtimes should also be able to take these requirements
into account when scheduling, both to make sure that the requirements are upheld, and also to
utilize the available hardware effectively.

\section{Data transfers}
After a task is computed, it can produce various data outputs, standard error or output streams,
files created on the disk or data objects that are then passed as inputs to dependent tasks. There
are many ways of storing and transferring these outputs. Some task frameworks store task outputs on
the filesystem, since it is relatively simple to implement, and it provides support for basic data
resiliency out-of-the-box.

HPC nodes might not contain any local disks, but instead use shared filesystems accessed over a
network. While this can be seen as an advantage, since with a shared filesystem it is much easier
to share task outputs amongst different workers, it can also be a severe bottleneck. Shared
networked filesystems can suffer from quite high latency, and accessing them can consume precious
network bandwidth that is also used e.g.\ for managing computation (sending commands to workers) or
for direct worker-to-worker data exchange. Furthermore, data produced in HPC computations can be
quite large, and thus storing it to a disk can be a bottleneck even without considering networked
filesystems.

These bottlenecks can be alleviated by transferring task outputs directly between workers over the
network (preferably without accessing the filesystem in the fast path), by streaming outputs
between tasks without the need to store them or by leveraging RAM disks~\cite{hyperloom}.
Making use of HPC specific technologies, such as MPI or InfiniBand, could be also worthwhile to
leverage the very fast interconnects available in HPC clusters.

Data outputs produced by tasks tend to be considered immutable in existing task runtimes, since a
single output can be used as an input to multiple tasks, and these might be executed on completely
different computational nodes. A problem that can arise with this approach is that if the data
outputs are large, but the computation within tasks that work with the data is short, the
serialization overhead (or even memory copy overhead, if the dependent task is executed on the same
node) can dominate the execution time. Such use-cases can be solved with stateful data management,
for example in the form of \emph{actors}, which can be considered stateful tasks that
operate on a single copy of some large piece of data.

\section{Fault tolerance}
Fault tolerance is relevant in all distributed computing environments, but HPC systems have
specific requirements in this regard. As was already mentioned, computational resources on HPC
clusters are provided through allocation managers. Computing nodes allocated by these managers are
provided only for a limited duration, which means that for long-running task graphs, some nodes
will disconnect and new nodes will appear dynamically during the execution of the task graph.
Furthermore, since the allocations go through a queue, it can take some time before new
computational resources arrive, therefore the task graph can remain in a paused state, where no
tasks are being executed, for potentially long periods of time.

It is important for task runtimes to be prepared for these situations; they must handle node
disconnections gracefully, even if a task was being executed on a node that disconnects, and they
should be able to restart previously interrupted tasks on newly arrived workers. In general, in HPC
scenarios, worker instability and frequent disconnects should be considered the norm, not just a
rare edge case.

\section{Multi-node tasks}
Many existing HPC applications are designed to be executed on multiple (potentially hundreds or
even thousands) nodes in parallel, using e.g.\ MPI libraries or other communication frameworks.
Multi-node execution could be seen as a special resource requirement, which states that a task
should be executed on multiple workers at once.

Support for multi-node tasks affects many design areas of a task runtime:
\begin{description}
	\item[Scheduling] When a task requires multiple nodes for execution and not enough nodes are available at a given
		moment, the scheduler has to decide on a strategy that will allow the multi-node task to execute.
		If it was constantly trying to backfill available workers with single-node tasks, the multi-node
		tasks could be starved.

		The scheduler might thus have to resort to keep some nodes idle for a while to enable the
		multi-node task to start as soon as possible. Another approach could be to interrupt the currently
		executing tasks and checkpoint their state to make space for a multi-node task, and then resume
		their execution once the multi-node task finishes.

		In a way, this decision-making already has to be performed on the level of individual cores even
		for single-node tasks, but adding multiple nodes per task makes the problem much more difficult.
	\item[Data transfers] It is relatively straightforward to express data transfers between single-node tasks in a task
		graph, because they naturally correspond to dependencies (edges) between the tasks. With multi-node
		tasks, the data distribution patterns become more complex, for example data can be replicated from
		a single node to multiple nodes when a multi-node task starts or gathered (reduced) from multiple
		nodes to a single node when such task finishes.

		When several multi-node tasks depend on one another, the task runtime should be able to exchange
		data between them in an efficient manner. This might require some cooperation with the used
		communication framework (e.g.\ MPI) to avoid needless repeated serialization and deserialization.
	\item[Fault tolerance] When a node executing a single-node task crashes or disconnects from the runtime, its task can be
		rescheduled to a different worker. In the case of multi-node tasks, failure handling requires more
		communication and is generally more complex. When a task is executing on four nodes and one of them
		fails, the runtime has to make sure that the other nodes will be notified of this situation, so
		that they can react accordingly (either by finishing the task with a smaller amount of nodes or by
		also failing immediately).
\end{description}

To enable common HPC usecases, task runtimes should be able to provide some support for multi-node
tasks and allow them to be combined with single-node tasks. Advanced multi-node task support could
be provided e.g.\ by offering some kind of integration with MPI or similar common HPC technologies.

\section{Scalability}
The sheer amount of HPC performance (node count, core count, network interconnect bandwidth) opens
up opportunities for executing large scale task graphs, but that in turn presents unique challenges
for task runtimes. Below you can find several examples of bottlenecks that might not matter in a
small computational scale, but that can become problematic in the context of HPC-scale task graphs.

\begin{description}
	\item[Task graph materialization] Large computations might require building massive task graphs that contain millions of tasks. The
		task graphs are typically defined and built outside the task runtime itself, for example on the
		login nodes of computing clusters or on client devices (e.g.\ laptops), which can provide only
		relatively low performance. It can be quite slow to build, serialize and transfer such graphs over
		the network to the task runtime. This can create a bottleneck even before any task is executed.
		This has been identified as an issue in existing task runtimes~\cite{dask-client-perf}.

		In such case, it can be beneficial to provide an API for defining task graphs in a symbolic way,
		for example by representing a potentially large group of similar tasks by a single entity. Such
		symbolic graphs could then be sent to the runtime in a compressed form and re-materialized only at
		the last possible moment. In an extreme form, the runtime could operate on such graphs in a fully
		symbolic way, without ever materializing them.
	\item[Communication overhead] Scaling the number of tasks and workers will necessarily put a lot of pressure on the communication
		network, both in terms of bandwidth (sending large task outputs between nodes) and latency (sending
		small management messages between the scheduler and the workers). Using HPC technologies, such as
		MPI or a lower-level interface like RDMA (Remote Direct Memory Access), could provide a non-trivial
		performance boost in this regard.

		As we have demonstrated in~\cite{di_girolamo_pspin_2021, spin2}, in-network computing, an active area of
		research, can be also used to optimize various networking applications by offloading some
		computations to an accelerated NIC (network interface controller). This approach could also be
		leveraged for task runtimes, for example by reducing the latency of management messages between the
		scheduler and workers or by increasing the bandwidth of large data exchanges amongst workers, by
		moving these operations directly onto the network card.
	\item[Runtime overhead] As we have shown in~\cite{rsds}, task runtimes with a centralized scheduler have to
		make sure that their overhead remains manageable. Even with an overhead of just
		$1ms$ per task, executing a task graph with a million tasks would result in
		total accumulated overhead of twenty minutes! Our results indicate that increasing the performance
		of the central scheduling and management component of a task runtime can have a large positive
		effect on the overall time it takes to execute the whole task graph.

		However, the performance of the central server cannot be increased endlessly, and from some point,
		using a centralized architecture, which is common to task runtimes, itself becomes a bottleneck.
		Even if the workers exchange large output data directly between themselves, any single, centralized
		component may become overloaded simply by coordinating and scheduling the workers.

		In that case, a decentralized architecture could be leveraged to avoid the reliance on a central
		component. Such a decentralized architecture can be found e.g.\ in Ray~\cite{ray}.
		However, to realize the gains of a decentralized architecture, task submission itself has to be
		decentralized in some way, which might not be a natural fit for common task graph workflows. If all
		tasks are generated from a single component, the bottleneck will most likely remain even in an
		otherwise fully decentralized system.
\end{description}

\section{Iterative computation}
A natural way of executing task graphs is to describe the whole computation with a single task
graph, submit the graph to the task runtime and wait until all the tasks are completed. However,
there are some computations that need a more iterative approach. Training a machine learning model
can be stopped early if the loss is no longer decreasing. A chemical or physical simulation is only
considered completed once a desired accuracy has been reached, which might take a previously
unknown number of steps. These scenarios and many others like them, are quite common in HPC use
cases.

To support iterative computation, task runtimes should allow the user to stop the execution of a
task graph (or its subgraph) once a specific condition is met, and also to add new tasks to the
task graph in a dynamic fashion, if it is discovered that more iterations are needed.

\section{Summary}
Even though more HPC use-cases and oddities could always be found, it is already clear from the
mentioned challenges that HPC use-cases that leverage task graphs can contain a lot of complexity.
It could be possible to add support for some mentioned requirements to existing task runtimes,
which are described in the next section. However, the described challenges are so diverse and
complex that a dedicated approach which considers them holistically could provide a better solution
that would avoid both ergonomics and performance from being compromised.

The aforementioned requirements will serve as a basis for further research in the proposed thesis.
The goal of the thesis is to design approaches for executing task graphs on HPC systems that take
the aforementioned requirements into account. These approaches will leverage the
%\hyperqueue{} task runtime, which is described further in Chapter~\ref{ch:hyperqueue}.
