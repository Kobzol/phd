\workshare{I have collaborated on this work with Ada Böhm, we have both contributed to it equally. While we are the primary contributors to
\hyperqueue{}, it should be noted that multiple other people have contributed to it, as its development is a team effort. Source code contribution statistics for \hyperqueue{}
can be found on GitHub\footnoteurl{https://github.com/it4innovations/hyperqueue/graphs/contributors}.}

%\hyperqueue{} is an HPC-tailored task runtime designed for executing task graphs in HPC
%environments. Its two primary objectives are to be as performant as possible and to be easy to use
%and deploy. It is developed in the Rust programming language and available as an open-source
%software\footnoteurl{https://github.com/it4innovations/hyperqueue}.
%
%The key idea of \hyperqueue{} is to disentangle the submission of computation and the provision of
%computational resources. With traditional HPC job managers, the computation description is
%closely tied to the request of computational resources, which leads to problems mentioned in
%Section~\ref{sec:challenges}, such as less efficient load balancing or the need to manually
%aggregate tasks into jobs. \hyperqueue{} separates these two actions; users submit task graphs
%independently of providing computational resources (workers) and let the task runtime take care of
%matching them together, based on requested resource requirements and other constraints.
%
%One of the driving use-cases for \hyperqueue{} is efficient node usage and load balancing. The
%latest HPC clusters contain a large number (hundreds) of cores, yet it is quite challenging to
%design a single program that can scale effectively with so many threads. Thus, in order to fully
%utilize the whole computational node, multiple tasks that each leverage a smaller amount of
%threads have to be executed on the same node at once. \hyperqueue{} is able to effectively schedule
%tasks to utilize all available computational nodes, and thanks to its design, it is able to do this
%not just within a single HPC job, but across many jobs at once.
%
%\hyperqueue{} is being used by users of various HPC centres, and it is also a key
%component of the Horizon 2020 European Union projects
%LIGATE\footnoteurl{https://www.ligateproject.eu},
%EVEREST\footnoteurl{https://everest-h2020.eu} and
%ACROSS\footnoteurl{https://across-h2020.eu}. It is also envisioned as one of the primary ways of
%executing computations on the LUMI supercomputer~\cite{lumi_it4innovations_2022}.
%
