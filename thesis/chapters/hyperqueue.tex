The previous two chapters focused primarily on the performance aspects of task graph execution, by
examining various task scheduling algorithms and also bottlenecks that can limit the scalability of
existing task runtimes. Performance is naturally crucial for \gls{hpc}, yet it should
not be the only factor to focus on. \Autoref{ch:sota} has discussed several other challenges
that affect \gls{hpc} task graphs, that are related to the second main focus of this
thesis, namely the \emph{ergonomics} of task graph execution.

This is a very broad term that encompasses several areas, such as providing an easy way to define
the structure of the task graph, allowing its execution it in an effortless manner (regardless
whether it is on a local computer or a distributed cluster), handling fault tolerance without the
user's intervention, allowing users to express complex resource requirements and many others. This
thesis primarily focuses on analyzing and resolving ergonomical challenges that form a barrier for
executing task graphs on \gls{hpc} clusters.

%While existing task runtimes are able to deal with some of the mentioned challenges with varying
%degrees of success, a unified, dedicated and \gls{hpc}-tailored approach for executing
%task graphs could provide added value to workflow authors, both in terms of ergonomics and
%performance.

%Apart from efficiency, there are several other important challenges have been discussed in. These
%fall under the umbrella of \emph{ergonomics} of task graph execution, which is the second
%primary focus of this thesis. Ergonomics and developer experience in general sometimes tends to be
%neglected in \gls{hpc} applications and tools, which can be notoriously difficult to
%deploy, configure and use.

%While using off-the-shelf task runtimes with \gls{hpc} clusters is possible, these
%tools are not prepared to deal with the idiosyncracies and complexities of supercomputing systems.

%Executing task graphs on \gls{hpc} clusters does not \emph{need} to be
%difficult though, if we use specialized approaches, rather than off-the-shelf tools that are not
%prepared to deal with the complexities of \gls{hpc} systems.

%Furthermore, ergonomics and performance do not
%have to contradict each other -- as we will see in this chapter, approaches that increase
%ergonomics can also help with improving performance and hardware utilization.

One approach to improve ergonomics could be to make incremental improvements to the various
mentioned challenges in off-the-shelf task runtimes. However, as we will see later in this chapter,
ergonomical areas such as supporting heterogeneous tasks and clusters and resource requirements,
providing fault tolerance and dynamic load balancing and overcoming allocation manager limits are
all intertwined, and they affect each other in non-trivial ways. Therefore, this chapter introduces
a holistic meta-scheduling design that aims to tackle the most important challenges with a unified
solution, in order to enable effortless, efficient and fault-tolerant execution of task graphs on
heterogeneous \gls{hpc} clusters in the presence of allocation managers.

This reference design has been implemented in \hyperqueue{}, an
\gls{hpc}-tailored task runtime that was designed to enable transparent, ergonomic and
efficient execution of task graphs on \gls{hpc} systems. \hyperqueue{} is a
spiritual successor of \rsds{}, and builds upon its server and task scheduler
implementation, however it does not use the \dask{} interface. It is the culmination
of the research and work presented in this thesis, which was made possible thanks to the experience
gained from \estee{} and \rsds{}, and also from interacting with many
\gls{hpc} workflows and use-cases over the course of several years.

% TODO: contributions?
%\hyperqueue{} contains the following contributions:
%\begin{enumerate}
%    \item
%\end{enumerate}

%The description of \hyperqueue{}'s implementation and other related concepts in this chapter were
%adapted from our publication~\cite{TODO}.

\workshare{I have collaborated on this work with Ada Böhm, we have both contributed to it equally. I am the sole author of the design and implementation of
specific components of \hyperqueue{}, these will be marked as such in the rest of this chapter. While me and Ada are the primary
contributors to \hyperqueue{}, it should be noted that multiple other people have contributed to it, as its development is a team effort. Source code contribution statistics for \hyperqueue{}
can be found on GitHub\footnoteurl{https://github.com/it4innovations/hyperqueue/graphs/contributors}.}

%First, we will go through the meta-scheduling design, which will be defined in terms of its
%\hyperqueue{} implementation. Then we will examine the implementation details of
%\hyperqueue{}, analyze its performance and evaluate it using several use-cases. Lastly,
%\hyperqueue{} will be compared with other state-of-the-art meta-scheduling task runtimes.

% TODO: rename/restructure
\section{Interaction with allocation managers}
The primary challenge that affects the simplicity of task graph execution on \gls{hpc}
clusters is the presence of an allocation manager, as it forms a barrier; on such clusters, it is
not possible to compute anything at all without interacting with allocations. In order for task
graph execution to become truly ergonomic, there is no point in tackling the rest of the
(ergonomical) challenges until there is a straightforward way of executing task graphs through an
allocation manager.

\Autoref{ch:sota} has described several approaches that can be used to map tasks to
allocations, which is necessary to execute task graphs, and also their various shortcomings. Due to
the limits imposed by allocation managers, it might be necessary to partition task graphs into
multiple subgraphs in order to fit them within an allocation, which can be challenging.
Furthermore, it can result in non-optimal usage of hardware resources, because tasks from
(sub)graphs submitted in separate allocations will only be load balanced within their own
allocation, and not across different allocations.

It is important to understand what is uniquely challenging about the interaction with allocations.
The fact that the computation needs to go through a queue is not an issue on itself, as we are
dealing with batch processing anyway. Therefore, some form of a delay between submitting the task
graph and receiving the results is expected, even if we did not have to submit allocations at all.

The main issue of allocations is that they strictly tie together two separate aspects.
\emph{What} does the user want to compute (the script that will be computed once the
allocation starts) and \emph{where} should the computation take place (specific hardware
resources and computational nodes reserved for the allocation). As was already described
previously, both of these things have to be specified together in an allocation. This is very a
inflexible design, for several reasons:

\begin{itemize}
	\item The user needs to consider both aspects at the same time. Ultimately, the main thing that the user
	      cares about is what do they want to compute. With allocations, they also need to think about
	      specific hardware resources that should be allocated for computing their task graph.
	\item Hardware resources are allocated for the whole duration of the allocation. This can result in
	      inefficient resource usage, especially for heterogeneous task graphs that consist of many types of
	      tasks that use different resources. In situations where not all resources can be used at the same
	      time (which is common in task graphs with multiple computational stages), some of the resources can
	      unnecessarily sit idle.
	\item The amount of used resources has to be decided up front, new hardware resources cannot be added nor
	      removed from existing allocations. This can possibly cause inefficiencies, as described above, but
	      it also limits load balancing. If balancing only happens within (and not across) allocations, users
	      cannot easily provide new hardware resources during the computation, if they realize that it is
	      progressing too slowly.
	\item Granularity of the allocated resources might not match the granularity of tasks. Unless the
	      allocation manager supports very fine-grained allocations (e.g.\ on the level of individual
	      \gls{cpu} cores, rather than whole computational nodes), there can be a large gap
	      between the resources required by a task graph and the resources provided to an individual
	      allocation. This can again lead to resources sitting idle.
	\item Binding computation with specific hardware resources up front complicates fault tolerance. When the
	      user submits thousands of tasks on a specific set of hardware resources, and some of these tasks
	      fail, the user might need to create a new allocation to recompute the failed tasks. This requires
	      figuring out again a new set of hardware resources that should be requested, as the original
	      resources will probably not be a good fit for just a subset of the original computation. A
	      fault-tolerant design would ideally allow recomputing tasks on any compatible hardware resource,
	      which goes against forcing computations and resources to be defined together.
\end{itemize}

It is interesting to note that some of these challenges are uniquely limiting task graphs, and in
general programming models that are different from the traditional ways of defining
\gls{hpc} computations. Consider a distributed application implemented using
\gls{mpi}, which has historically been the most common way of executing computations
on supercomputers. \gls{mpi} applications typically assume that they will run on a set
of computational nodes for a relatively long duration (hours or even days). This set of nodes
(\gls{mpi} processes) usually does not change during the computation; it is not
trivial to add new processes, and if some of the processes crash, it typically leads to the whole
computation being aborted, as fault tolerance is not a default property of \gls{mpi}
applications~\cite{fault_tolerant_mpi}.

These properties of \gls{mpi} applications are similar to the mentioned properties of
allocations, therefore they fit together quite well, and using allocations to execute them is
relatively straightforward; it is clear that the allocation model itself was designed with
\gls{mpi}-like use-cases in mind. At the same time, it is also not very surprising
that the allocation model is not a good match for programming models that are very different from
\gls{mpi}, such as task-based workflows.

Note that several of the problems associated with using allocations are directly relevant to
various challenges from~\Autoref{ch:sota}. Factors such as heterogeneous resource
requirements, fault tolerance or efficient hardware usage are affected by the usage of allocations.
Because of that, resolving the problems with allocations should also go a long way towards
alleviating other ergonomical challenges of task graph execution.

\section{Meta-scheduling design}
This section describes a design for executing task graphs that aims to alleviate the mentioned
challenges by treating task-based programming models as first-class citizens, but still remaining
compatible with allocation managers to enable straightforward execution on \gls{hpc}
clusters. First, we will see a general overview of the design, then we will discuss its properties,
and the following sections will describe its concrete implementation within the
\hyperqueue{} task runtime.

We have seen that the most problematic aspect of allocations is the requirement to define
computations and hardware resources together. The key idea of the proposed design is to remove this
limitation by completely disentangling the definition of what does the user want to compute (tasks)
from the hardware resources where the computation should take place (computational nodes,
\gls{cpu} cores, etc.). By separating these two concerns, we enable users to focus on
what they care about the most (their task graphs) while also automating the aspect that should be
handled by the task runtime (providing and scheduling computational resources).

How can we disentangle these two concepts when allocations require us to specify both? The
``trick'' lies in leveraging meta-scheduling on top of allocation managers, which allows us to
change what kind of computation is submitted in allocations. The proposed approach can be
summarized with the following high-level properties:

\begin{itemize}
	\item The task graph is defined independently of allocations. This is enabled by submitting
	      tasks into a single instance of a task runtime that is able to use computational resources across
	      different allocations. This is where the term \emph{meta-scheduling} comes in; the core of the idea
	      is to use a task runtime as a sort of high-level scheduler on top of an allocation manager. Instead
	      of forcing users to think in terms of allocations, they are able to submit task graphs in the same
	      way they would do it on a system that is not supervised by allocation managers.
	\item The approach above resolves the issues with defining the task graph, since users no
	longer need to manually split it into subgraphs or manually map tasks to allocations in
	order to work around the limitations of allocation managers. However, it is still needed to
	submit \emph{some} allocations, otherwise it would not be possible to compute anything on an
	\gls{hpc} cluster at all. Instead of defining specific tasks that should be computed within
	an allocation, the allocation will execute a generic computational provider, which will be
	able to execute any task that is scheduled to it by the task runtime. This approach removes
	the need to decide which tasks should be computed on which hardware resources and in which
	allocations up front, and thus it improves the ability of the task runtime to load balance
	tasks and improve hardware utilization. It also simplifies the submitted allocations,
	because all of them will have the same structure; they will simply start the computational
	provider, regardless of how is the user's task graph structured.

	These generic allocations can be submitted either by users or even fully automatically by
	the meta-scheduling task runtime. Note that even if the user submits these allocations
	manually, it becomes much simpler for them than with the previously mentioned approaches.
	They would simply ask for a set of computational providers to be started, without
	having to think about mapping tasks to allocations.
	\item The first two properties have
	- resource requirements
\end{itemize}

- allocation managers
- fault-tolerance
- heterogeneous task
- efficient hardware utilization
- load balancing across allocations

\section{\hyperqueue{}}

\hyperqueue{} takes a holistic approach to facilitating task graph scheduling on
\gls{hpc} clusters.

- moves the responsibility of performing the task to allocation mapping from the user to
\hyperqueue{},
which improves ergonomics.

\label{sec:hq-design}
This section describes the high-level design of \hyperqueue{}, its most important features
and also how it tackles the challenges mentioned in~\Autoref{ch:sota}. First, we will take a
look at how \hyperqueue{} interacts with \gls{hpc} allocation managers,
because that is the most important aspect of its design. Note that for simplicity, Slurm will be
used as an example of an allocation manager throughout this whole chapter, but it could be replaced
by PBS or any other commonly used allocation manager without the loss of generality.

Dask, Merlin

As you may recall from~\Autoref{sec:allocation-manager}, perhaps the most pronounced ergonomic challenge of
executing a task graph on an \gls{hpc} cluster is the presence of an allocation
manager. Due to the limits imposed by it, it might be necessary to partition task graphs into
multiple subgraphs in order to be fit them within an allocation, which can be challenging.
Furthermore, it can result in non-optimal usage of hardware resources, because tasks from
(sub)graphs submitted in separate allocations will only be load balanced within their own
allocation, and not across different allocations.

The primary issue of this mechanism is that it strictly ties together two separate aspects --
\emph{what} does the user want to compute (the script that will be computed in the
allocation) and \emph{where} should the computation take place (specific hardware
resources and computational nodes). As was described in the process above, both of these things
have to be specified together in an allocation.

%TODO: https://merlin.readthedocs.io/en/latest/tutorial/1_introduction/?h=many#how-can-merlin-run-so-many-simulations
%TODO: Balsam https://arxiv.org/abs/1909.08704
%TODO: FireWorks https://onlinelibrary.wiley.com/doi/10.1002/cpe.3505
%TODO: Parsl

This approach works well when the granularity of the computation very closely matches the
granularity of the requested hardware resources, and when the computation can make use of all the
available hardware resources for the whole duration of the allocation. For example, distributed
applications implemented using \gls{mpi} typically expect that they will be executed
on a fixed amount of nodes, and that they will (ideally) use all their resources for the whole
computation. They can also run for a potentially long time, e.g.\ hours, days or even more. Due to
these properties, these applications fit the allocation model quite well.

However, the situation is quite different for applications implemented with a task-based
programming model. Task graphs do not assume a fixed amount of hardware resources. Even though
individual tasks can specify their resource requirements, the specific set of resources on which
the tasks execute can change during the computation of the task graph. Task graphs are thus
designed to scale both up and down, in terms of hardware resource usage. However, this is not
easily achieved when we compute a task graph inside a single allocation, because new resources
cannot be dynamically added or removed from an allocation.

The absence of scaling down can also lead to a waste of resources. For example, assume that we will
execute a task graph inside an allocation that contains two nodes. The tasks will be load-balanced
amongst these two nodes, but towards the end of the computation, there might be tasks that will
take a long time to finish. This might lead to a situation where only one of the nodes will be
computing the long tail of the remaining tasks, while the second node will be idle. Due to the fact
that the allocation cannot release the second node while the computation is still ongoing, the user
will unnecessarily pay the cost of both nodes until the task graph finishes computing, even though
the second node might be idle.

- an even bigger problem if you combine resources, e.g.\ CPU + GPU tasks





\hyperqueue{} takes a holistic approach to facilitating task graph scheduling on
\gls{hpc} clusters.



The key idea of \hyperqueue{} is to disentangle the definition of the computation (tasks)
and the hardware resources (computational nodes) where the computation will take place.

- moves the responsibility of performing the task to allocation mapping from the user to
\hyperqueue{},
which improves ergonomics.


Features: - deployment - load balancing - resources - auto allocation - local experimentation -
efficiency

%\hyperqueue{} is an \gls{hpc}-tailored task runtime designed for executing task graphs in \gls{hpc}
%environments. Its two primary objectives are to be as performant as possible and to be easy to use
%and deploy. It is developed in the Rust programming language and available as an open-source
%software\footnoteurl{https://github.com/it4innovations/hyperqueue}.
%
%The key idea of \hyperqueue{} is to disentangle the submission of computation and the provision of
%computational resources. With traditional \gls{hpc} job managers, the computation description is
%closely tied to the request of computational resources, which leads to problems mentioned in
%\Autoref{sec:challenges}, such as less efficient load balancing or the need to manually
%aggregate tasks into jobs. \hyperqueue{} separates these two actions; users submit task graphs
%independently of providing computational resources (workers) and let the task runtime take care of
%matching them together, based on requested resource requirements and other constraints.
%
%One of the driving use-cases for \hyperqueue{} is efficient node usage and load balancing. The
%latest \gls{hpc} clusters contain a large number (hundreds) of cores, yet it is quite challenging to
%design a single program that can scale effectively with so many threads. Thus, in order to fully
%utilize the whole computational node, multiple tasks that each leverage a smaller number of
%threads have to be executed on the same node at once. \hyperqueue{} is able to effectively schedule
%tasks to utilize all available computational nodes, and thanks to its design, it is able to do this
%not just within a single \gls{hpc} job, but across many jobs at once.
%
%\hyperqueue{} is being used by users of various \gls{hpc} centres, and it is also a key
%component of the Horizon 2020 European Union projects
%LIGATE\footnoteurl{https://www.ligateproject.eu},
%EVEREST\footnoteurl{https://everest-h2020.eu} and
%ACROSS\footnoteurl{https://across-h2020.eu}. It is also envisioned as one of the primary ways of
%executing computations on the LUMI supercomputer~\cite{lumi_it4innovations_2022}.
%

Modern \gls{hpc} clusters contain a large number of heterogeneous resources that
provide vast amounts of computational power. It is challenging to design monolithic programs that
can leverage that performance potential effectively (e.g., by scaling to hundreds of cores);
\gls{hpc} users often design their computational workflows as a set of smaller,
interdependent tasks that use only a fraction of the resources of a single cluster node. Yet
executing these workflows on \gls{hpc} clusters in the presence of job managers such
as \gls{pbs} or Slurm can be challenging. They can impose limits on the concurrent
execution of multiple tasks on a single node, thus hampering node utilization, and their design in
general is not accustomed to an enormous number of smaller, less resource-intensive tasks, which
can lead to the manager being overloaded.
