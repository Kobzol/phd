The previous two chapters focused primarily on the performance aspects of task graph execution, by
examining various task scheduling algorithms and also bottlenecks that can limit the scalability of
existing task runtimes. Performance is naturally crucial for \gls{hpc}, yet it should
not be the only factor to focus on. \Autoref{ch:sota} has discussed several other challenges
that affect \gls{hpc} task graphs, that are related to the second main focus of this
thesis, namely the \emph{ergonomics} of task graph execution.

This is a very broad term that encompasses several areas, such as providing an easy way to define
the structure of the task graph, allowing its execution it in an effortless manner (regardless
whether it is on a local computer or a distributed cluster), handling fault tolerance without the
user's intervention, allowing users to express complex resource requirements and many others. This
thesis primarily focuses on analyzing and resolving ergonomical challenges that form a barrier for
executing task graphs on \gls{hpc} clusters.

%While existing task runtimes are able to deal with some of the mentioned challenges with varying
%degrees of success, a unified, dedicated and \gls{hpc}-tailored approach for executing
%task graphs could provide added value to workflow authors, both in terms of ergonomics and
%performance.

%Apart from efficiency, there are several other important challenges have been discussed in. These
%fall under the umbrella of \emph{ergonomics} of task graph execution, which is the second
%primary focus of this thesis. Ergonomics and developer experience in general sometimes tends to be
%neglected in \gls{hpc} applications and tools, which can be notoriously difficult to
%deploy, configure and use.

%While using off-the-shelf task runtimes with \gls{hpc} clusters is possible, these
%tools are not prepared to deal with the idiosyncracies and complexities of supercomputing systems.

%Executing task graphs on \gls{hpc} clusters does not \emph{need} to be
%difficult though, if we use specialized approaches, rather than off-the-shelf tools that are not
%prepared to deal with the complexities of \gls{hpc} systems.

%Furthermore, ergonomics and performance do not
%have to contradict each other -- as we will see in this chapter, approaches that increase
%ergonomics can also help with improving performance and hardware utilization.

One approach to improve ergonomics could be to make incremental improvements to the various
mentioned challenges in off-the-shelf task runtimes. However, as we will see later in this chapter,
ergonomical areas such as supporting heterogeneous tasks and clusters and resource requirements,
providing fault tolerance and dynamic load balancing and overcoming allocation manager limits are
all intertwined, and they affect each other in non-trivial ways. Therefore, this chapter introduces
a holistic meta-scheduling design that aims to tackle the most important challenges with a unified
solution, in order to enable effortless, efficient and fault-tolerant execution of task graphs on
heterogeneous \gls{hpc} clusters in the presence of allocation managers.

This reference design has been implemented in \hyperqueue{}, an
\gls{hpc}-tailored task runtime that was designed to enable transparent, ergonomic and
efficient execution of task graphs on \gls{hpc} systems. \hyperqueue{} is a
spiritual successor of \rsds{}, and builds upon its server and task scheduler
implementation, however it does not use the \dask{} interface. It is the culmination
of the research and work presented in this thesis, which was made possible thanks to the experience
gained from \estee{} and \rsds{}, and also from interacting with many
\gls{hpc} workflows and use-cases over the course of several years.

% TODO: contributions?
%\hyperqueue{} contains the following contributions:
%\begin{enumerate}
%    \item
%\end{enumerate}

%The description of \hyperqueue{}'s implementation and other related concepts in this chapter were
%adapted from our publication~\cite{TODO}.

\workshare{I have collaborated on this work with Ada Böhm, we have both contributed to it equally. I am the sole author of the design and implementation of
specific components of \hyperqueue{}, these will be marked as such in the rest of this chapter. While me and Ada are the primary
contributors to \hyperqueue{}, it should be noted that multiple other people have contributed to it, as its development is a team effort. Source code contribution statistics for \hyperqueue{}
can be found on GitHub\footnoteurl{https://github.com/it4innovations/hyperqueue/graphs/contributors}.}

%First, we will go through the meta-scheduling design, which will be defined in terms of its
%\hyperqueue{} implementation. Then we will examine the implementation details of
%\hyperqueue{}, analyze its performance and evaluate it using several use-cases. Lastly,
%\hyperqueue{} will be compared with other state-of-the-art meta-scheduling task runtimes.

% TODO: rename/restructure


\section{Interaction with allocation managers}
The primary challenge that affects the simplicity of task graph execution on \gls{hpc}
clusters is the presence of an allocation manager, as it forms a barrier; on such clusters, it is
not possible to compute anything at all without interacting with allocations. In order for task
graph execution to become truly ergonomic, there is no point in tackling the rest of the
(ergonomical) challenges until there is a straightforward way of executing task graphs through an
allocation manager.

\Autoref{ch:sota} has described several approaches that can be used to map tasks to
allocations, which is necessary to execute task graphs, and also their various shortcomings. Due to
the limits imposed by allocation managers, it might be necessary to partition task graphs into
multiple subgraphs in order to fit them within an allocation, which can be challenging.
Furthermore, it can result in non-optimal usage of hardware resources, because tasks from
(sub)graphs submitted in separate allocations will only be load balanced within their own
allocation, and not across different allocations.

It is important to understand what is uniquely challenging about the interaction with allocations.
The fact that the computation needs to go through a queue is not an issue by itself, as we are
dealing with batch processing anyway. Therefore, some form of a delay between submitting the task
graph and receiving the results is expected, even if we did not have to submit allocations at all.

The main issue of allocations is that they strictly tie together two separate aspects;
\emph{what} does the user want to compute (the computation performed once the
allocation starts) and \emph{where} should the computation take place (specific hardware
resources and computational nodes reserved for the allocation). As was already described
previously, both of these things have to be specified together in an allocation. This is very a
inflexible design, for several reasons:

\begin{itemize}
    \item The user needs to consider both aspects at the same time. Ultimately, the main thing that the user
    cares about is what do they want to compute. With allocations, they also need to think about
    specific hardware resources that should be allocated for computing their task graph.
    \item Hardware resources are allocated for the whole duration of the allocation. This can result in
    inefficient resource usage, especially for heterogeneous task graphs that consist of many types of
    tasks that use different resources. In situations where not all resources can be used at the same
    time, some of the resources can unnecessarily sit idle. For example, consider a task graph execuded
    inside an allocation that provides two computational nodes. Tasks will be load-balanced
    amongst these two nodes, but towards the end of the computation, there might be tasks that will
    take a long time to finish. This might lead to a situation where only one of the nodes will be
    computing the long tail of the remaining tasks, while the second node will be idle. Due to the fact
    that the allocation cannot release the second node while the computation is still ongoing, the user
    unnecessarily pays the cost of both nodes until the task graph finishes computing, even though
    the second node is not doing any useful work.

    \item The amount of used resources has to be decided up front, new hardware resources cannot be added nor
    removed from existing allocations. This can possibly cause inefficiencies, as described above, but
    it also limits load balancing. If balancing only happens within (and not across) allocations, users
    cannot easily provide new hardware resources during the computation, if they realize that it is
    progressing too slowly.
    \item Granularity of the allocated resources might not match the granularity of tasks. Unless the
    allocation manager supports very fine-grained allocations (e.g.\ on the level of individual
    \gls{cpu} cores, rather than whole computational nodes), there can be a large gap
    between the resources required by a task graph and the resources provided to an individual
    allocation. This can again lead to resources sitting idle.
    \item Binding computation with specific hardware resources up front complicates fault tolerance. When the
    user submits thousands of tasks on a specific set of hardware resources, and some of these tasks
    fail, the user might need to create a new allocation to recompute the failed tasks. This requires
    figuring out again a new set of hardware resources that should be requested, as the original
    resources will probably not be a good fit for just a subset of the original computation. A
    fault-tolerant design would ideally allow recomputing tasks on any compatible hardware resource
    transparently, which goes against forcing computations and resources to be defined together.
\end{itemize}

It is interesting to note that some of these challenges are uniquely affecting task graphs, and in
general programming models that are different from the traditional ways of defining
\gls{hpc} computations. Consider a distributed application implemented using
\gls{mpi}, which has historically been the most common way of defining computations
on supercomputers. \gls{mpi} applications typically assume that they will run on a set
of computational nodes for a relatively long duration (hours or even days). This set of nodes
(\gls{mpi} processes) usually does not change during the computation; it is not
trivial to add new processes, and if some of the processes crash, it typically leads to the whole
computation being aborted, as fault tolerance is not a default property of \gls{mpi}
applications~\cite{fault_tolerant_mpi}.

These properties of \gls{mpi} applications are similar to the mentioned properties of
allocations, therefore they fit together quite well, and using allocations to execute them is
relatively straightforward. In fact, it is clear that the allocation model itself was designed with
\gls{mpi}-like use-cases in mind. At the same time, it is also not very surprising
that the allocation model is not a good match for programming models that are very different from
\gls{mpi}, such as task-based workflows.

%Note that several of the problems associated with using allocations are directly relevant to
%various challenges from~\Autoref{ch:sota}. Factors such as heterogeneous resource
%requirements, fault tolerance or efficient hardware usage are affected by the usage of allocations.
%Because of that, resolving the problems with allocations should also go a long way towards
%alleviating other ergonomical challenges of task graph execution.


\section{Meta-scheduling design}
This section describes a design for executing task graphs that aims to alleviate the mentioned
challenges by treating task-based programming models as first-class citizens, but still remaining
compatible with allocation managers to enable straightforward execution on \gls{hpc}
clusters. First, we will see a general overview of the design, then we will discuss its properties,
and the following sections will describe its concrete implementation within the
\hyperqueue{} task runtime.

We have seen that the most problematic aspect of allocations is the requirement to define
computations and hardware resources together. The key idea of the proposed design is thus to remove this
property, by completely disentangling the definition of what does the user want to compute (tasks)
from the hardware resources where the computation should take place (computational nodes,
\gls{cpu} cores, etc.). By separating these two concerns, we enable users to focus on
what they care about the most (their task graphs), instead of having to think about mapping tasks to allocations
and graph partitioning.

But how can we disentangle these two concepts, when allocations require us to specify both? The
``trick'' lies in leveraging meta-scheduling on top of allocation managers, which allows us to
change what kind of computation is submitted in allocations and thus simplify the submission process.
The proposed approach is built upon the following three principles.

\begin{description}
    \item[Task runtime runs outside of allocations] Moving the task runtime outside of allocations and running
    it at some persistent location in the cluster (e.g.\ on a login node) enables users to submit tasks to it
    in a straightforward way and crucially also to define the task graph independently of allocations.
    This removes the need to decide which
    tasks should be computed on which hardware resources and in which allocations up front. It can also
    improve hardware utilization, because it gives the task runtime the ability to load balance tasks
    across all active allocations.

    This is where the term \emph{meta-scheduling} comes in; the essence of the idea is to use
    a task runtime as a sort of high-level scheduler on top of an allocation manager. Instead of
    forcing users to think in terms of allocations, they would be able to submit task graphs in the same way
    they would do it on a system that is not supervised by allocation managers, and let the task
    runtime decide in which allocation should a task be executed in an automatic way.

    \item[Allocations are uniform] Even if the task runtime runs outside an allocation,
    we still need to submit \emph{some} allocations to provide hardware resources that will actually
    execute tasks. However, instead of defining specific tasks that should be computed within an allocation,
    we can submit allocations that execute a generic computational provider (worker), which will connect to the task runtime
    instance running outside the allocation, and then execute any tasks dynamically assigned to it.

    With this approach, allocations become simple and completely uniform, because they all
    have the same structure; each allocation starts a computational provider. This makes it
    much easier for users to submit allocations. Instead of thinking about how to partition their task
    graphs, users simply decide how many computational resources they want to spend at any given moment,
    and start the corresponding number of allocations.
    Furthermore, this also allows the task runtime to submit allocations completely automatically based on
    computational needs, without the user having to do anything manually.
    \item[Pair tasks with workers using abstract resources] The principles described above separate the two core pillars of task graph execution; the
    definition of the task graph and the provisioning of hardware resources. However, we actually do
    need to have \emph{some} way of tying these two concepts together, because tasks might
    have various constraints on the environment in which they can be executed. While using allocations
    directly for submitting tasks provided this binding implicitly, we saw that it also had many disadvantages.
    What we can do instead is to use an abstract resource system. Each task will have a set of abstract resources
    required by it, and each worker will in turn provide a set of abstract resources provided by it.
    The task runtime will then dynamically assign tasks to workers by matching these resources together.

    As an example, a task can specify that it requires four \gls{cpu} cores and two \glspl{gpu}.
    If we provide the task runtime with a worker that has 128 cores and 8 \glspl{gpu},
    it could schedule up to four instances of this task on this worker at the same time.

    This approach allows defining complex heterogeneous tasks while still allowing users to provide workers independently
    of tasks.
\end{description}

The proposed design enables straightforward execution of large task graphs that contain dependencies and are
heterogeneous, because it sidesteps the allocation manager by moving the task runtime outside of allocations.
This avoids the limits associated with executing tasks in allocations directly, such as limited support for
dependencies, limited scale of task graphs that can be executed, and non-optimal load balancing and hardware utilization.
This scheme also directly alleviates additional challenges mentioned in~\Autoref{ch:sota}.

Achieving fault tolerance becomes easier, because tasks are no longer tied to specified allocations. Instead,
tasks are stored in a persistent task runtime instance. When a task fails, the runtime can simply recompute it, even
in a different allocation than where it was originally started. Heterogeneous task graphs can be executed without having to consider allocation limits, due to the usage
of abstract resources. And task scheduling becomes more flexible, since the runtime can dynamically use resources that are provided by
multiple allocations, which can help with improving hardware utilization.

Of course, this approach is not a panacea, and it also introduces certain trade-offs. Running the task runtime on a
login node might be problematic if the login nodes are severely computationally constrained or if they are not even
able to connect to compute nodes of the \gls{hpc} cluster through the network.
While the runtime could also be executed elsewhere, for example in a cloud partition of a cluster, or even inside
allocations themselves (in that case it would still communicate with workers running in different allocations),
although this would complicate the deployment process and might require additional resiliency features to be implemented
in the task runtime.

To evaluate how does the proposed approach work in practice, what are its performance and usage implications and how does
it compare to existing meta-schedulers, we have implemented this design in a task runtime called \hyperqueue{}, which
will be described in the rest of this chapter.

\section{\hyperqueue{}}
\hyperqueue{} is a distributed task runtime that aims to make task graph execution a first-class concept on supercomputers.
It does that by implementing the meta-scheduling design described in the previous section, and providing functionality optimized
for \gls{hpc} use-cases. In terms of architecture and implementation, it is an evolution of the \rsds{} task runtime described
in~\Autoref{ch:rsds}, however instead of using the \dask{} components and \glspl{api}, it implements its own task-based
programming model. \hyperqueue{} is developed in the Rust programming language and it is provided as
an \mbox{MIT-licensed} open-source tool~\cite{hq_github}.

This section describes the high-level design of \hyperqueue{}, its most important features
and also how it tackles the challenges mentioned in~\Autoref{ch:sota}. Note that for simplicity, Slurm will be
used as an example of an allocation manager throughout this whole chapter, but it could be replaced
by PBS or any other commonly used allocation manager without the loss of generality.

\subsection{Architecture}
\hyperqueue{} consists of three main components, a central \emph{server}, a set of computational providers called
\emph{workers} and a \emph{client} interface used by users to submit tasks.

\subsubsection*{Server}
The \emph{server} stores information about submitted workflows and manages the execution of tasks by scheduling them to available computational providers (\emph{workers}).
Workers based on their resource requirements and current computational load.

The \emph{worker} component manages

- connection to workers on a shared filesystem

\subsubsection*{Worker}

\subsubsection*{Client}
Client

\subsubsection*{Deployment}
- deployment
- local experimentation

\subsection{Scheduling and resource management}

\subsection{Output streaming}

\subsection{Interface}
- CLI
- Python API

\subsection{Automatic allocation submission}

\section{Case study: LiGen}

\section{Case study: GROMACS}

\section{Efficiency}

\section{Comparison with state-of-the-art}
%TODO: Merlin https://merlin.readthedocs.io/en/latest/tutorial/1_introduction/?h=many#how-can-merlin-run-so-many-simulations
%TODO: Balsam https://arxiv.org/abs/1909.08704
%TODO: FireWorks https://onlinelibrary.wiley.com/doi/10.1002/cpe.3505
%TODO: Parsl
%TODO: Dask

\section{Impact}
\hyperqueue{} has been adapted as a task execution backend in several workflow management
systems, such as Aiida~\cite{aiida-hq}, NextFlow~\cite{nextflow-hq} or ERT~\cite{ert}. It is
straightfoward to use it for this use-case, because it offers a comprehensive \gls{cli} that can output machine readable
(\gls{json}) output.

%\hyperqueue{} is being used by users of various \gls{hpc} centres, and it is also a key
%component of the Horizon 2020 European Union projects
%LIGATE\footnoteurl{https://www.ligateproject.eu},
%EVEREST\footnoteurl{https://everest-h2020.eu} and
%ACROSS\footnoteurl{https://across-h2020.eu}. It is also envisioned as one of the primary ways of
%executing computations on the LUMI supercomputer~\cite{lumi_it4innovations_2022}.

\section{Summary}
