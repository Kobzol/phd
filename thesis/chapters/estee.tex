The scheduler component, which assigns tasks to individual workers, is one of the
most crucial parts of a task runtime, because scheduling decisions can severely affect the
duration required to execute the whole task graph. Since task scheduling is NP-hard, various
heuristic algorithms are used in practice. These algorithms can suffer from non-obvious edge cases
that produce bad quality schedules and also from low runtime efficiency, which can erase any
speedup gained from producing a higher quality schedule.

To better understand the behaviour and performance of various scheduling algorithms, we have
performed an extensive analysis of several task scheduling algorithms in
\emph{Analysis of workflow schedulers in simulated distributed environments}~\cite{estee}.
We have benchmarked several task schedulers under various conditions, including parameters that
have not been explored so far, like the minimum delay between invoking the
scheduler or the amount of knowledge about task durations available for the scheduler.

Our analysis has shown that despite its simplicity, the foundational HLFET
algorithm~\cite{hlfet1974} produces high quality schedules in various scenarios and should thus
serve as a good baseline scheduler for task runtimes. We have also found out that even a
completely random scheduler can be competitive with other scheduling approaches for certain task
graphs and cluster configurations.

%During our attempts to implement various scheduling algorithms, we have also realized that the
%descriptions of many task scheduling algorithms in existing literature is incomplete. More
%specifically, seemingly inconsequential implementation details that are often missing from the
%algorithm's description can have a very large effect on the final performance of the scheduler,
%which makes it difficult to precisely reproduce and compare the performance of the existing
%algorithms.

One of the contributions of this work was \estee{}, a simulation framework for task
schedulers that is available as an open-source software\footnote{https://github.com/it4innovations/estee}.
It can be used to define a cluster of workers, connect them together using a configurable network
model, implement a custom scheduling algorithm and test its performance on arbitrary task graphs.
\estee{} also contains implementations of several task scheduler baselines from existing literature
and a task graph generator that can be used to generate randomized graphs with similar properties as
real-world task graphs. This tool serves as an experimentation testbed for task runtime and scheduler
developers.

I have collaborated on this work with Stanislav Böhm and Vojtěch Cima, we have all contributed
equally to this work.
