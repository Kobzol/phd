Even though task graphs are commonly being executed on distributed systems,
executing them on HPC clusters specifically presents unique challenges, both in terms of
efficient and scalable task execution, and of programmer productivity and ergonomics when
designing and prototyping HPC-scale task graphs.

%The main objective of the proposed thesis is to analyze how are task workflows executed on HPC
%clusters and what specific problems and bottlenecks exist in this area, and to design and develop
%tools and approaches that would improve the ergonomics and efficiency of task workflows in HPC.

%Since the primary goal of the proposed thesis is to improve HPC task graph execution, it is
%crucial
%to first identify the current bottlenecks that limit its potential. This section discusses some
%of the most important challenges and limitations of executing task graphs on HPC systems. These
%identified bottlenecks will also serve as objectives for further research in the thesis and will
%guide future work. In particular, the challenges will mention properties that should be upheld
%by a task runtime in order for it to properly support the mentioned HPC task graph requirements.

This section discusses several challenges of executing task graphs on HPC systems and also
use-cases that are fairly unique in the HPC environment. In order to enable truly ergonomic and
efficient usage of task graphs, task runtimes should support the mentioned use-cases and should
also be aware of the challenges, and ideally deal with them as best they could. This section will
thus also mention various properties and features that task runtimes should offer to deal with the
mentioned challenges.

\subsection{Job manager}
The vast majority of HPC systems use some kind of job manager (usually PBS/Torque~\cite{pbs} or
Slurm~\cite{slurm}) to facilitate job submission, resource management and project
accounting~\cite{slurm-schedmd}. To perform any computation using a job manager, the user has to
submit a job that describes how many nodes they want to allocate and what is the expected
runtime of their computation. The job is then submitted into the job manager \emph{queue} and
starts to execute only once there are enough free computational resources. Job managers are used to
provide fair access to HPC resources that avoids oversubscription and also to handle accounting
of computation. They tend to have fairly strict limits on the number of jobs that users can submit
and the amount of nodes that they can have reserved for their jobs at any given time.

To distinguish the often overloaded terms \emph{task} and \emph{job}, we will use the following
definitions in the rest of the text. The term \emph{task} will be used for description of a
computation that can be very fine-grained (e.g.\ a function call or an execution of a single
program), is composed with other tasks in a task graph and executed by a task runtime. The term
\emph{job} will be reserved for coarse-grained HPC jobs submitted to job managers (like Slurm or
PBS/Torque), which require whole computational nodes to execute and which tend to run for hours
or even days.

In a way, HPC job managers can also be viewed as task runtimes that operate on a very coarse
level -- their tasks are HPC jobs that can span hundreds of nodes, run for days or even longer and
consist of many various programs being executed. In theory, users could submit fine-grained task
graphs to these job managers, which could serve as a natural way for computing complex workflows on
HPC systems and thus facilitate their usage.

In practice, it is not feasible to use the current popular job managers (Slurm and PBS/Torque) in
this way, because they operate on a level that is far too coarse-grained for large and complex task
graphs. Their overhead for scheduling and executing a single job is orders of magnitude larger
than for typical task runtimes (seconds vs milliseconds), their support for dependencies between
tasks/jobs is very basic, and they consider computational nodes to be the primary units of
computation, which does not map nicely to fine-grained tasks that can leverage only a part of a
node (e.g.\ only a few cores).

This creates a certain dichotomy between the coarse-grained job manager and (usually a
fine-grained) task runtime, and instead of facilitating simple usage of HPC clusters, it creates a
barrier for users. Instead of building a task graph of the whole computation and executing it with
a single command, they have to think about how to map the task graph to HPC jobs in order to be
able to execute their tasks on an HPC cluster and also to amortize the overheads of the job
manager.

It should be noted that even though there is definitely room for improving the performance of
HPC job managers, some of their complexity and performance limitations are inherent. They have
to provide accurate accounting, handle robust and secure cleanup of jobs, take care of job and
process isolation, ensure user fairness and many other things. Many of these responsibilities are
out of scope for task runtimes, which enables them to achieve higher performance.

There are several approaches that can be used to map task graphs to HPC jobs.
Here are some examples of these approaches, ordered from the simplest to the most complicated:

\begin{description}
    \item[Execute the whole task graph in a single job] If the task graph is small enough, it
    could be executed in a single job, which would mostly be as simple for the user as executing
    the task graph on a non-HPC cluster or a personal computer. However, since jobs are bound
    both by node counts and strict time limits, this approach will only be usable for rather
    small task graphs. Indeed, if the computation is small, it probably does not even make sense
    to even use an HPC cluster to compute it. A more realistic scenario is that the user has a
    large number of small task graph to execute, but this situation corresponds to a large task
    graph that consists of many disjoint components (subgraphs).
    \item[Execute each task as an individual job] While this is certainly tempting, since this
    approach is mostly straightforward to implement using existing job managers, it is
    impractical because of the mentioned difference in granularity between tasks and jobs.
    Job managers have an enormous overhead per each job, and furthermore they seldom allow the
    user to create more than a few hundreds of jobs at the same time, both to provide fairness and
    also because they simply cannot scale to such amount of jobs. Furthermore, a single job
    often has to span at least a (complete) single node, and if a single task cannot leverage a
    whole computational node, this would lead to wasting resource.
    \item[Split the task graph into a smaller amount of jobs] This is the ultimate approach that
    the
    user has to resort to if their task runtime does not provide any special support for job
    managers. The task graph has to be split into smaller parts which will then be executed in
    individual jobs. In addition to manually splitting the task graph, additional infrastructure
    has to be implemented, for example to store the intermediate results of the computed tasks
    before the jobs ends, to merge the intermediate results from multiple jobs and also to
    periodically submit new jobs until the whole task graph is computed.

    This reduces the ergonomics of using task graphs, because it basically forces the user to
    reimplement part of the task runtime behavior to overcome the limitations of job managers.
    Splitting the task graph into a fixed amount of jobs also has the disadvantage that the
    individual tasks cannot be load balanced across jobs, even if multiple jobs run
    concurrently, because each job will simply execute its own separate copy of some task
    runtime that will execute a part of the task graph.
\end{description}

This dichotomy creates a large gap for users attempting to scale their task graph
computation. Running on a personal computer tends to be quite simple. After that, moving to an
HPC cluster and executing the entire task graph inside a single job is also quite
straightforward. But once the task graph has to be split into multiple jobs, the nice
abstraction of implicitly parallel task graphs that can be executed with a single command
quickly falls apart, as the user has to perform a lot of additional work to make this scenario
work efficiently.

Ideally, the users wouldn't have to deal with the job manager; they should be
able to construct a task graph and execute it directly on an HPC cluster in a straightforward
way. This could be achieved either by adding support for executing fine-grained task graphs to
job managers or by integrating task runtimes with job managers to provide transparent execution
of task graphs on HPC systems.

\subsection{Cluster heterogeneity}
In recent years, HPC clusters have started to become increasingly heterogeneous.
Individual cluster nodes contain varying amounts of cores, memory, NUMA nodes or accelerators
like GPUs or FPGAs. This complexity also propagates to task definitions and their requirements.
Some tasks can be single-threaded, some multithreaded, some are offloadable to accelerators if
there is one available, some can only run on accelerators, while some of them can only execute
on CPUs.

To uphold these task requirements, it should be possible for users to define fine-grained
task resource requests (e.g.\ "this task requires two GPUs, sixteen cores and at least 32 GiB of
memory"). To match these requests, it should be possible to attach arbitrary resources to
computational providers (workers). The task runtime should then manage the dynamic resource
allocations of workers to individual tasks to make sure that tasks only execute on workers that
have enough resources.

A unique resource requirement that is fairly specific to HPC systems is the requirement of using
multiple nodes (workers) per a single task. This requirement is necessary for executing programs
that are designed to be executed in a distributed fashion, such as programs using MPI, which are
quite common in HPC software. This use case is discussed further below.

\subsection{Data transfers}
After a task is computed, it can produce various data outputs, standard error or output
streams, files created on the disk or data objects that are then passed as inputs to dependent
tasks. There are many ways of storing and transferring these outputs. Some task frameworks store
task outputs on the filesystem, since it is relatively simple to implement, and it provides
support for basic data resiliency out-of-the-box.

HPC nodes often do not contain any local disks, instead they tend to use shared filesystems
accessed over a network. While this might be seen as an advantage, since with a shared filesystem
it is much easier to share task outputs amongst different workers, it can also be a severe
bottleneck. Shared, networked filesystems can suffer from quite high latency, and accessing them
can consume precious network bandwidth that is also used e.g.\ for managing computation
(sending commands to workers) or for direct worker-to-worker data exchange.
Furthermore, data produced in HPC computations can be quite large, and thus storing it to a disk
can be a bottleneck even without considering networked filesystems.

It should be possible to alleviate these bottlenecks, for example by directly transferring task
outputs between workers over the network (preferably without accessing the filesystem in the
fast path), by streaming outputs between tasks without the need to store them or by leveraging
RAM disks. Making use of HPC specific technologies, such as MPI or InfiniBand, could be also
worthwhile to leverage the very fast interconnects available in HPC clusters.

Data outputs produced by tasks tend to be considered immutable, since a single output can be
used as an input to multiple tasks, and these might be executed on completely different
computational nodes. A problem that can arise with this approach is that if the data outputs are
large, but the computation within tasks that work with the data is short, the serialization
overhead
(or even memory copy overhead, if the dependent task is executed on the same node) starts to
dominate the execution time. To support such scenarios, some support for stateful data management
can be useful, for example in the form of \emph{actors}, which can be considered stateful tasks
that
operate on a single copy of some large piece of data.

\subsection{Fault tolerance}
Some level of fault tolerance should be provided by all task runtimes, but HPC systems have
specific requirements in this regard. As was already mentioned, computational resources
(computing nodes) on HPC clusters are provided through job managers. These node allocations have
a temporary duration, which means that for long-running task graphs, workers will disconnect and
new workers will connect dynamically during the execution of the task graph. Furthermore, since
the job manager allocations go through a queue, it can take some time before new computational
resources arrive, therefore the task graph can remain in a paused state where no tasks are being
executed, for potentially long periods of time.

Task runtimes should be prepared for these situations; they must handle worker disconnection
gracefully, even if that worker was currently executing some task, and they should be able to
restart previously interrupted tasks on newly arrived workers. In HPC scenarios, worker
instability and frequent disconnects should be considered a common behaviour, not just a rare
edge case.

\subsection{Multi-node tasks}
Many existing HPC applications are designed to be executed on multiple (hundreds or even
thousands) nodes in parallel, using MPI libraries or other communication frameworks. To properly
support these use-cases, we can introduce multi-node tasks by creating a special resource
requirement, which states that a task should be executed on multiple workers at once.

Support for multi-node tasks affects many design areas of a task runtime:
\begin{description}
    \item[Scheduling] When a task requires multiple nodes for execution and not enough nodes are
    available at a given moment, the scheduler has to decide whether it should plan additional
    tasks on the currently available nodes to maximize utilization or if it should keep the nodes
    idle to enable the multi-node task to start as soon as possible.
    In a way, this decision-making already has to be performed on the level of individual cores
    even for single-node tasks, but adding multiple nodes per task makes the problem even more
    difficult.
    \item[Data transfers] It is relatively straightforward to express data transfers between
    single-node tasks in a task graph, because they naturally correspond to dependencies (edges)
    between the tasks. With multi-node tasks, the situation becomes much more complicated. Data
    distribution strategies have to be introduced to allow replicating data from a single node
    to multiple nodes when a multi-node task starts, and to gather (reduce) the results from
    multiple nodes to a single node when such task finishes.

    When several multi-node tasks depend on one another, the task runtime should be able to
    exchange data between them in an efficient manner. This might require some cooperation with
    the used communication framework (e.g.\ MPI) to avoid needless repeated serialization and
    deserialization. The process becomes even more complicated if the individual multi-node tasks
    have different node counts.
    \item[Fault tolerance] When a node executing a single-node task crashes or disconnects from
    the runtime, its task can be rescheduled to a different worker. In the case of multi-node
    tasks, the situation becomes more complicated. When a task is executing on four nodes and
    one of them fails, the runtime has to make sure that the other nodes will be notified of
    this situation, so that they can react accordingly (either by finishing the task with a
    smaller amount of nodes or by failing immediately).
\end{description}

To enable common HPC usecases, task runtimes should be able to provide some support
for multi-node tasks and allow them to be combined with single-node tasks. Advanced multi-node
task support could be provided e.g.\ by offering some kind of integration with MPI or similar
common HPC technologies.

\subsection{Scalability}
The sheer amount of HPC performance (node count, core count, network interconnect bandwidth)
opens up opportunities for executing large scale task graphs, but that in turn presents
unique challenges for task runtimes. Below you can find several examples of bottlenecks that
might not matter in a small computational scale, but that can become problematic in the context of
HPC-scale task graphs.

\begin{description}
    \item[Task graph materialization] Large computations might require building massive task
    graphs that contain millions of tasks. The task graphs are typically defined and built
    outside the task runtime itself, for example on the login nodes of computing clusters or on
    client devices (e.g.\ laptops), which can provide only relatively low performance. It can be
    quite slow to build, serialize and transfer such graphs over the network to the task
    runtime. This can create a bottleneck even before any task is executed. This has been
    identified as an issue in existing task runtimes~\cite{dask-client-perf}.

    In such case, it can be beneficial to provide an API for defining task graphs in a symbolic
    way, for example by representing a potentially large group of similar tasks by a single
    entity. Such symbolic graphs could then be sent to the runtime in a
    compressed form and re-materialized only at the last possible moment. In an extreme form,
    the runtime could operate on such graphs in a fully symbolic way, without ever
    materializing them.
    \item[Communication overhead] Scaling the number of tasks and workers will necessarily put a
    lot of pressure on the communication network, both in terms of bandwidth (sending large task
    outputs between nodes) and latency (sending small management messages between the scheduler
    and the workers). Using HPC technologies, such as MPI or a lower-level interface like
    RDMA (Remote Direct Memory Access), could provide a non-trivial performance boost in this
    regard.

    As we have demonstrated in~\cite{spin, spin2}, in-network computing, an active area of
    research, can be also used to optimize various networking applications by offloading some
    computations to an accelerated NIC (network interface controller). This approach could also
    be leveraged for task runtimes, for example by reducing the latency of management messages
    between the scheduler and workers or by increasing the bandwidth of large data exchanges
    amongst workers, by moving these operations directly onto the network card.
    \item[Runtime overhead] As we have shown in~\cite{rsds}, task runtimes with a centralized
    scheduler have to make sure that their overhead remains manageable. Even with an overhead of
    just $1ms$ per task, executing a task graph with 1 million tasks would result in total
    accumulated overhead of twenty minutes! Our results indicate that increasing the performance
    of the central scheduling and management component of a task runtime can have a large positive
    effect on the overall time it takes to execute the whole task graph.

    However, the performance of the central server cannot be increased endlessly, and from some
    point, using a centralized architecture, which is common to task runtimes, itself becomes a
    bottleneck. Even if the workers exchange large output data directly between themselves, any
    single, centralized component may become overloaded simply by coordinating and scheduling
    the workers.

    In that case, a decentralized architecture could be leveraged to avoid the reliance on a
    central component. Such a decentralized architecture can be found e.g.\ in Ray~\cite{ray}.
    However, to realize the gains of a decentralized architecture, task submission itself has to
    be decentralized in some way, which might not be a natural fit for common task graph workflows.
    If all tasks are generated from a single component, the bottleneck will most likely remain
    even in an otherwise fully decentralized system.
\end{description}

\subsection{Iterative computation}
A natural way of executing task graphs is to describe the whole computation with a single task
graph, submit the graph to the task runtime and wait until all the tasks are completed. However,
there are some computations that need a more iterative approach. Training a machine
learning model can be stopped early if the loss is no longer decreasing. A chemical or physical
simulation is only considered completed once a desired accuracy has been reached, which might
take a previously unknown number of steps. These scenarios, and many others like them, are quite
common in HPC use cases.

To support iterative computation, task runtimes should allow the user to stop the execution of a
task graph (or its subgraph) once a specific condition is met, and also to add new tasks to the
task graph in a dynamic fashion, if it is discovered that more iterations are needed.

\subsection{Summary}
Even though more HPC use-cases and oddities could always be found, it is already clear from the
mentioned challenges that HPC use-cases that leverage task graphs can contain a lot of complexity.
It could be possible to add support for some of the mentioned requirements to existing task
runtimes, which are described in the next section. However, the described challenges are so
diverse and complex that a dedicated approach which considers them holistically could provide a
better solution that would avoid both ergonomics and performance from being compromised.

The aforementioned challenges and use-cases will serve as a basis for further research in the
proposed thesis. The goal of the thesis is to design solutions for executing task graphs on HPC
systems that are aware of the mentioned use-cases and are able to resolve them efficiently.
These solutions will be implemented in a task runtime called \hyperqueue{}, which is described
further in Section~\ref{subsec:hyperqueue}.
