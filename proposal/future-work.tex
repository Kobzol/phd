The research related to this thesis that has been conducted so far has been mostly focused on
analysing task schedulers and task runtime bottlenecks, which are both part of the
challenges mentioned in Section~\ref{sec:challenges}. In terms of the areas discussed in
Section\ref{sec:introduction}, these concepts fall into the area of efficiency.

In future work, I plan to also focus on the ergonomics side of HPC task graph execution,
primarily on overcoming the issue of integrating task graph execution with HPC job managers. As
a step in this direction, I have been working on the automatic allocator of \hyperqueue{}, which
I plan to further develop and analyse.

In alignment with the mentioned thesis objectives, the future work will focus on designing ways
of overcoming some challenges of task graph execution on HPC systems mentioned earlier and
use \hyperqueue{} as the integration component where these ideas would be tested and applied in
real-world HPC scenarios.
