The research related to this thesis that has been conducted so far has been mostly focused on
analysing task schedulers and task runtime bottlenecks, which are both part of the
challenges mentioned in Section~\ref{sec:challenges}. In terms of the areas discussed in
Section\ref{sec:introduction}, these concepts fall into the area of efficiency.

In future work, I plan to also focus on the ergonomics side of HPC task graph execution,
primarily on overcoming the issue of integrating task graph execution with HPC job managers.

\hyperqueue{} is a step in this direction, since it already makes the execution
of task graphs on HPC clusters simpler. I have designed and implemented an \emph{automatic allocator}
into \hyperqueue{} that is able to ask HPC job managers for computational resources automatically,
depending on the current computational load. This frees users from interacting with the job manager
directly and moves one step closer to a straightforward execution of task graphs on HPC systems.
In future work, I plan to improve automatic allocation and better understand its behavior in
various HPC use-cases, and also to design new approaches for resolving the mentioned HPC challenges
and apply them to real-world HPC scenarios while leveraging \hyperqueue{}.
