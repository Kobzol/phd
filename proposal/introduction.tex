High Performance Computing (HPC) infrastructures are crucial for the advancement of scientific
research, as they offer unparalleled computing resources that can be leveraged to perform the most
complex scientific experiments. Over the last several decades, the performance of HPC clusters
has been increasing steadily, effectively doubling every few years. However, such gains are not
always for free. One notable cost associated with the growth of HPC performance is its
increasing complexity. Computing nodes now consist of hundreds of cores, making it quite
challenging to write programs that can scale to such high core counts. They contain hundreds of
gigabytes of memory, which necessitates non-uniform memory architectures (NUMA) that require
specialized programming techniques to achieve optimal performance. HPC clusters are also becoming
increasingly heterogeneous, with devices such as graphics accelerators (GPUs) or reconfigurable
hardware (FPGAs) providing most of their (theoretically attainable) performance. Reaching the
full potential of these accelerators requires using further specialized programming languages
and interfaces.

Historically, optimized HPC software was usually written using system or scientifically focused
programming languages (e.g.~\texttt{C}, \texttt{C++} or \texttt{Fortran}) and specialized libraries
for parallelizing and distributing computation (such as OpenMP and MPI)~\cite{mpistudy}. While
these rather low-level technologies are able to provide the best possible performance, it can be
quite challenging and slow to develop (and maintain) applications that use them effectively. And
with the advent of heterogeneous clusters, even more technologies (such as \texttt{CUDA} for GPUs
or \texttt{HLS} for FPGAs) have to be adopted. It is unreasonable to expect that most scientists
that use HPC resources (who are often not primarily software developers) will be able to use these
low-level technologies in an efficient manner without making the development process too slow
and cumbersome.

Increasingly powerful hardware enables more complex computations, which are in turn more
demanding to develop. Areas such as weather prediction, machine learning model training or big
data analysis require launching thousands or even millions of simulations and experiments and
require a lot of prototyping during development.
These experiments can be quite complex, consisting of multiple dependent steps, such as data
ingestion, preprocessing, computation, postprocessing, visualisation etc. It is thus imperative for
scientists to have a quick way of prototyping these applications, otherwise the development
would just be too slow.

That is why in recent years, HPC users are increasingly moving towards task-based programming
paradigms. These paradigms allow users to focus on the problem domain, quickly prototype and
describe computation that consists of a large number of individual steps. Using a task-based
approach, computations are described using a set of atomic computing blocks (\emph{tasks}) that
are composed together in a \emph{task graph} which captures the dependencies between the
individual tasks. Using task graphs abstracts away most of the complexity of network
communication and parallelization and allows the user to express their computation in a simple
way. Combined with the fact that task-based tools often provide interfaces written in high-level
languages, such as Python, or various domain-specific languages (DSLs), it makes them an ideal
tool for rapid scientific prototyping.

While task graphs are already commonly being used and deployed on various distributed systems,
there are certain challenges that limit their development productivity and scalability
when deployed on HPC systems specifically. They stem from the interaction with HPC job managers,
various quirks commonly present in HPC clusters or from the sheer computational scale that is
possible thanks to vast resources provided by HPC systems. Removing or alleviating some of those
challenges could lower the barrier of entry and make task graph execution better suited for
various HPC use-cases.

The proposed thesis aims to improve task graph execution in HPC by focusing on two important areas,
namely developer ergonomics and efficient hardware utilization. The primary goal is to enable
computing task graphs on HPC systems in a way that is simple and comfortable for the user, supports
essential HPC-specific use-cases and performs in an efficient manner. To achieve this goal, the
thesis has the following objectives:
\begin{itemize}
    \item Identify challenges of task graph execution on HPC systems.
    \item Design approaches for enabling task graph execution
    on HPC systems in a way that is both ergonomic for the user and also resource-efficient.
    \item Implement and integrate the designed approaches in a task graph execution tool to
    validate them in real world usage.
\end{itemize}

The thesis proposal is structured as follows. Section~\ref{sec:task_graph_definition} introduces
task graphs. Section~\ref{sec:challenges} describes current challenges of task graph execution on
HPC systems. Section~\ref{sec:state_of_the_art} mentions existing task runtime systems and
related works. Section~\ref{sec:current_progress} enumerates what has been achieved so far.
Section~\ref{sec:future-work} summarizes the thesis proposal and outlines future work.
